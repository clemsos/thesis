\section[Discussions]{Discussions}

%Avec prudence ( tu n{\textquoteright}es qun gravillon qui prétend s{\textquoteright}ajouter au Grand Mur de la science) . Avec quels biais éventuels, quelles limites, comment tu aurais pu mieux faire (apprends à battre ta coulpe, imagines toi à une séance d{\textquoteright}autocritique pendant la RévoCul).


L{\textquoteright}analyse et la visualisation de données des réseaux  sociaux nous a permis de constituer une série de figures représentant une série de mèmes ayant circulé sur le service de réseau social chinois \textit{Sina Weibo}. A l{\textquoteright}aide d{\textquoteright}un outil d{\textquoteright}analyse et de visualisation spécialement con\c{c}u pour l{\textquoteright}occasion, nous avons extraits leurs \textit{topogrammes}. La représentation des dimensions sémantiques, conversationnelles, temporelles et géographiques sous forme de réseaux nous permet d'explorer de comprendre et comparer les structures de leurs diffusions. La dizaine de mèmes observés nous a permis de dresser une première list des spécificités de diffusion pour chacun des types de mèmes : comique, commercial, publicitaire, actualité, faits divers et scandale politique.

En rappelant chacune des hypothèses formulées au début de ce travail (voir section \ref{sec:hypotheses}), nous allons maintenant proposer une lecture critique des résultats obtenus et des démarches mises en œuvre lors de la réalisation de ce travail. 

\subsubsection{Le modèle chinois pour les industries culturelles en Chine} 

Hypothèse : \textit{La majorité des contenus circulant sur les réseaux sociaux s'apparentent largement à ceux des médias traditionnels} 

La première partie de l{\textquoteright}analyse de données portant sur les hashtags les plus mentionnés nous a permis de valider partiellement notre deuxième hypothèse en constatant que les contenus majoritairement discutés sur \textit{Sina Weibo} étaient largement similaires à ceux des médias traditionnels : divertissements, produits culturels, trafic routier, sports, etc. Néanmoins, nous avons aussi pu observer que les hashtags étaient la plupart du temps un artefact de campagnes de communication organisées, reflétant un usage des réseaux sociaux pour le marketing promotionnel politique, médiatique ou commercial. Cette analyse des hashtags montre bien que les contenus issus des médias traditionnels se retrouvent largement sur \textit{Sina Weibo}.

En Chine, l'intégration très rapide des réseaux sociaux à la chaîne des médias traditionnels a été rendue possible par un ensemble de conditions. La première est le protectorat économique qui a été mis en place sur tout le secteur des médias, lié à la fois à la volonté de contrôle politique mais surtout à la volonté économique des acteurs du pays. Comme nous l'avons montré (chap. \ref{sec:internet-chine}), le développement des TIC et de l'Internet a été dès le début conçu par Pékin comme un des piliers chargé de soutenir la croissance économique du pays. Ce protectionnisme sur tout le secteur des industries culturelles a permis l'apparition rapide de nombreuses firmes qui ont bénéficié d'un marché captif et non soumis à la concurrence internationale. Les opportunités de verticalisation ont donc été plus nombreuses. 

La nécessité de contrôle de la ``qualité'' des informations publiés par ces services a également accéléré cette intégration entre producteurs et distributeurs de contenus. Les fournisseurs et diffuseurs de contenus chinois sont soumis à des règles strictes de contrôle politique. Ils doivent en effet s'assurer que les contenus qu'ils publient sont ``en harmonie'' avec les règles et tendances définies par le gouvernement de Pékin. La gestion de ce risque a rendu indispensable la mise en place d'un contrôle continu des informations publiées. De ce fait, l'intégration verticale entre producteurs de et distributeurs de contenu s'impose comme la solution la plus logique - la moins coûteuse. Les processus de contrôle de l'information et la gestion des ``risques politiques'' liés à la publication ont donc favorisé l'intégration des réseaux sociaux au cycle de production des médias traditionnels.

La perspective de gérer la conformité des contenus aux législations et à l'agenda politique d'un gouvernement peut paraît au premier abord une spécificité de l'Internet chinois. Les termes du débat et le cadre légal qui régit les relations entre les entreprises privées et les gouvernements sont bien différents dans le cas des firmes américaines comme \textit{Facebook} ou \textit{Twitter}. Néanmoins, la gestion des aspects légaux de la possession et de l'usage des données utilisateurs devient une compétence de plus en plus centrale des services de réseaux sociaux. Culminant autour de la récente ``affaire Snowden'' \citep{Greenwald2013}, les tensions autour des questions de ``vie privée'' (\textit{privacy}) sur Internet viennent notamment faire pression pour davantage de responsabilité légale de la part des services web. Également, l'inégalité de traitement des flux de données est au centre des débats législatifs et managériaux avec la question de la neutralité des réseaux \citep{Schafer2011}. La pression grandissante pour légiférer et faire appliquer les décisions légales auprès des firmes Internet va nécessiter un contrôle accrue des différents maillons de la chaîne de production d'information. Ainsi, on peut gager que l'intégration verticale sera également une des premières solutions pour rentabiliser les coûteux dispositifs de contrôle à mettre en place. Comme dans le cas de Sina Weibo et des firmes de l'industrie médiatique en Chine, la pression légale et politique pourrait amener à une intégration des services de production et de diffusion de contenus.

Publicité, radio, télévision, le secteur des industries culturelles est historiquement sujet à une très forte concentration \citep{Martel2010}. Les grands groupes médiatiques fonctionnent selon ce principe d'intégration verticale, diversifiant même largement leurs activités à d'autres secteurs. Comme nous l'avons vu avec \textit{Sina Weibo}, les informations publiés par médias \textit{online} et \textit{offline} sont de plus en plus homogènes et sont quotidiennement repris. La dichotomie entre ``web'' et ``traditionnels'' s'estompe rapidement, avec le partage quotidien d'articles de journaux ou d'émissions de télévision sur la Toile. Les canaux de distribution eux-mêmes se recentre autour du format mobile. Télévision web,  web documentaire, data journalisme, ces pratiques se rejoignent et les services de réseaux sociaux jouent souvent le rôle de diffuseur principal. Dans cette dynamique, le rapprochement entre \textit{pure players} du web comme Twitter ou Facebook et les firmes médiatiques traditionnels ne paraît pas impensable. Cette tendance se voit notamment très clairement dans la nécessité de rentabiliser au maximum le placement d'information pour l'obtention de revenus publicitaires sur des écrans toujours plus petits. Les dispositifs de captation de l'attention que sont les services de réseaux sociaux deviennent la clé de voûte de l'industrie médiatique. 

Ainsi, l'histoire récente des médias chinois et notamment celle de Sina Weibo ne doit pas être lue seulement dans les termes d'une lutte politique entre les utilisateurs et le gouvernement de Pékin. La dimension stratégique du développement industriel des médias en Chine apportent aujourd'hui des clés de lecture éclairantes non seulement pour le pays, mais pour le monde entier. L'équilibre entre contrôle politique, réussite commerciale et développement industriel crée un précédent à une échelle nouvelle pour l'Internet qui ne doit pas être minimisé. L'intégration forte entre entités gouvernementales, fournisseur de contenus et canaux de distribution de l'information propose également un modèle inédit. Les études sur l'Internet chinois doivent s'intéresser davantage aux détails des mécanismes et relations entre ces différents acteurs. Cette relation d'apparence fusionnelle des différents pouvoirs dans le champ médiatique pourrait même être considérée comme une forme de maturité de l'industrie, acquise lors d'un développement hyper-rapide permis par un environnement protectionniste et autoritaire. En effet, les  discussions entourant le cadre législatif et les lois cherchant à réguler l'usage et la gouvernance d'Internet (\textit{Hadopi} en France, \textit{SOPA} aux États-Unis, \textit{Data Protection Directive} de l'Union Européenne) tendent à une plus forte intégration des pouvoirs dans la sphère médiatique dans le but d'une gestion des risques politiques. L'expérience de l'Internet chinois établit dans le le réseau mondial un modèle  radicalement différent de celui imaginé initialement par les fondateurs du réseau ARPANET. S'il est possible de l'écarter comme un archaïsme face au discours du \textit{new-age} des géants californiens, son existence a néanmoins établi un mode de développement et de gouvernance qui s'est jusqu'ici montré d'une efficacité et d'une rentabilité sans faille et pourrait bien dans le futur séduire davantage d'états et de gouvernements.

\subsubsection{Plate-formes web et lieux d'énonciation}

Cette dimension infra-structurelle de l'Internet n'est bien sûr qu'une seule des facettes qui entourent la production de l'espace web. L'immense diversité des usages et les nombreuses actions quotidiennes des utilisateurs sont l'autre pilier de ce large développement industriel. Les services de réseaux sociaux peuvent être analysé selon deux activités majeures qui s'entrelacent constamment : la lecture et consultation de ressources médiatiques (textes, images, vidéoas, photos, etc.) et l'écriture (liée à la discussion et la conversation). La rencontre de ces deux usages au quotidien donne lieu à de multiples phénomènes, portant notamment la pratique du commentaire vers des sphères nouvelles.

Les analyses machines préliminaires de notre corpus de données pour l'identification de mèmes ont montrés la difficulté d'isoler chacune de ces pratiques de façon satisfaisante. Le mème Internet se situe en effet à l'exacte croisée de ces deux usages, utilisant la déclinaison d'un support médiatique comme fondement de la conversation. Ce statut particulier du mème rend son approche complexe mais en fait également un objet passionnant. 

Hypothèse : \textit{Les mèmes sont à comprendre comme des actes d'énonciation.}

Les analyses de notre corpus nous ont permis de montrer que les hashtags ne recouvraient pas le statut particulier que nous attribuons ici au mème (voir section \ref{sec:hashtags}). Objet hybride, le hashtag reste avant tout un artefact exogène à la conversation, plutôt qu'un support discursif \textit{per se}. L'irruption du caractère dièse \textit{(\#)} dans le langage témoigne de ce caractère peu naturel. Même s'il tend à se généraliser, il appartient encore au domaine du dialecte propre à des plate-formes spécifiques. Son usage nécessite une appropriation, souvent par jeu de connotation ou de dénotation, et de ce fait participe d'une sous-culture des plate-formes web. A l'opposé, le mème ne se construit pas comme artefact remplissant une fonction annexe dans la conversation (identifier un événement, un lieu, une marque) mais bien comme une discussion en soi. Comme nous avons pu le constater de manière expérimentale (voir chapitre \ref{sec:id-meme}), identifier un mème dans un ensemble de conversations est une tâche difficile. Les approches algorithmiques ne peuvent s'appuyer sur des constructions en-dehors du langage et sont donc inefficaces (voir \ref{sec:protomemes}). Finalement, c'est l'entrée par le mot lui-même qui permet d'accéder à la représentation la plus fidèle du mème (voir \ref{sec:keywords}). Le régime particulier de l'énonciation que nous avons choisi pour appréhender l'existence des mèmes nous permet de considérer différents aspects de la conversation. Ce faisant, nous mettons également à jour la limite de notre méthodologie fondée sur l'analyse de données, à jamais incapable de voir l'action ou l'intention qui a motivé un acte de parole en ligne. 

Nos résultats semblent avant tout indiquer que plusieurs régimes d'expression et de conversation co-existent sur les réseaux sociaux. Les mèmes humoristiques semblent mieux correspondre au régime de l'énonciation. Proche de l'humour, les mèmes semblent définir des lieux intimes au sein du vaste espace d'expression qu'offre les plate-formes Web. A l'inverse,les campagnes de marketing ou de communication procèdent plus au régime du discours, cherchant à territorialiser des domaines précis notamment par la création de signes nouveaux (les hashtags par exemple, équivalent des marques du Web). Ces deux régimes co-existent largement au quotidien et ne sont pas nécessairement exclusifs. C'est dans la rencontre de ces différents modes d'expression que se nichent sûrement les approches les plus intéressantes de l'analyse des échanges en ligne.

Hypothèse : \textit{L'analyse de la diffusion doit mobiliser les modèles étudiant l'énonciation} 

Les modèles à construire de l'analyse de données doivent donc s'inspire du dialogue mais aussi prendre en compte les parties manquantes ou incomplètes lors de l'analyse. Les actes de communication observables sur les réseaux sociaux existent dans une continuité physique, sont le produit de corps qui parlent ou écrivent. Le statut de trace que constitue le texte écrit sur les réseaux sociaux donne à la conversation une nouvelle matérialité. Considérer de façon autonome l'existence textuel des messages échangés peut être intéressant dans une étude linguistique. Néanmoins, dans le cadre d'une étude sur la communication, il paraît frauduleux de réifier la conversation en un simple échange textuel. Également, considérer uniquement les conversations comme des ``flux'' entre utilisateurs échoue à prendre en compte la pré-existence de leurs relations et du contexte sur l'acte de communication. 

Le système de visualisation que nous avons choisi de développer voulait mettre en perspective les différentes dimensions de la conversation dans un même espace visuel (voir section \ref{sec:viz}). L'analyse d'un' échantillon de contenus issus de Sina Weibo nous a permis de vérifier partiellement notre hypothèse méthodologique. En donnant à voir différents aspects des échanges et conversations, nous avons cherché à observer les phénomènes de diffusion en prenant en compte quoi, qui, comment et où ce qui est dit. Le développement d'un dispositif technologique sur-mesure nous a permis d'obtenir des visualisations répondant à ces besoins. Néanmoins, la disparité des données ne nous a pas permis une granularité suffisante pour observer de façon détaillée chacun de ces aspects. Toutefois, nous avons pu montrer une image générale et observer que chacun d'eux posséder des particularités. Les différentes figures obtenues que nous nommons \textit{topogrammes} montrent plusieurs spécificités pour chaque type de contenus qui encouragent à poursuivre vers une plus grande diversité d'approche dans l'analyse des échanges en ligne.

Hypothèse : \textit{L'analyse et la visualisation de données permette d'établir une classification des discussions en ligne selon les structures de leurs diffusions.}


La lecture des différents topogrammes nous permet d'ébaucher une première typologie des contenus en ligne selon les modalités de leurs diffusions (voir fig. \ref{fig:viz-results}). Néanmoins, notre étude à dimension expérimentale s'intéresse à un nombre restreint de mèmes. La première validation notre hypothèse méthodologique nécessiterait une systématisation à des corpus plus diversifiés afin de confirmer s'il est réellement possible de constituer une typologie générale et cohérente des \textit{topogrames} des différents contenus disponibles en ligne. Nos premières observations permettent néanmoins de tirer des conclusions provisoires sur les spécificités des différents types de contenus.

La diffusion des faits d{\textquoteright}actualité est entourée de grands volumes de discussion, qui lui permette de se propager rapidement. Dans le cas du fait divers, son démarrage brutal puis sa transformation par le commentaire en débat de société parfois brûlants se caractérise par une un grand nombre de communautés très actives dans la discussion pendant un laps de temps plutôt court. La conversation s'oriente généralement autour de termes simples et structurés selon des chemins assez typiques. Les conversations autour de faits d'actualité sont au début composées de communautés diverses. L'enjeu pour le contrôle de la discussion est de s'approprier le cœur de la diffusion, souvent en centralisant et fédérant l'ensemble des conversations. Cette stratégie qui cherche à couper cours à une diffusion non-programmée de l'information est particulièrement présente et étudiée en Chine. Une des stratégies est notamment de relayer et de diffuser le maximum d'information afin de recentrer la discussion autour de mots et de groupes d{\textquoteright}individus bien définis. 

Les discussions qui entourant une actualité politique créent des communautés importantes qui restent néanmoins relativement distantes. Discutant sans vraiment se rencontrer, elle donne lieu à des discussions souvent très polarisées avec peu de dialogues entre les différentes communautés. La projection cartographique montre que Canton joue un r\^ole de précurseur dans les faits d{\textquoteright}actualité et para\^it {\textquotedblleft}sortir{\textquotedblright} les affaires. Pékin agit plutôt comme diffuseur en disséminant largement l{\textquoteright}information. Ce résultat renforce néanmoins l{\textquoteright}hypothèse d{\textquoteright}une similarité entre réseaux sociaux et médias traditionnels, puisque Canton possède traditionnellement une presse plus portée sur l'investigation que celle plus officielle de Pékin. Les disparités entre Est et Ouest sont également flagrantes lors de la diffusion, avec une domination des grandes zones urbaines de l{\textquoteright}Est chinois qui, même pondérées, restent omniprésentes sur les cartes de la diffusion. Les zones de l{\textquoteright}Ouest souvent moins urbanisée semblent être moins associées aux discussions, sauf dans le cas où celles-ci les concernent directement (comme le Xinjiang) 

L{\textquoteright}émission de télévision \textit{The Voice} propose en quelque sorte un contre-exemple du mème. Le réseau de conversations entourant cette émission s'organise autour de très peu de diffuseurs dialoguant peu mais entourés de très nombreux fans. Les mots de moindre importance sont exclus des conversations, contraints à la marginalité dans le réseau sémantique. Le diffusion de \textit{The Voice} ne possède donc pas à proprement parler les spécificités qu'on attendrait d'une diffusion sur les réseaux sociaux avec peu de conversations. Son topogramme se caractérise par une très forte agrégation et une très faible modularité (peu de groupes fortement dominés par des diffuseurs importants). Il ne se pose pas du tout comme un lieu commun du Web, ouvert et associatif,  mais plut\^ot comme une zone d'échanges clairement identifiée et territorialisée. 

L'observation des spécificités de diffusion des faits d'actualité ou des campagnes commerciales nous a permis de vérifier des choses déjà connues : l'actualité se propage selon des cycles courts, les médias de Canton sont plus avant-gardistes que ceux de Pékin, la communication télévisuelle est soigneusement préparée, etc. Comme souvent dans les études utilisant l'analyse de données. Ces quelques observations viennent infirmer des connaissances préalables sans vraiment apporter de nouveautés. 

Néanmoins, l'observation des modèles peu connus des mèmes absurdistes est une des dimensions intéressantes mises à jour dans cette étude. En effet, le topogramme des mèmes comiques possède des structures atypiques. Contrairement aux autres diffusion qui présentent des graphes conversationnels bien organisés, celui des mèmes \textit{dufu} et \textit{yuanfang} est fragmentée en nombreux petits groupes de discussions très intégrés qui communiquent très peu entre eux. Tout se passe comme si nous n'assistions pas à une vaste conversation mais plutôt à de nombreuses petites conversations. Les origines géographiques diverses des utilisateurs montre que la blague fonctionne localement. Également, le caractère anecdotique et humoristique semble être un fort vecteur de diffusion avec une présence beaucoup plus durable dans le temps que dans le cas des actualités. Leurs graphes sémantiques se composent autour de peu de mots-clés qui sont réutilisés de manière non-définitive, permettant ainsi une grande variation et une grande appropriation par des utilisateurs même isolés. Ainsi, la structuration des conversations en groupes lexicaux ouverts aux associations peu communes semblent favoriser la diffusion des mèmes et leur permettre de durer dans le temps. Les utilisateurs de Taiwan sont présent dans la diffusion des mèmes absurdistes, alors qu'ils sont absents des discussions politiques. Ainsi, les mèmes possèdent des topogrammes associatifs qui encouragent une participation durable.

Nous pouvons ainsi statuer que dans notre échantillon, parmi les éléments observés seuls \textit{dufu} et \textit{yuanfang} semblent réellement correspondre à la définition de \textit{``mème''}. Les débats de société et les faits d'actualité semblent plus être des ``événements web'', reconduction de phénomènes pré-existant à l'Internet. Le débat  sur les faits divers semblent avoir toujours existé, même si l'Internet en change radicalement l'échelle. Les modèles issus des médias traditionnels ont également été reconduits comme le montre l'exemple de \textit{The Voice}. La nouveauté dans cet ensemble sont bien les mèmes comiques. 

L'humour, s'il pré-existe heureusement à l'Internet, procède d'un régime éminemment conversationnel et prend la forme particulière des mèmes sur Internet. Le jeu de mots, le détournement d'images ou de slogans sont tous des pratiques communes qui se voient reconduites sous des formes nouvelles en ligne. L'étude de ces formes humoristiques de l'Internet permet d'accéder à des exemples de conversations qui ne sont pas nécessairement planifiées et possèdent un caractère très informel. Ce type d'échanges presque spontané semble pourtant suivre des modèles identifiables qu'il est possible d'étudier plus en détails grâce à l'usage des topogrammes. 

Hypothèse : \textit{La circulation des contenus sur les réseaux sociaux accroît la fragmentation du milieu numérique}.

Les graphes conversationnels extraits des différents types de contenus montre la structure atomique et rhizomique des échanges en ligne. Nous avons pu voir comment les communautés de discussions se constituent. Ce modèle de discussions en groupes fondé sur l'inclusion ou l'exclusion de la conversation est un facteur de fragmentation (voir \ref{}). Les mèmes humoristiques présentent notamment la particularité intéressante de générer beaucoup de conversations pendant une longue durée, mais seulement au sein de petits groupes et non entre ses groupes. L'effet \textit{``private joke''} qui fait de la blague partagée un signe de reconnaissance mutuelle et un des rôles importants joués par les mèmes Internet (voir chap. \ref{chap:memes}). Néanmoins, dans cette circulation ne se joue pas nécessairement une ``socialisation'' conçue comme le propre des réseaux mais plutôt une fragmentation en petits groupes à l'identité forte. Cette différentiation entre les utilisateurs vers des communautés d'appartenance existent donc au sein d'une même plate-forme et se constituent lors de la circulation des contenus. 

Une lecture des graphes conversationnels pourrait permettre d'inférer la nature plus ou moins dissociatives ou associative de certains contenus. Nous avons vu comment des diffusions plus ou moins centralisées et contrôlées invitent ou non à la conversation. Nous avons également montré que les modèles dissociatifs et associatifs coexistent largement sur le réseaux social Sina Weibo, même si l'activité historique de Sina comme fournisseur de contenus semble produire un espace de discussion où les contenus plus dissociatifs dominent. A l{\textquoteright}inverse, la vivacité des débats entourant les développements économiques, sociaux et politiques de la Chine moderne donnent lieu à de nombreuses tentatives d{\textquoteright}associer la population chinoise par les médias. Les mèmes quant à eux semblent remettre en question cette dichotomie. Au contraire, les observations semblent indiquer que les mèmes agissent en faveur d'une fragmentation plus accrue du réseau. Ainsi, les résultats montrés par les mèmes remettent en question cette définition de la nature associative et dissociative d'un événement ou d'un média. Les vastes conversations des mèmes humoristiques impliquent de nombreux utilisateurs très actifs, sans pour autant les associer réellement au sein d'un dialogue. 

La faible taille de notre échantillon ne nous permet pas d{\textquoteright}observer de manière empirique les structures récurrentes à un échelle suffisante. Néanmoins, les topogrammes observés permettent de voir que la description des actes de communication ne peut suivre une simple dichotomie. La multitude des situations rend périlleux de qualifier de fa\c{c}on définitive un espace de conversation en ligne. \'A plus forte, la tâche devient périlleuse lorsqu'il s'agit de qualifier un milieu numérique. L'approche ``par le milieu'' que nous avions présenté en introduction connaît donc des limites importantes. La première est qualitative, car les données contenaient seulement une indication de lieu vague donnée par les utilisateurs. L'absence de géolocalisation en tant que telle n'a pas permis une étude extensive des modèles géographiques mis en œuvre lors de la diffusion des mèmes. Les observations faites dans les résultats viennent tout au plus valider des éléments déjà connus, en disant plus sur l'outil et la méthode que sur les phénomènes réellement observés. D'autre part, la plasticité théorique du terme milieu en fait également un outil faiblement opérationnel du point de vue méthodologique. L'histoire du mot teintée de déterminisme et plus généralement l'ambition globalisante du concept nous dirigent vers des généralisations tentantes mais néanmoins hasardeuses. L'analyse de données comme la recherche de terrain connaît bien des difficultés à observer empiriquement ce ``milieu'' dont l'existence et les limites restent finalement non résolues par l'étude. 


La t\^ache plus large de compréhension et de description du r\^ole des échanges en ligne dépasse largement les seuls réseaux sociaux. Loin d{\textquoteright}être une exception chinoise, le contr\^ole du caractère dissociatif et associatif des nouvelles formes langagières du Web est une problématique entourant les technologies de la parole dans le monde entier. Les premiers apports méthodologiques de cette étude demandent désormais à être poursuivis, testés et vérifiés sur différents terrains d{\textquoteright}études. La description des invariants et variations au sein d{\textquoteright}un corpus plus large de mèmes ou d{\textquoteright}autres formes textuelles (e-mail, commentaires, etc.) pourrait permettre de définir les canons des nouvelles formes d{\textquoteright}énonciation en ligne, comme autant d{\textquoteright}archétypes de la fa\c{c}on dont s{\textquoteright}écrivent les lieux communs du Web. Une étude plus large concernant un ensemble de corpus de différents médias pourrait donc à terme permettre de classifier les topogrammes et de dresser ainsi une typologie des actes de communication publiques ou privés. 

Une fois identifiés, les éclairages apportés par les topogrammes sur les faits médiatiques peuvent devenir à la fois un outil d{\textquoteright}analyse de la diffusion a posteriori mais également de manière plus stratégique un support pour concevoir des actes de communications. Identifier Les caractéristiques particulières de diffusion peut également permettre de caractériser qualitativement des diffusion à différents moments et offrir une lecture rapide de leur nature (diffusion de masse, faits divers, artefacts de communication, etc.) pouvant s{\textquoteright}avérer très utiles pour la vérification des sources journalistiques sur les réseaux sociaux ou la communication en situation de crise notamment. Les notions d{\textquoteright}impact et d{\textquoteright}influence pourraient également être réajustée à l{\textquoteright}aune de facteurs paramétriques et contextuels propres à des topogrammes particuliers, notamment en terme de moment et de centralité dans les réseaux sémantiques, géographiques ou conversationnels. Également, une réévaluation du concept de milieu numérique pour une prise en compte plus importante des phénomènes physiques (appartenances territoriales, réseaux de lieux, etc.) offrirait une meilleure assise théorique à cette méthodologie d{\textquoteright}analyse permettant notamment à l{\textquoteright}analyse de données d{\textquoteright}être combiné à un travail de terrain. 

Cette limite conceptuelle inhérente s{\textquoteright}exprime ici notamment dans la visualisation des conversations sous forme de graphe. L'espace de la représentation où le graphe est projeté possède des propriétés topologiques non définies qui devraient pourtant être celles du milieu numérique, s'il eût été observable. 

Comment pourtant définir un milieu s{\textquoteright}il est propre à un usage voire à une personne? La description des technologies mises en {\oe}uvre dans les actes de communication est-elle suffisante? Comment le milieu et ses protocoles existe-t-il lors des la transduction et l{\textquoteright}individuation? Cette existence est-elle alors observable? Au regard de ses limites conceptuelles fortes, l{\textquoteright}idée de milieu numérique gagnerait s\^urement à être problématisée autour d{\textquoteright}une définition plus précise des relations directes existantes entre le cyber-espace dans sa forme physique (les serveurs, machines, usagers, etc.) et sa manifestation {\textquotedblleft}en ligne{\textquotedblright}. L{\textquoteright}histoire des sciences et plus particulièrement de la géographie a montré comment le concept plut\^ot déterministe de milieu peut être repensé en termes d{\textquoteright}espace, de lieu et de territoire, voire de paysages offrant des prises théoriques et des approches méthodologiques nouvelles. 

Peut-être la vision trop généralisatrice voire éthérique induite par l{\textquoteright}usage du concept de milieu pourrait être précisée par une approche incarnée dans des lieux, des espaces et des territoires, support structurant physiquement à la fois pour l{\textquoteright}étude mais également pour la représentation. Le choix et le statut de l'espace de la représentation des phénomènes d'échanges en ligne est donc une des questions centrales que soulèvent à la fois l'usage de la visualisation et le concept de milieu numérique. Ainsi, l'ambition de cette recherche a été d'apporter une pierre à cette réflexion en réfléchissant à l'articulation possible de différents espaces de projection et de représentation. Nous avons notamment essayé de mobiliser des apports théoriques venues de différentes disciplines afin de problématiser notre lecture dans un contexte élargi, donnant à voir une existence à la fois langagière et géographique des actes de communication. 


Le cas de la Chine nous fourni un terrain idéal pour s'interroger sur la production et la matérialité de cet objet physique devenu espace d'expression qu'est l'Internet. Un vaste travail doit encore être mené sur les usages de l'Internet pour repenser  la séparation artificielle entre ``bits'' et ``atoms'' et resituer la magie du \textit{``cloud''} dans le domaine de l'actuel. La distinction chaque jour plus ténue entre \textit{réseaux}, \textit{software} et \textit{hardware} rend notamment le travail sur la réalité physique et situationnelle des actes de communication de plus en plus pressant. Pour rendre lisible les processus d'individuation en jeu dans les usages des technologies du réseau, nous avons besoin d'un cadre de lecture qui s'intéresse aux particuliers de l'Internet et à leurs relations avec l'environnement, jusqu'aux mines dont proviennent les ordinateurs.

La linéarité de la \textit{timeline} qui constituent normalement l'interface de lecture des réseaux sociaux ne doit pas non plus éclipser la diversité des formes d'expressions qui s'y déroulent. L'analyse de différents types de contenus nécessitent le recours à des approches particulières, hérités des réflexions sur les différents régimes du discours. Les méthodes mobilisées pour l'étude des actes de communication en ligne doivent être capables de recourir aux savoirs qui les précèdent sur le langage notamment. Nous avons discuté dans ce travail des travaux sur l'énonciation, des lieux communs et de l'analyse rhétorique ou encore du modèle \textit{word-of-mouth} de la diffusion en ligne. Ces théories considérées conjointement forment un ensemble en apparence hétéroclite. Néanmoins, chacune de ces approches peut faire sens si elle est invoquée pour l'analyse d'un type de contenu adéquat. Identifier le régime d'expression qui accompagne un objet numérique, voire son existence et son évolution entre discours et énonciation peut permettre une approche plus juste des échanges sur Internet. Au-delà d'une grammatologie des contenus, l'étude des usages doit chercher à prendre en compte les acteurs multiples prenant part aux phénomènes observables sur les plate-formes du Web. 

L'interrogation sur l'existence ou la conception d'objets numériques doit donc accompagner d'une recherche dans la cohérence des intentions de ces différents acteurs. Notamment, la recherche d'utilisateurs ``représentatifs'' semblent vaines tant les intentions qui président à la production d'un contenu peuvent être différentes. Sur une plate-forme comme \textit{Sina Weibo} ou \textit{Facebook}, un même utilisateur peut dans la même heure partager une blague avec l'un de ses amis, promouvoir une action collective ou sa propre marque puis commenter un article de journal. La recherche d'une cohérence intrinsèque dans l'ensemble de ces pratiques doit prendre en compte la diversité de ces différents modes d'expression. Ainsi, l'étude de l'activité ayant cours sur les réseaux sociaux et à plus forte raison sur Internet doit prendre en compte les moments et situations qui amènent à la création des objets numériques. D'une part, l'actualisation de notre milieu numérique sous une forme particulière (mème, news, etc.) dépend de la structure technologique économique et politique de l'espace d'expression défini par la plate-forme Web - ici \textit{Sina Weibo}. D'autre part, l'intention dans l'acte de communication doit être interroger comme une des clés de compréhension qui permettront une lecture pertinente des usages. La complexité de cette tâche nécessite une attention soutenue aux contenus ainsi qu'une approche méthodologique renouvelée dont nous avons tenté de montrer ici un exemple.


\subsubsection[Validité et limites des méthodologies Big Data]{Validité et limites des méthodologies Big Data}

%à la lumière de cela, dire si ton travail \ (l{\textquoteright}analyse de tes résultats) valide tes choix méthodologiques (les algorithmes mais aussi les outils conceptuels.

La validation des hypothèses par un travail d'analyse de données est une tâche délicate. Nous avons pu voir comment l'accroissement du volume de données amènent à l'augmentation de la complexité des processus de traitement de l'information. Les algorithmes pouvant donner des résultats utiles sont souvent sophistiqués, obligeant à un difficile travail d'optimisation des calculs. Ces couches successives de développements ingénieriques doivent la plupart du temps être effectuées spécifiquement pour une tâche ou un jeu de données précis. La cohérence et la pertinence des résultats est alors conditionnée à un travail d'écriture du code. 

Le concept de \textit{validité interne} décrit la pertinence ou la justesse de l'explication fournie par le chercheur pour expliquer les résultats qu'il a obtenus dans son expérience \citep{Yin2009a}. Dans le cas de l'analyse de données, un des premiers problèmes pour sa reproductibilité est celui de la disponibilité des données. Dans ce travail, nous avons choisi un jeu de données librement disponible afin de permettre facilement de recréer les étapes. La seconde condition néanmoins est la disponibilité du code et des outils qui ont servis à mettre en œuvre l'analyse. Le travail d'écriture associé à la production de code est à la fois pratique et réflexif. La possibilité de simplement consulter les lignes ayant été écrites pour réaliser une analyse spécifique permet de s'approprier dans le détail la méthode utilisée. En effet, contrairement au domaine de l'algorithmique ou des mathématiques appliquées qui se fonde sur le développement logique, les méthodologies de l'analyse de données se fondent sur l'éventualité d'obtenir un résultat. Bien souvent, l'analyse statistique ne nécessite pas de développement algorithmique complexe, mais plutôt l'utilisation d'outils existants. Un des travaux importants menés par la communauté scientifique en informatique est la publication de librairies pour l'usage scientifique aux qualités d'écriture et performances rigoureuses. De plus, ces librairies sont souvent conçues pour être utilisées par d'autres personnes, étape de validation supplémentaire du travail. 

Néanmoins, le code écrit avec comme but l'analyse de données n'est pas soumis aux mêmes contraintes. Tout d'abord, il est très peu publié ou réutilisé et à ce titre ne nécessite pas de de rigueur particulière dans l'écriture (tests, commentaires, lisibilité, disponibilité, etc.). D'autre part, sa formulation sous forme algorithmique ne représente pas nécessairement son action. En effet, le premier travail de l'analyste de données est de repérer parmi l'abondance des librairies disponibles les outils les plus adaptés. Ces choix sont difficilement communicables puisqu'ils dépendent d'un ensemble de critères souvent définis par l'analyste lui-même. D'autre part, le travail d'écriture et de référencement de ces librairies se fait directement au contact de son jeu de données. La science algorithmique est un développement ayant pour base la logique. L'analyse de données s'apparente davantage à une discussion avec un jeu de données. Le travail préliminaire pour l'identification de mèmes (voir section \ref{sec:id-meme}) se présente sous la forme d'un ensemble d'expérimentations \textit{nécessaires} qui ont permis à l'analyse des données d'obtenir des résultats. La tâche dans cette partie du travail était de pouvoir décrire un modèle de mème compréhensible par un ordinateur. La définition complexe et débattue du mème ne s'écrit donc pas une fois pour toute, mais procède d'une itération qui nécessite de nombreuses expérimentations. Ces séries d'essais permettent de construire une représentation de l'objet recherché. Les informaticiens disent d'ailleurs souvent que l'un des problèmes les plus complexes de la programmation est de nommer les objets\footnote{\textit{``There are only two hard problems in Computer Science: cache invalidation and naming things.''}, citation attribuée au co-fondateur de \textit{Netscape} Phil Karlton}. Comme dans le travail de terrain, les modèles conceptuels doivent passer cet ``épreuve du réel''. Ce dialogue avec les données s'écrit dans un ou des langages informatiques. Comme pour les entretiens de la sociologie, il semble nécessaire de les documenter de façon rigoureuse car il constitue finalement un des aspects fondamentaux de la méthodologie et des résultats de ce type d'étude. La formulation sous forme d'algorithmes de la méthodologie retenue ne saurait remplacer le texte original utilisé pour construire l'analyse de données. 

Le passage à une validité dite ``externe'', c'est à dire à une généralisation à d'autre terrains ou populations, dépend notamment de la nature de ces échanges préalables. Quelles sont les questions exactes qui ont été posées et comment ont-elles été formulées? La longue réflexion des sciences ethnographiques et anthropologiques sur le rôle du regard et la présence de l'observateur doivent se poursuivre dans les méthodologies liés à l'analyse de données en sciences sociales. Des conclusions faites sur la base de questions mal formulées sont nuisibles. Dans le cas de l'analyse de donnes, le texte de ces questions est écrit sous la forme de scripts, référençant et mobilisant des travaux d'auteurs précédents que sont les librairies. Comme celui de l'étude de terrain, le savoir-faire de l'analyse de données doit sujet à une réflexion critique sur la base de références et d'exigences communes.

Les données sont les traces d{\textquoteright}activités passées et doivent être comprises en tant que tel. Ce travail s'est écrit en forme d'interrogation sur les limites et les enjeux de l'usage des données issues de réseaux sociaux dans les études en sciences humaines. \'A ce stade, il est difficile voire impossible de généraliser les résultats de notre étude. L'absence d'informations de géolocalisation a notamment rendu difficile l'approche cartographique. L{\textquoteright}étude a donc nécessité une vaste reconstruction du contexte et des intentions qui sont finalement souvent l'enjeu central d'une grande part de notre méthodologie. Une connaissance étendue du terrain est ici requise, peut-être même davantage que dans des études plus traditionnels sur le terrain lui-même. 

La discussion entre plusieurs disciplines scientifiques est la seule solution pour pallier aux limite des données pour parvenir à des conclusions intéressantes. La faible connaissance des langages informatiques dans les équipes en sciences sociales à l'heure actuelle \citep{Wieviorka2013} rend difficile la définition d'une méthode commune autour l'analyse de données. Pourtant, la compréhension d'un jeu de données spécifique offre une opportunité intéressante de construire une approche pluri-disciplinaire autour d'un même objet. En présentant les échecs et réussites face aux données dans ce travail, nous avons essayer de montrer comment le choix d'une méthodologie peut avoir une portée opératoire permettant de mobiliser des disciplines et savoirs en apparence hétéroclite. Un préalable de l'analyse de données est de faire ce travail de réflexion large sur l'objet étudié afin de bien définir d'une part le contexte de production et la nature des données, d'autre part les aspects qui nécessitent d'être examinés.


Finalement, comme tout travail scientifique, les biais des méthodes et la fragilité des résultats existent comme autant d'opportunités invitant à la discussion et aux débats pour en repousser les limites.

% Quelles questions nouvelles ton travail soulève-t-il? 