\section[Discussions]{Discussions}

% Rappel de la thèse
Les mèmes Internet sont une forme de contenu possédant de nombreuses particularités. 
Nous avons besoin de développer un nouvel outil afin de pouvoir les observer dans le détail et de mieux les connaître.

L{\textquoteright}analyse et la visualisation de données des réseaux  sociaux permettent une meilleure connaissance des détails de la  diffusion des contenus en ligne. Néanmoins, les approches couramment  utilisées pour l{\textquoteright}analyse de la diffusion  d{\textquoteright}informations sur les réseaux sociaux se fondent sur  des modèles conceptuels très réducteurs  (émetteur-message-récepteur), faisant peu de cas des dimensions  énonciatives des actes de communication en ligne (fonctions, enjeux,  lieux, contexte, etc.).


% Rappel des question(s) de recherche
Comment s'opère la diffusion des mèmes sur Sina Weibo? A plus forte raison, quelles sont les particularités de ce type de contenu par rapport à d'autres contenus plus traditionnels? Comment pouvons-nous observer ces mèmes Internet? Quelles relations entretiennent-t-ils avec le milieu numérique qui les produit? Ont-t-ils une existence territoriale.

% Qu'est-ce qui a été fait?
En nous intéressant à la diffusion des  conversations sur le service de réseau social chinois \textit{Sina  Weibo}, nous avons cherché ici à comprendre comment cette  connaissance des modèles de diffusion de  l{\textquoteright}information en ligne pouvait être enrichie  d{\textquoteright}apports conceptuels et méthodologiques  d{\textquoteright}autres disciplines comme les théories de la  communication et des systèmes d{\textquoteright}information, la  critique des médias ainsi que la géographie des technologies. Plus  particulièrement, nous avons entrepris de développer un système  d{\textquoteright}analyse et de visualisation de données pour  explorer les dimensions sémantiques, conversationnelles, temporelles  et géographiques de la diffusion de mèmes Internet.   

% Rappel méthodologie
un vaste jeu de données de 200 millions de tweets anonymisés comprenant à la fois le texte des messages ainsi qu{\textquoteright}un ensemble de méta-données dont la géolocalisation par province des utilisateurs. Cet échantillon a été collecté de manière aléatoire par l{\textquoteright}Université de Hong Kong pour représenter de fa\c{c}on fidèle l{\textquoteright}activité totale de l{\textquoteright}année 2012 sur Sina Weibo (Fu \& Chau, 2013). 

Nous avons ensuite procéder à différentes analyses préliminaires pour mettre en place le cadre expérimental d'analyse des données.

Notre première analyse s{\textquoteright}est intéressé tout d{\textquoteright}abord à l{\textquoteright}ensemble des hashtags de l{\textquoteright}année 2012 pour comprendre les contenus les plus discutés. 

Ensuite, nous avons sélectionné un ensemble d{\textquoteright}une dizaine de mèmes d{\textquoteright}après leur représentativité thématique : comique, commercial, publicitaire, actualité, faits divers et scandale politique. 

A l{\textquoteright}aide d{\textquoteright}un outil d{\textquoteright}analyse et de visualisation spécialement con\c{c}u pour l{\textquoteright}occasion, nous avons ensuite extraits les topogrames de ces différents mèmes afin de comprendre et comparer les structures de leurs diffusions.


% Les hypothèses + résultats

En rappelant chacune des hypothèses formulées précédemment (voir section \ref{sec:hypotheses}), nous allons maintenant proposer une lecture critique des résultats obtenues et des démarches mises en œuvre lors de la réalisation de ce travail. 


\begin{itemize}


\item{L'analyse de la diffusion ne doit pas se contenter de l'analyse conversationnelle, mais doit mobiliser les modèles étudiant l'énonciation} 

\item{L'analyse et la visualisation de données permette d'établir une classification des discussions en ligne selon les structures de leurs diffusions.}

\item{La circulation des contenus sur les réseaux sociaux accroît la fragmentation du milieu numérique.}
\end{itemize}


discussion : la particularité est le mème comique. Il est différent, c'est un modèle.




\subsubsection{Les industries culturelles en Chine Sina Weibo} 

Hypothèse : \textit{La majorité des contenus circulant sur les réseaux sociaux s'apparentent largement à ceux des médias traditionnels} 

La première partie de l{\textquoteright}analyse de données portant sur les hashtags les plus mentionnés nous a permis de valider partiellement notre deuxième hypothèse en constatant que les contenus majoritairement discutés sur Sina Weibo étaient en largement similaires à ceux des médias classiques : divertissements, produits culturels, traffic routier, sports, etc. Néanmoins, nous avons aussi pu observer que les hashtags étaient la plupart du temps un artefact de campagnes de communication organisées, reflétant un usage des réseaux sociaux pour le marketing promotionnel politique, médiatique ou commercial. 


En Chine, cet intégration très rapide des grands groupes médiatiques a été rendu possible par un ensemble de conditions. La première est le protectorat économique qui a été mis en place sur tout le secteur des médias, lié à la fois à la volonté de contrôle politique mais surtout à la volonté économique des acteurs du pays. Comme nous l'avons montré (chap. \ref{sec:internet-chine}), le développement des TIC et de l'Internet a été dès le début conçu par Pékin comme un des piliers chargé de soutenir la croissance économique du pays. Ce protectionnisme sur tout le secteur des industries culturelles a permis l'apparition rapide de nombreuses firmes qui ont bénéficié d'un marché captif et non soumis à la concurrence internationale. Les opportunités de verticalisation ont donc été plus nombreuses. 

Un autre facteur important la nécessité de contrôle de la ``qualité'' des informations publiés par ces services. En effet, les fournisseurs et diffuseurs de contenus chinois sont soumis à des règles strictes de contrôle politique. Ils doivent en effet s'assurer que les contenus qu'ils publient sont ``en harmonie'' avec les règles et tendances définies par le gouvernement de Pékin. La gestion de ce risque a rendu indispensable la mise en place d'un contrôle continu des informations publiées. De ce fait, l'intégration verticale entre producteurs et distributeurs de contenu parait la solution la plus logique. Les processus de contrôle de l'information liés à la gestion des risques ``politiques'' de publication ont donc favorisé cette intégration.

Cette perspective de gérer la conformité des contenus aux législations et à l'agenda politique d'un gouvernement peut paraît au premier abord une spécificité de l'Internet chinois. En effet, les termes du débat et le cadre légal qui régit les relations entre les entreprises privées et les gouvernements est bien différents dans le cas des firmes américaines comme Facebook ou Twitter. Néanmoins, la gestion des aspects légaux de la possession et de l'usage des données devient une compétences de plus en plus centrales des services de réseaux sociaux. Culminant dans la récente ``affaire Snowden'' \cite{Greenwald2013}, les questions de vie privée viennent notamment faire pression pour davantage de responsabilité légale pour les services de réseaux sociaux. Également, l'inégalité de traitement des flux de données est déjà au centre des débats législatifs et managériaux avec la question de la neutralité des réseaux \cite{Schafer2011}. La pression grandissante pour légiférer et faire appliquer les décisions légales auprès des firmes Internet va nécessiter un contrôle accrue des différents maillons de la chaîne de production d'information. Ainsi, on peut gager que l'intégration verticale sera également une des premières solutions pour rentabiliser les coûteux dispositifs de contrôle à mettre en place. Comme dans le cas de Sina Weibo et des firmes de l'industrie médiatique en Chine, la pression légale et politique pourrait amener à une intégration des services de contenus avec les producteurs de contenus.

Du coté des contenus également, le secteur des industries culturelles est historiquement sujet à une très forte concentration \cite{Martel2010}. Les grands groupes médiatiques fonctionnent selon ce principe d'intégration verticale, diversifiant même largement leurs activités dans le secteurs. Comme nous l'avons vu avec Sina Weibo, les contenus publiés par différents médias sont déjà largement similaires et quotidiennement repris. La dichotomie entre contenus ``web'' et ``traditionnels'' s'estompe largement, avec le partage quotidien d'articles de journaux ou d'émissions de télévision sur la Toile. Les canaux de distribution eux-mêmes tendent à se recentrer, notamment autour du format mobile. Télévision web,  web documentaire, data journalisme, les pratiques tendent à se rejoindre et les services de réseaux sociaux jouent alors le rôle de diffuseur principal. Dans cette dynamique, le rapprochement entre \textit{pure players} du web comme Twitter ou Facebook et les firmes médiatiques traditionnels ne paraît pas impossible. Cette tendance se voit notamment très clairement dans la nécessité de rentabiliser au maximum le placement des informations pour l'obtention revenus publicitaires sur des écrans toujours plus petits. Les dispositifs de captation de l'attention que sont les services de réseaux sociaux deviennent la clé de voûte de l'industrie médiatique. 

Ainsi, l'histoire récente des médias chinois et notamment celle de Sina Weibo ne doit pas être lue seulement dans les termes d'une lutte politique entre les utilisateurs et le gouvernement de Pékin. La dimension stratégique du développement industriel des médias en Chine apportent aujourd'hui des clés de lecture éclairantes non seulement pour le pays, mais pour le monde entier. L'équilibre entre contrôle politique, réussite commerciale et développement industriel crée un précédent à une échelle nouvelle pour l'Internet qui ne doit pas être minimisé. L'intégration forte entre entités gouvernementales, fournisseur de contenus et canaux de distribution de l'information propose également un modèle inédit. Les études sur l'Internet chinois doivent s'intéresser davantage aux mécanismes et relations entre ces différents acteurs, en évitant de les considérer comme une seule entité. Cette relation fusionnelle de pouvoir dans le champ médiatique pourrait même être considérée comme une forme de maturité, acquise lors d'un développement hyper-rapide permis par un environnement protectionniste et autoritaire. En effet, les  discussions entourant le cadre législatif de l'Internet et les lois cherchant à réguler son usage (\textit{Hadopi} en France, \textit{SOPA} aux États-Unis, \textit{Data Protection Directive} de l'UNion Européenne) invitent à considérer l'évolution de la gouvernance du réseau Internet vers une plus forte intégration des pouvoirs dans la sphère médiatique pour une gestion des risques politiques plus contrôlée. Ainsi, l'expérience de l'Internet chinois vient changer radicalement le réseau mondial en établissant un modèle différent de celui imaginé initialement par les fondateurs du réseau ARPANET. S'il est possible d'écarter le modèle chinois de l'Internet comme un archaïsme face au discours du \textit{new-age} des géants californiens, son existence a néanmoins établi un mode de développement et de gouvernance qui s'est jusqu'ici montré d'une efficacité et d'une rentabilité sans faille et pourrait bien dans le futur séduire davantage d'états et de' gouvernements.


\subsubsection{Les échanges en ligne sont à comprendre comme des actes d'énonciation, dont les mèmes sont l'expression la plus actuelle.}

Les hashtags sont avant tout des artefacts de campagne planifiées. Il est impossible d{\textquoteright}infirmer que les hashtags propose un reflet de l{\textquoteright}activité réelle des utilisateurs.


\subsubsection{Outil de visualisation multi-graphes de la diffusion (hypothèse 1)} 

La seconde étape de notre travail d{\textquoteright}analyse a consisté à concevoir un système ingénierique permettant l{\textquoteright}étude et la visualisation de structures et de tendances dans les données pour l{\textquoteright}ensemble des mèmes sélectionnés. En problématisant notre étude gr\^ace à la notion de \textit{topograme,} nous avons vu que la circulation de messages au sein du Web ne s{\textquoteright}effectuait pas seulement au sein d{\textquoteright}un réseau d{\textquoteright}individus mais dans un contexte plus vaste o\`u entre en jeu la sémantique des mots ainsi que l{\textquoteright}espace et les lieux d{\textquoteright}utilisation. Nous avons donc procédé pour chaque mème à l{\textquoteright}extraction des graphes sémantiques (co-occurence de mots), conversationnels (interactions entre les utilisateurs), temporels (évolution du volume de tweets sur une période donnée) et géographiques (origine des provinces des utilisateurs lors des conversations). L{\textquoteright}étude de ces différents aspects de la diffusion a nécessité une modélisation particulière des relations entre chacun de ces aspects et nous avons donc été amené à créer un outil d{\textquoteright}observation et d{\textquoteright}analyse permettant de visualiser sous forme de graphes multiples ces différentes lectures de la diffusion des objets digitaux.  
  
\subsubsection{Modèles de mèmes et topogrames (hypothèse 3)}
 
Gr\^ace à l{\textquoteright}analyse des quelques échantillons de contenus issus de Sina Weibo, nous avons pu vérifier partiellement notre hypothèse méthodologique en lisant dans les différents topogrames des particularités pour les mèmes sélectionnés. Notamment, le caractère anecdotique et humoristique semble \^etre un fort vecteur de diffusion avec un succès plus durable dans le temps chez les mèmes comiques \textit{dufu} et \textit{yuanfang}. Les grands volumes de discussion entourant la diffusion des faits d{\textquoteright}actualité semblent permettre également qu{\textquoteright}un mème se propage dans la durée mais auprès d{\textquoteright}un plus grand nombre de communautés. également, les mèmes absurdistes ou humoristique se compose de peu de mots-clés qui sont réutilisés de manière non-définitive, permettant ainsi une grande variation et une grande appropriation par des utilisateurs m\^eme isolés. A l{\textquoteright}inverse un débat sur un sujet de société brulant démarre de manière brutale et centralise les conversations en mettant en relations des communautés diverses, provoquant des concentrations de mots et d{\textquoteright}individus dans des groupes bien définis. Les discussions qui entourent une actualité politique créent des communautés importantes de discussions qui restent néanmoins relativement distantes, discutant sans se rencontrer de fa\c{c}on plus polarisée. Parmi les cas étudiés, nous avons pu identifier un contre-exemple avec l{\textquoteright}émission de télévision \textit{The Voice}\textit{. }Sa diffusion en ligne morcelée montre que le réseau des conversations s{\textquotesingle}organise autour de peu de diffuseurs importants entourés de fans. Le discours est sémantiquement très établi et inclue difficilement d{\textquoteright}autres discussions ou éléments annexes qui sont contraints à la marginalité. Ainsi, \textit{The Voice }ne doit pas \^etre compris comme un mème Internet, puisqu{\textquoteright}il ne crée pas de conversations et ne possède pas de spécificités de diffusion propres aux réseaux sociaux. Son topograme se caractérise par une très forte agrégation et une très faible modularité (peu de groupes fortement dominés par des diffuseurs importants) et peut donc \^etre considéré comme largement dissociatif. Il ne se pose pas en lieu commun du Web mais plut\^ot en lieu spécifique, clairement identifié et territorialisé. A l{\textquoteright}inverse, la diffusion des mèmes absurdistes ou des sujets d{\textquoteright}actualité est fragmentée en nombreux petits groupes très intégrés et présente ainsi des qualités plus associatives qui encouragent la participation. La structuration des conversations en groupes lexicaux ouverts aux associations peu communes semblent \^etre un facteur important de diffusion des mèmes.
 
\subsubsection{Existence géographique et milieu numérique (hypothèse 4)}
 
Gr\^ace à l{\textquoteright}outil de visualisation interactif développé spécialement pour cette analyse, il nous a donc été possible d{\textquoteright}observer les structures et relations mutuelles de ces graphes. Cette exploration nous a notamment permis de mettre à jour des caractéristiques spécifiques à certains mèmes parmi ceux sélectionnés. En s{\textquoteright}intéressant à un nombre restreint de mèmes, notre étude cherchait avant tout la validation d{\textquoteright}une hypothèse méthodologique mais sa systématisation à des corpus plus diversifiés permettrait peut-\^etre d{\textquoteright}identifier une typologie plus générale des modèles de \textit{topogrames} de contenus en ligne. L{\textquoteright}approche géographique a néanmoins donné assez peu de résultats, en proposant pas de structures nécessairement redondantes. Ainsi, la qualification d{\textquoteright}un milieu numérique par la classification de ces objets digitaux selon ses topogrammes semblent une t\^ache irréalisable dans cette étude. Les patterns géographiques observables dans la diffusion des mèmes montrent seulement une diffusion centrée autour de deux p\^oles médiatiques importants : Canton qui joue un r\^ole de précurseur dans les faits d{\textquoteright}actualité et para\^it {\textquotedblleft}sortir{\textquotedblright} les affaires et Pékin qui agit comme diffuseur en dispersant l{\textquoteright}information. Ce résultat renforce néanmoins l{\textquoteright}hypothèse d{\textquoteright}une similarité entre réseaux sociaux et médias traditionnels, puisque la localisation géographique des entités de diffusion les plus importantes sont identiques - regroupées dans ces deux villes de Chine.

\begin{itemize}
    \item Canton joue un rôle d'avant-garde dans l'info. Pékin est le diffuseur principal.
    \item Absence de Taiwan des discussions politiques internes à la Chine.
    \item Concentration de l'activité dans l'Est chinois.
\end{itemize}

Egalement, notre étude sur les hashtags puis sur les mèmes nous montre que les modèles dissociatifs et associatifs coexistent tout deux largement sur les réseaux sociaux chinois. La dimensions industrielle du fournisseur historique de contenus Sina produit un espace de discussion o\`u les contenus plus dissociatifs dominent avec une diffusion de masse proche des médias traditionnels. A l{\textquoteright}inverse, la vivacité des débats entourant les développements économiques, sociaux et politiques de la Chine moderne donnent lieu à de nombreuses tentatives d{\textquoteright}associer la population chinoise à ces discussions au travers des médias. Le r\^ole des mèmes à caractère politique, marketing ou des discussions de société est ici important. Les disparités entre Est et Ouest sont également flagrantes, avec une domination des grandes zones urbaines de l{\textquoteright}Est chinois qui, m\^eme pondérées, restent omniprésentes sur les cartes de la diffusion. Les zones de l{\textquoteright}Ouest souvent moins urbanisée semblent \^etre moins associées aux discussions, sauf dans le cas o\`u celles-ci les concernent directement (comme le Xinjiang et le Qinghai pour \textit{qiegao}).

Si les contenus diffusés sur Sina Weibo semblent résonner en de nombreux points avec ceux des médias traditionnels, nous n{\textquoteright}avons néanmoins pas pu réellement observé le milieu numérique en tant que tel, étant donné la taille \ réduite de l{\textquoteright}échantillon de mèmes étudiés. Egalement, l{\textquoteright}approche méthodologique ne proposé pas une historicité suffisante pour statuer des changements réellement effectifs dans la structuration des relations au sein du milieu numérique.
 

recommandations et portée opératoire de ton étude


\subsection[Validité et limites des méthodologies Big Data]{Validité et limites des méthodologies Big Data}


validité interne/externe (Tannery)


%GP 
%à la lumière de cela, dire si ton travail valide ta thèse de départ (totalement, un peu, beaucoup, pas du tout) et pourquoi.  
% 
%Avec prudence ( tu n{\textquoteright}es qun gravillon qui prétend s{\textquoteright}ajouter au Grand Mur de la science) . Avec quels biais éventuels, quelles limites, comment tu aurais pu mieux faire (apprends à battre ta coulpe, imagines toi à une séance d{\textquoteright}autocritique pendant la RévoCul). Quelles questions nouvelles ton travail soulève{}-t{}-il? 
% 
%à la lumière de cela, dire si ton travail valide chacune des tes hypothèses, sur le m\^eme mode. (il n{\textquoteright}y a pas de honte à reconna\^itre \ une hypothèse non vérifiée) 
% 
%à la lumière de cela, dire si ton travail \ (l{\textquoteright}analyse de tes résultats) valide tes choix méthodologiques (les algorithmes mais aussi les outils conceptuels. 

cohérence et pertinence de ces développements ingénieriques  spécifiques

Au regard des résultats, nous voyons qu{\textquoteright}il reste très difficile de qualifier de fa\c{c}on systématique une conversation en ligne et qui plus est l{\textquoteright}ensemble constitué par un milieu numérique. Le concept de topograme a permis de mettre en perspective différents aspects de la conversation pour les rendre observable et comparable et d{\textquoteright}en ébaucher une typologie mais la faible taille de notre échantillon ne nous permet pas d{\textquoteright}observer de manière empirique les structures récurrentes associés à des conversations. L{\textquoteright}apport théorique des différentes disciplines a notamment permis de problématiser notre lecture dans un contexte élargi en mettant en perspective les dimensions performatives et l{\textquoteright}existence géographique des actes de communication. Néanmoins, l{\textquoteright}approche par le \textit{milieu} a été un relatif échec. D{\textquoteright}une part, l{\textquoteright}usage de données ne contenant pas d{\textquoteright}information de géolocalisation en tant que telle (seulement une indication de lieu) ne permet pas une étude extensive des modèles géographiques mis en oeuvre lors de la diffusion des mèmes. D{\textquoteright}autre part, la définition d{\textquoteright}un {\textquotedblleft}milieu{\textquotedblright} en tant que tel pose la question de ses limites et de son déterminisme, qui reste non résolues par cette étude. Comment définir un milieu s{\textquoteright}il est propre à un usage voire à une personne? La description des technologies mises en {\oe}uvre dans les actes de communication est-elle suffisante? Comment le milieu et ses protocoles existe-t-il lors des la transduction et l{\textquoteright}individuation? Cette existence est-elle alors observable? Au regrd de ses limites conceptuelles fortes, l{\textquoteright}idée de milieu numérique gagnerait s\^urement à \^etre problématisée autour d{\textquoteright}une définition plus précise des relations directes existantes entre le cyber-espace dans sa forme physique (les serveurs, machines, usagers, etc.) et sa manifestation {\textquotedblleft}en ligne{\textquotedblright}. L{\textquoteright}histoire des sciences et plus particulièrement de la géographie a montré comment le concept plut\^ot déterministe de milieu peut \^etre repensé en termes d{\textquoteright}espace, de lieu et de territoire, voire de paysages offrant des prises théoriques et des approches méthodologiques nouvelles.

La plasticité théorique du terme milieu en fait également un outil faiblement opérationnel du point de vue méthodologique dans l{\textquoteright}étude de données ou de terrain. Cette limite conceptuelle inhérente s{\textquoteright}exprime ici lors de la visualisation des conversations sous forme de graphe pour l{\textquoteright}analyse : l{\textquoteright}espace de la représentation sur lequel se déroule la projection de graphe possède des propriétés topologiques encore indéfinies. Nous pouvons donc nous questionner sur la légitimité d{\textquoteright}une qualification de conversations et de pratiques numériques selon ce type de graphes. Le choix de cet espace de représentation suppose notamment que le modèle du réseau comme abstraction utile pour comprendre une conversation - ce que cette recherche après d{\textquoteright}autres a essayé de prouver. Peut-\^etre la vision trop généralisante voire étherique induite par l{\textquoteright}usage du concept de milieu pourrait \^etre précisé par une approche incarnée dans des lieux, des espaces et des territoires, support structurant physiquement à la fois pour l{\textquoteright}étude mais également pour la représentation.


La disponibilité des données nous a permis ici de nous livrer à un genre d{\textquoteright}analyse sur les mots et les discussions encore très récent. Nous savons néanmoins que ces données elles-m\^emes sont avant tout des traces d{\textquoteright}activités passées, dans lesquels de nombreuses informations sont manquantes - pensons aux aspects de géolocalisation ainsi que l{\textquoteright}ensemble des informations non-verbales des actes de communication dans cette étude par exemple. Ainsi, l{\textquoteright}étude de phénomènes particuliers d{\textquoteright}après ces réalités {\textquotedblleft}données{\textquotedblleft} puis reconstruites conna\^it de multiples limites qui doivent \^etre prises en compte lors de l{\textquoteright}analyse. L{\textquoteright}exigence d{\textquoteright}une connaissance étendue du terrain et la nécessité d{\textquoteright}une discussion entre plusieurs disciplines scientifiques semblent finalement un préalable aux études utilisant l{\textquoteright}analyse de données.

\subsubsection[Travaux à poursuivre]{Travaux à poursuivre} 
% 
%GP  
%tu conclues sur des travaux à poursuivre{\dots} 

La t\^ache plus large de compréhension et de description du r\^ole des échanges en ligne dépasse largement les seuls réseaux sociaux. Loin d{\textquoteright}\^etre une exception chinoise, le contr\^ole du caractère dissociatif et associatif des nouvelles formes langagières du Web est une problématique entourant les technologies de la parole dans le monde entier. Les premiers apports méthodologiques de cette étude demandent désormais à \^etre poursuivis, testés et vérifiés sur différents terrains d{\textquoteright}études. La description des invariants et variations au sein d{\textquoteright}un corpus plus large de mèmes ou d{\textquoteright}autres formes textuelles (e-mail, commentaires, etc.) pourrait permettre de définir les canons des nouvelles formes d{\textquoteright}énonciation en ligne, comme autant d{\textquoteright}archétypes de la fa\c{c}on dont s{\textquoteright}écrivent les lieux communs du Web. Une étude plus large concernant un ensemble de corpus de différents médias pourrait donc à terme permettre de classifier les topogrames et de dresser ainsi une typologie des actes de communication publiques ou privés. Une fois identifié, les éclairages apportés par les topogrames sur les faits médiatiques peuvent devenir à la fois un outil d{\textquoteright}analyse de la diffusion a posteriori mais également de manière plus stratégique un support pour concevoir des actes de communications. Identifier Les caractéristiques particulières de diffusion peut également permettre de caractériser qualitativement des diffusion à différents moments et offrir une lecture rapide de leur nature (diffusion de masse, faits divers, artefacts de communication, etc.) pouvant s{\textquoteright}avérer très utiles pour la vérification des sources journalistiques sur les réseaux sociaux ou la communication en situation de crise notamment. Les notions d{\textquoteright}impact et d{\textquoteright}influence pourraient également \^etre réajustée à l{\textquoteright}aune de facteurs paramétriques et contextuels propres à des topogrames particuliers, notamment en terme de moment et de centralité dans les réseaux sémantiques, géographiques ou conversationnels. Egalement, une réévaluation du concept de milieu numérique pour une prise en compte plus importante des phénomènes physiques (appartenances territoriales, réseaux de lieux, etc.) offrirait une meilleure assise théorique à cette méthodologie d{\textquoteright}analyse permettant notamment à l{\textquoteright}analyse de données d{\textquoteright}\^etre combiné à un travail de terrain. 