\section[Discussions]{Discussions}

% Rappel de la thèse
Les mèmes Internet sont une forme de contenu possédant de nombreuses particularités. 
Nous avons besoin de développer un nouvel outil afin de pouvoir les observer dans le détail et de mieux les connaître.

L{\textquoteright}analyse et la visualisation de données des réseaux  sociaux permettent une meilleure connaissance des détails de la  diffusion des contenus en ligne. Néanmoins, les approches couramment  utilisées pour l{\textquoteright}analyse de la diffusion  d{\textquoteright}informations sur les réseaux sociaux se fondent sur  des modèles conceptuels très réducteurs  (émetteur-message-récepteur), faisant peu de cas des dimensions  énonciatives des actes de communication en ligne (fonctions, enjeux,  lieux, contexte, etc.).


% Rappel des question(s) de recherche
Comment s'opère la diffusion des mèmes sur Sina Weibo? A plus forte raison, quelles sont les particularités de ce type de contenu par rapport à d'autres contenus plus traditionnels? Comment pouvons-nous observer ces mèmes Internet? Quelles relations entretiennent-t-ils avec le milieu numérique qui les produit? Ont-t-ils une existence territoriale.

% Qu'est-ce qui a été fait?
En nous intéressant à la diffusion des  conversations sur le service de réseau social chinois \textit{Sina  Weibo}, nous avons cherché ici à comprendre comment cette  connaissance des modèles de diffusion de  l{\textquoteright}information en ligne pouvait être enrichie  d{\textquoteright}apports conceptuels et méthodologiques  d{\textquoteright}autres disciplines comme les théories de la  communication et des systèmes d{\textquoteright}information, la  critique des médias ainsi que la géographie des technologies. Plus  particulièrement, nous avons entrepris de développer un système  d{\textquoteright}analyse et de visualisation de données pour  explorer les dimensions sémantiques, conversationnelles, temporelles  et géographiques de la diffusion de mèmes Internet.   

% Rappel méthodologie
un vaste jeu de données de 200 millions de tweets anonymisés comprenant à la fois le texte des messages ainsi qu{\textquoteright}un ensemble de méta-données dont la géolocalisation par province des utilisateurs. Cet échantillon a été collecté de manière aléatoire par l{\textquoteright}Université de Hong Kong pour représenter de fa\c{c}on fidèle l{\textquoteright}activité totale de l{\textquoteright}année 2012 sur Sina Weibo (Fu \& Chau, 2013). 

Nous avons ensuite procéder à différentes analyses préliminaires pour mettre en place le cadre expérimental d'analyse des données.

Notre première analyse s{\textquoteright}est intéressé tout d{\textquoteright}abord à l{\textquoteright}ensemble des hashtags de l{\textquoteright}année 2012 pour comprendre les contenus les plus discutés. 

Ensuite, nous avons sélectionné un ensemble d{\textquoteright}une dizaine de mèmes d{\textquoteright}après leur représentativité thématique : comique, commercial, publicitaire, actualité, faits divers et scandale politique. 

A l{\textquoteright}aide d{\textquoteright}un outil d{\textquoteright}analyse et de visualisation spécialement con\c{c}u pour l{\textquoteright}occasion, nous avons ensuite extraits les topogrames de ces différents mèmes afin de comprendre et comparer les structures de leurs diffusions.

% Les hypothèses + résultats

En rappelant chacune des hypothèses formulées précédemment (voir section \ref{sec:hypotheses}), nous allons maintenant proposer une lecture critique des résultats obtenues et des démarches mises en œuvre lors de la réalisation de ce travail. 

\subsubsection{Le modèle chinois pour les industries culturelles en Chine} 

Hypothèse : \textit{La majorité des contenus circulant sur les réseaux sociaux s'apparentent largement à ceux des médias traditionnels} 

La première partie de l{\textquoteright}analyse de données portant sur les hashtags les plus mentionnés nous a permis de valider partiellement notre deuxième hypothèse en constatant que les contenus majoritairement discutés sur Sina Weibo étaient en largement similaires à ceux des médias classiques : divertissements, produits culturels, traffic routier, sports, etc. Néanmoins, nous avons aussi pu observer que les hashtags étaient la plupart du temps un artefact de campagnes de communication organisées, reflétant un usage des réseaux sociaux pour le marketing promotionnel politique, médiatique ou commercial. 

En Chine, cet intégration très rapide des grands groupes médiatiques a été rendu possible par un ensemble de conditions. La première est le protectorat économique qui a été mis en place sur tout le secteur des médias, lié à la fois à la volonté de contrôle politique mais surtout à la volonté économique des acteurs du pays. Comme nous l'avons montré (chap. \ref{sec:internet-chine}), le développement des TIC et de l'Internet a été dès le début conçu par Pékin comme un des piliers chargé de soutenir la croissance économique du pays. Ce protectionnisme sur tout le secteur des industries culturelles a permis l'apparition rapide de nombreuses firmes qui ont bénéficié d'un marché captif et non soumis à la concurrence internationale. Les opportunités de verticalisation ont donc été plus nombreuses. 

Un autre facteur important la nécessité de contrôle de la ``qualité'' des informations publiés par ces services. En effet, les fournisseurs et diffuseurs de contenus chinois sont soumis à des règles strictes de contrôle politique. Ils doivent en effet s'assurer que les contenus qu'ils publient sont ``en harmonie'' avec les règles et tendances définies par le gouvernement de Pékin. La gestion de ce risque a rendu indispensable la mise en place d'un contrôle continu des informations publiées. De ce fait, l'intégration verticale entre producteurs et distributeurs de contenu parait la solution la plus logique. Les processus de contrôle de l'information liés à la gestion des risques ``politiques'' de publication ont donc favorisé cette intégration.

Cette perspective de gérer la conformité des contenus aux législations et à l'agenda politique d'un gouvernement peut paraît au premier abord une spécificité de l'Internet chinois. En effet, les termes du débat et le cadre légal qui régit les relations entre les entreprises privées et les gouvernements est bien différents dans le cas des firmes américaines comme Facebook ou Twitter. Néanmoins, la gestion des aspects légaux de la possession et de l'usage des données devient une compétences de plus en plus centrales des services de réseaux sociaux. Culminant dans la récente ``affaire Snowden'' \cite{Greenwald2013}, les questions de vie privée viennent notamment faire pression pour davantage de responsabilité légale pour les services de réseaux sociaux. Également, l'inégalité de traitement des flux de données est déjà au centre des débats législatifs et managériaux avec la question de la neutralité des réseaux \cite{Schafer2011}. La pression grandissante pour légiférer et faire appliquer les décisions légales auprès des firmes Internet va nécessiter un contrôle accrue des différents maillons de la chaîne de production d'information. Ainsi, on peut gager que l'intégration verticale sera également une des premières solutions pour rentabiliser les coûteux dispositifs de contrôle à mettre en place. Comme dans le cas de Sina Weibo et des firmes de l'industrie médiatique en Chine, la pression légale et politique pourrait amener à une intégration des services de contenus avec les producteurs de contenus.

Du coté des contenus également, le secteur des industries culturelles est historiquement sujet à une très forte concentration \cite{Martel2010}. Les grands groupes médiatiques fonctionnent selon ce principe d'intégration verticale, diversifiant même largement leurs activités dans le secteurs. Comme nous l'avons vu avec Sina Weibo, les contenus publiés par différents médias sont déjà largement similaires et quotidiennement repris. La dichotomie entre contenus ``web'' et ``traditionnels'' s'estompe largement, avec le partage quotidien d'articles de journaux ou d'émissions de télévision sur la Toile. Les canaux de distribution eux-mêmes tendent à se recentrer, notamment autour du format mobile. Télévision web,  web documentaire, data journalisme, les pratiques tendent à se rejoindre et les services de réseaux sociaux jouent alors le rôle de diffuseur principal. Dans cette dynamique, le rapprochement entre \textit{pure players} du web comme Twitter ou Facebook et les firmes médiatiques traditionnels ne paraît pas impossible. Cette tendance se voit notamment très clairement dans la nécessité de rentabiliser au maximum le placement des informations pour l'obtention revenus publicitaires sur des écrans toujours plus petits. Les dispositifs de captation de l'attention que sont les services de réseaux sociaux deviennent la clé de voûte de l'industrie médiatique. 

Ainsi, l'histoire récente des médias chinois et notamment celle de Sina Weibo ne doit pas être lue seulement dans les termes d'une lutte politique entre les utilisateurs et le gouvernement de Pékin. La dimension stratégique du développement industriel des médias en Chine apportent aujourd'hui des clés de lecture éclairantes non seulement pour le pays, mais pour le monde entier. L'équilibre entre contrôle politique, réussite commerciale et développement industriel crée un précédent à une échelle nouvelle pour l'Internet qui ne doit pas être minimisé. L'intégration forte entre entités gouvernementales, fournisseur de contenus et canaux de distribution de l'information propose également un modèle inédit. Les études sur l'Internet chinois doivent s'intéresser davantage aux mécanismes et relations entre ces différents acteurs, en évitant de les considérer comme une seule entité. Cette relation fusionnelle de pouvoir dans le champ médiatique pourrait même être considérée comme une forme de maturité, acquise lors d'un développement hyper-rapide permis par un environnement protectionniste et autoritaire. En effet, les  discussions entourant le cadre législatif de l'Internet et les lois cherchant à réguler son usage (\textit{Hadopi} en France, \textit{SOPA} aux États-Unis, \textit{Data Protection Directive} de l'UNion Européenne) invitent à considérer l'évolution de la gouvernance du réseau Internet vers une plus forte intégration des pouvoirs dans la sphère médiatique pour une gestion des risques politiques plus contrôlée. Ainsi, l'expérience de l'Internet chinois vient changer radicalement le réseau mondial en établissant un modèle différent de celui imaginé initialement par les fondateurs du réseau ARPANET. S'il est possible d'écarter le modèle chinois de l'Internet comme un archaïsme face au discours du \textit{new-age} des géants californiens, son existence a néanmoins établi un mode de développement et de gouvernance qui s'est jusqu'ici montré d'une efficacité et d'une rentabilité sans faille et pourrait bien dans le futur séduire davantage d'états et de gouvernements.

\subsection{Plate-formes web et lieux d'énonciation}

La dimension infra-structurelle de l'Internet n'est bien sûr qu'une seule des facettes qui entourent la production de l'espace web. L'immense diversité des usages et les nombreux changements dans le quotidien des utilisateurs sont l'autre pilier de ce large développement industriel. Les services de réseaux sociaux peuvent être analysé selon deux usages majeurs qui s'entrelacent constamment : la lecture et consultation de ressources médiatiques (textes, images, vidéoas, photos, etc.) et la discussion ou conversation. La rencontre de ces deux pratiques au quotidien donne lieu à de multiples phénomènes, portant notamment la pratique du commentaire dans des sphères nouvelles.Les analyses préliminaires de notre corpus de données pour l'identification de mème ont montrés la difficulté d'isoler chacune de ces pratiques de façon satisfaisante. Le mème Internet se situe en effet à l'exacte croisée de ces deux usages, utilisant la déclinaison d'un support médiatique comme fondement de la conversation. Ce statut particulier du mème rend son approche complexe mais en fait également un objet passionnant. 

Hypothèse : \textit{Les échanges en ligne sont à comprendre comme des actes d'énonciation, dont les mèmes sont l'expression la plus actuelle.}

Les analyses de notre corpus nous ont permis de montrer que les hashtags ne recouvraient pas ce statut particulier d'objet médiatique conversationnel que nous attribuons ici au mème (voir section \ref{sec:hashtags}). Objet hybride, le hashtag restent avant tout un artefact exogène à la conversation, au-delà d'un support discursif per se. L'irruption du caractère dièse \textit{(#)} dans le langage témoigne de ce caractère peu naturel. Même s'il tend à se généraliser, il appartient davantage au domaine du dialecte propre à des plate-formes spécifiques. Son usage nécessite une appropriation, souvent par jeu de connotation ou de dénotation, et de ce fait participe d'une sous-culture des plate-formes web. A l'opposé, le mème ne se construit pas comme artefact remplissant une fonction annexe dans la conversation (identifier un évènement, un lieu, une marque) mais bien comme une discussion en soi. Comme nous avons pu le constater de manière expérimentale (voir chapitre \ref{sec:id-meme}), identifier un mème dans un ensemble de conversations est une tâche difficile. Les approches algorithmiques ne peuvent s'appuyer sur des constructions en-dehors du langage et sont donc inefficaces (voir \ref{sec:protomemes}). Finalement, c'est l'entrée par le mot lui-même qui nous a permis d'accéder à une représentation fidèle du mème (voir \ref{sec:keywords}). Le régime particulier de l'énonciation que nous avons choisi pour appréhender l'existence des mèmes nous permet de considérer différents aspects de la conversation. Ce faisant, nous mettons également à jour la limite de notre méthodologie fondée sur l'analyse de données, à jamais incapable de voir l'action ou l'intention qui a motivé un acte de parole en ligne. 

Hypothèse : \textit{L'analyse de la diffusion ne doit pas se contenter de l'analyse conversationnelle, mais doit mobiliser les modèles étudiant l'énonciation} 

Les modèles à construire de l'analyse de données doivent donc s'inspire du dialogue mais aussi prendre en compte les parties manquantes ou incomplètes lors de l'analyse. Les actes de communication observables sur les réseaux sociaux existent dans une continuité physique, sont le produit de corps qui parlent ou écrivent. Le statut de trace que constitue le texte écrit sur les réseaux sociaux donne à la conversation une nouvelle matérialité. Considérer de façon autonome l'existence textuel des messages échangés peut être intéressant dans une étude linguistique. Néanmoins, dans le cadre d'une étude sur la communication, il paraît frauduleux de réifier la conversation en un simple échange textuel. Également, considérer uniquement les conversations comme des ``flux'' entre utilisateurs échoue à prendre en compte la pré-existence de leurs relations et du contexte sur l'acte de communication. 

Le système de visualisation que nous avons choisi de développer voulait mettre en perspective les différentes dimensions de la conversation dans un même espace visuel (voir section \ref{sec:viz}). L'analyse d'un' échantillon de contenus issus de Sina Weibo nous a permis de vérifier partiellement notre hypothèse méthodologique. En donnant à voir différents aspects des échanges et conversations, nous avons cherché à observer les phénomènes de diffusion en prenant en compte quoi, qui, comment et où ce qui est dit. Le développement d'un dispositif technologique sur-mesure nous a permis d'obtenir des visualisations répondant à ces besoins. Néanmoins, la disparité des données ne nous a pas permis une granularité suffisante pour observer de façon détaillée chacun de ces aspects. Toutefois, nous avons pu montrer une image générale et observer que chacun d'eux posséder des particularités. Les différentes figures obtenues que nous nommons \textit{topogrammes} montrent plusieurs spécificités pour chaque type de contenus qui encouragent à poursuivre vers une plus grande diversité d'approche dans l'analyse des échanges en ligne.

Hypothèse : \textit{L'analyse et la visualisation de données permette d'établir une classification des discussions en ligne selon les structures de leurs diffusions.}


La lecture des différents topogrammes nous permet d'ébaucher une première typologie des contenus en ligne selon les modalités de leurs diffusions (voir fig. \ref{fig:viz-results}). Néanmoins, la dimension expérimentale de notre étude la limite à s'intéresser à un nombre restreint de mèmes. La première validation notre hypothèse méthodologique nécessiterait une systématisation à des corpus plus diversifiés afin de confirmer s'il est réellement possible de constituer une typologie générale et cohérente des \textit{topogrames} pour les différents contenus disponibles en ligne. Nos premières observations permettent néanmoins de tirer des conclusions provisoires sur les spécificités des différents types de contenus.

La diffusion des faits d{\textquoteright}actualité est entourée de grands volumes de discussion, qui lui permette de se propager rapidement. Dans le cas du fait divers, son démarrage brutal puis sa transformation par le commentaire en débat de société parfois brûlants se caractérise par une un grand nombre de communautés très actives dans la discussion pendant un laps de temps plutôt court. La conversation s'oriente généralement autour de termes simples et structurés selon des chemins assez typiques. Les conversations autour de faits d'actualité sont au début composées de communautés diverses. L'enjeu pour le contrôle de la discussion est de s'approprier le cœur de la diffusion, souvent en centralisant et fédérant l'ensemble des conversations. Cette stratégie qui cherche à couper cours à une diffusion non-programmée de l'information est particulièrement présente et étudiée en Chine. Une des stratégies est notamment de relayer et de diffuser le maximum d'information afin de recentrer la discussion autour de mots et de groupes d{\textquoteright}individus bien définis. A l'opposé, la stratégie du silence s'avère sûrement moins utile et plus dangereuse quand il s'agit d'étouffer une affaire. On observe également que les discussions qui entourant une actualité politique créent des communautés importantes de discussions qui restent néanmoins relativement distantes. Discutant sans vraiment se rencontrer, elle donne lieu à des discussions souvent très polarisée avec peu de dialogues entre les différentes communautés. La projection cartographique montre Canton qui joue un r\^ole de précurseur dans les faits d{\textquoteright}actualité et para\^it {\textquotedblleft}sortir{\textquotedblright} les affaires et Pékin qui agit comme diffuseur en dispersant l{\textquoteright}information. Ce résultat renforce néanmoins l{\textquoteright}hypothèse d{\textquoteright}une similarité entre réseaux sociaux et médias traditionnels, puisque Canton possède traditionnellement une presse plus portée sur l'investigation que celle plus officielle de Pékin. Les disparités entre Est et Ouest sont également flagrantes lors de la diffusion, avec une domination des grandes zones urbaines de l{\textquoteright}Est chinois qui, même pondérées, restent omniprésentes sur les cartes de la diffusion. Les zones de l{\textquoteright}Ouest souvent moins urbanisée semblent être moins associées aux discussions, sauf dans le cas où celles-ci les concernent directement (comme le Xinjiang et leAvec l{\textquoteright}émission de télévision \textit{The Voice}, nous avons en quelque sorte pu identifier un contre-exemple du mème. Le réseau de conversations entourant cette émission s'organise autour de très peu de diffuseurs importants entourés de nombreux fans et très peu de dialogues. La Les mèmes quant à eux semblent remettre en questionsémantique du discours est  cette dichotomie. En effet, les vastes conversations impliquent de nombreux utilisateurs très actifs, sans pour autant les associerétablie avec  réellement au sein d'un dialogue. Au contraire, les observations semblent indiquer que les mèmes agissent en faveur d'une fragmentation plus accrue du réseau. Ainsi, nous pouvons remettre en question cette définition de la  d'les résultats montrés par les mème événement natured'un champ lexical pensé en.nt On voit que des mots de ou média moindre importance sont exclus des conversations, contraints à la marginalité dans le réseau sémantique. Le diffusion de \textit{The Voice} ne possède pas à proprement parler les spécificités qu'on attendrait d'une diffusion sur les réseaux sociaux avec peu de conversations. Son topograme se caractérise par une très forte agrégation et une très faible modularité (peu de groupes fortement dominés par des diffuseurs importants). Il ne se pose pas du tout comme un lieu commun du Web, ouvert et associatif,  mais plut\^ot comme une zone d'échanges clairement identifié et territorialisé. 

L'observation des spécificités de diffusion des faits d'actualité ou des campagnes commerciales nous à tout au plus permis de vérifier des choses déjà connues : l'actualité se propage selon des cycles courts, les médias de Canton sont plus avant-gardistes que ceux de Pékin, la communication télévisuelle est soigneusement préparée, etc. Comme souvent dans les études utilisant l'analyse de données, les observations viennent infirmer des connaissances préalables sans vraiment apporter de nouveautés. 

Néanmoins, l'observation des modèles peu connus des mèmes absurdistes est une des dimensions intéressantes mises à jour dans cette étude. En effet, le topogramme des mèmes comiques possède des structures atypiques. Contrairement aux autres diffusion qui présentent des graphes conversationnels bien organisés, celui des mèmes \textit{dufu} et \textit{yuanfang} est fragmentée en nombreux petits groupes de discussions très intégrés qui communiquent très peu entre eux. Tout se passe comme si nous n'assistions pas à une vaste conversation mais plutôt à de nombreuses petites conversations. L'observation des origines géographiques des utilisateurs montrent également une grand diversité, même s'il est parfois possible de voir que la blague semble fonctionner plus ou moins bien localement. Également, le caractère anecdotique et humoristique semble être un fort vecteur de diffusion avec une présence beaucoup plus durable dans le temps que dans le cas des actualités. Leurs graphes sémantiques se composent autour de peu de mots-clés qui sont réutilisés de manière non-définitive, permettant ainsi une grande variation et une grande appropriation par des utilisateurs même isolés. Ainsi, la structuration des conversations en groupes lexicaux ouverts aux associations peu communes semblent favoriser la diffusion des mèmes et leur permettre de durer dans le temps. On voit également que les utilisateurs de Taiwan sont présent dans la diffusion des mèmes absurdistes, alors qu'ils sont absents des discussions politiques. Ainsi, il semblerait que les mèmes possèdent des plus associatives qui encouragent une participation durable.

Nous pouvons ainsi statuer que dans notre échantillon, parmi les éléments observés seuls \textit{dufu} et \textit{yuanfang} semblent réellement correspondre à la définition de mème. Les débats de société et les faits d'actualité semblent plus être des ``événements web'', somme toute reconduction de phénomènes qui pré-existent largement l'Internet. Le débat  sur les faits divers semblent avoir toujours exister, même si l'Internet en change radicalement l'échelle. Les modèles issus des médias traditionnels ont également été reconduits comme le montre l'exemple de \textit{The Voice}. La nouveauté dans cet ensemble sont bien les mèmes comiques. L'humour, s'il pré-existe heureusement à l'Internet, procède d'un régime éminemment conversationnel et prend la forme particulière des mèmes sur Internet. Le jeu de mots, le détournement d'images ou de slogans sont tous des pratiques communes qui se voient reconduits sous des formes nouvelles. L'étude de ces formes humoristiques de l'Internet permette d'accéder à des exemples de conversations qui ne sont pas nécessairement planifiées et possèdent un caractère très informelle. Ce type d'échanges presque spontané semble pourtant suivre des modèles identifiables qu'il est possible d'étudier plus en détails grâce à l'usage des topogrammes. 

Hypothèse : \textit{La circulation des contenus sur les réseaux sociaux accroît la fragmentation du milieu numérique}.

Les graphes conversationnels extraits des différents types de contenus montre la structure atomique et rhizomique des échanges en ligne. Nous avons pu voir comment les communautés de discussions se constituent. Ce modèle de discussions en groupes fondé sur l'inclusion ou l'exclusion de la conversation est un facteur de fragmentation (voir \ref{}). Les mèmes humoristiques présentent notamment la particularité intéressante de générer beaucoup de conversations pendant une longue durée, mais seulement au sein de petits groupes et non entre ses groupes. L'effet \textit{``private joke''} qui fait de la blague partagée un signe de reconnaissance mutuelle et un des rôles importants joués par les mèmes Internet (voir chap. \ref{chap:memes}). Néanmoins, dans cette circulation ne se joue pas nécessairement une ``socialisation'' conçue comme le propre des réseaux mais plutôt une fragmentation en petits groupes à l'identité forte. Cette différentiation entre les utilisateurs vers des communautés d'appartenance existent donc au sein d'une même plate-forme et se constituent lors de la circulation des contenus. 

Une lecture des graphes conversationnels pourrait permettre d'inférer la nature plus ou moins dissociatives ou associative de certains contenus. Nous avons vu comment des diffusions plus ou moins centralisées et contrôlées invitent ou non à la conversation. Nous avons également montré que les modèles dissociatifs et associatifs coexistent largement sur le réseaux social Sina Weibo, même si l'activité historique de Sina comme fournisseur de contenus semble produire un espace de discussion où les contenus plus dissociatifs dominent. A l{\textquoteright}inverse, la vivacité des débats entourant les développements économiques, sociaux et politiques de la Chine moderne donnent lieu à de nombreuses tentatives d{\textquoteright}associer la population chinoise à ces discussions au travers des médias. Les mèmes quant à eux semblent remettre en question cette dichotomie. En effet, les vastes conversations impliquent de nombreux utilisateurs très actifs, sans pour autant les associer réellement au sein d'un dialogue. Au contraire, les observations semblent indiquer que les mèmes agissent en faveur d'une fragmentation plus accrue du réseau. Ainsi, les résultats montrés par les mèmes remettent en question cette définition de la nature associative et dissociative d'un événement ou d'un média. 

\bigskip
Finalement, nos résultats semblent indiquer que plusieurs régimes d'expression et de conversation co-existent sur les réseaux sociaux. Les mèmes semblent exister sous le régime de l'énonciation, cherchant à définir des lieux intimes au sein du vaste espace d'expression qu'offre les plate-formes. A l'inverse,les campagne de marketing ou de communication s'apparente plus à un régime du discours cherchant à territorialiser un domaine en le définissant parfois même par un signe nouveau (le hashtag). Ces deux régimes co-existent largement mais ne sont pas nécessairement exclusifs. D'autres peuvent sans doute également être observés.


recommandations et portée opératoire de ton étude


\subsection[Validité et limites des méthodologies Big Data]{Validité et limites des méthodologies Big Data}


validité interne/externe (Tannery)


%GP 
%à la lumière de cela, dire si ton travail valide ta thèse de départ (totalement, un peu, beaucoup, pas du tout) et pourquoi.  
% 
%Avec prudence ( tu n{\textquoteright}es qun gravillon qui prétend s{\textquoteright}ajouter au Grand Mur de la science) . Avec quels biais éventuels, quelles limites, comment tu aurais pu mieux faire (apprends à battre ta coulpe, imagines toi à une séance d{\textquoteright}autocritique pendant la RévoCul). Quelles questions nouvelles ton travail soulève{}-t{}-il? 
% 
%à la lumière de cela, dire si ton travail valide chacune des tes hypothèses, sur le même mode. (il n{\textquoteright}y a pas de honte à reconna\^itre \ une hypothèse non vérifiée) 
% 
%à la lumière de cela, dire si ton travail \ (l{\textquoteright}analyse de tes résultats) valide tes choix méthodologiques (les algorithmes mais aussi les outils conceptuels. 

cohérence et pertinence de ces développements ingénieriques  spécifiques

Au regard des résultats, nous voyons qu{\textquoteright}il reste très difficile de qualifier de fa\c{c}on systématique une conversation en ligne et qui plus est l{\textquoteright}ensemble constitué par un milieu numérique. Le concept de topograme a permis de mettre en perspective différents aspects de la conversation pour les rendre observable et comparable et d{\textquoteright}en ébaucher une typologie mais la faible taille de notre échantillon ne nous permet pas d{\textquoteright}observer de manière empirique les structures récurrentes associés à des conversations. L{\textquoteright}apport théorique des différentes disciplines a notamment permis de problématiser notre lecture dans un contexte élargi en mettant en perspective les dimensions performatives et l{\textquoteright}existence géographique des actes de communication. Néanmoins, l{\textquoteright}approche par le \textit{milieu} a été un relatif échec. D{\textquoteright}une part, l{\textquoteright}usage de données ne contenant pas d{\textquoteright}information de géolocalisation en tant que telle (seulement une indication de lieu) ne permet pas une étude extensive des modèles géographiques mis en oeuvre lors de la diffusion des mèmes. D{\textquoteright}autre part, la définition d{\textquoteright}un {\textquotedblleft}milieu{\textquotedblright} en tant que tel pose la question de ses limites et de son déterminisme, qui reste non résolues par cette étude. Comment définir un milieu s{\textquoteright}il est propre à un usage voire à une personne? La description des technologies mises en {\oe}uvre dans les actes de communication est-elle suffisante? Comment le milieu et ses protocoles existe-t-il lors des la transduction et l{\textquoteright}individuation? Cette existence est-elle alors observable? Au regrd de ses limites conceptuelles fortes, l{\textquoteright}idée de milieu numérique gagnerait s\^urement à être problématisée autour d{\textquoteright}une définition plus précise des relations directes existantes entre le cyber-espace dans sa forme physique (les serveurs, machines, usagers, etc.) et sa manifestation {\textquotedblleft}en ligne{\textquotedblright}. L{\textquoteright}histoire des sciences et plus particulièrement de la géographie a montré comment le concept plut\^ot déterministe de milieu peut être repensé en termes d{\textquoteright}espace, de lieu et de territoire, voire de paysages offrant des prises théoriques et des approches méthodologiques nouvelles.

La plasticité théorique du terme milieu en fait également un outil faiblement opérationnel du point de vue méthodologique dans l{\textquoteright}étude de données ou de terrain. Cette limite conceptuelle inhérente s{\textquoteright}exprime ici lors de la visualisation des conversations sous forme de graphe pour l{\textquoteright}analyse : l{\textquoteright}espace de la représentation sur lequel se déroule la projection de graphe possède des propriétés topologiques encore indéfinies. Nous pouvons donc nous questionner sur la légitimité d{\textquoteright}une qualification de conversations et de pratiques numériques selon ce type de graphes. Le choix de cet espace de représentation suppose notamment que le modèle du réseau comme abstraction utile pour comprendre une conversation - ce que cette recherche après d{\textquoteright}autres a essayé de prouver. Peut-être la vision trop généralisante voire étherique induite par l{\textquoteright}usage du concept de milieu pourrait être précisé par une approche incarnée dans des lieux, des espaces et des territoires, support structurant physiquement à la fois pour l{\textquoteright}étude mais également pour la représentation.


La disponibilité des données nous a permis ici de nous livrer à un genre d{\textquoteright}analyse sur les mots et les discussions encore très récent. Nous savons néanmoins que ces données elles-mêmes sont avant tout des traces d{\textquoteright}activités passées, dans lesquels de nombreuses informations sont manquantes - pensons aux aspects de géolocalisation ainsi que l{\textquoteright}ensemble des informations non-verbales des actes de communication dans cette étude par exemple. Ainsi, l{\textquoteright}étude de phénomènes particuliers d{\textquoteright}après ces réalités {\textquotedblleft}données{\textquotedblleft} puis reconstruites conna\^it de multiples limites qui doivent être prises en compte lors de l{\textquoteright}analyse. L{\textquoteright}exigence d{\textquoteright}une connaissance étendue du terrain et la nécessité d{\textquoteright}une discussion entre plusieurs disciplines scientifiques semblent finalement un préalable aux études utilisant l{\textquoteright}analyse de données.

\subsubsection[Travaux à poursuivre]{Travaux à poursuivre} 
% 
%GP  
%tu conclues sur des travaux à poursuivre{\dots} 

La t\^ache plus large de compréhension et de description du r\^ole des échanges en ligne dépasse largement les seuls réseaux sociaux. Loin d{\textquoteright}être une exception chinoise, le contr\^ole du caractère dissociatif et associatif des nouvelles formes langagières du Web est une problématique entourant les technologies de la parole dans le monde entier. Les premiers apports méthodologiques de cette étude demandent désormais à être poursuivis, testés et vérifiés sur différents terrains d{\textquoteright}études. La description des invariants et variations au sein d{\textquoteright}un corpus plus large de mèmes ou d{\textquoteright}autres formes textuelles (e-mail, commentaires, etc.) pourrait permettre de définir les canons des nouvelles formes d{\textquoteright}énonciation en ligne, comme autant d{\textquoteright}archétypes de la fa\c{c}on dont s{\textquoteright}écrivent les lieux communs du Web. Une étude plus large concernant un ensemble de corpus de différents médias pourrait donc à terme permettre de classifier les topogrames et de dresser ainsi une typologie des actes de communication publiques ou privés. Une fois identifié, les éclairages apportés par les topogrames sur les faits médiatiques peuvent devenir à la fois un outil d{\textquoteright}analyse de la diffusion a posteriori mais également de manière plus stratégique un support pour concevoir des actes de communications. Identifier Les caractéristiques particulières de diffusion peut également permettre de caractériser qualitativement des diffusion à différents moments et offrir une lecture rapide de leur nature (diffusion de masse, faits divers, artefacts de communication, etc.) pouvant s{\textquoteright}avérer très utiles pour la vérification des sources journalistiques sur les réseaux sociaux ou la communication en situation de crise notamment. Les notions d{\textquoteright}impact et d{\textquoteright}influence pourraient également être réajustée à l{\textquoteright}aune de facteurs paramétriques et contextuels propres à des topogrames particuliers, notamment en terme de moment et de centralité dans les réseaux sémantiques, géographiques ou conversationnels. Egalement, une réévaluation du concept de milieu numérique pour une prise en compte plus importante des phénomènes physiques (appartenances territoriales, réseaux de lieux, etc.) offrirait une meilleure assise théorique à cette méthodologie d{\textquoteright}analyse permettant notamment à l{\textquoteright}analyse de données d{\textquoteright}être combiné à un travail de terrain. 