\section[Discussions]{Discussions}
 
%NB: n{\textquoteright}oublie pas d{\textquoteright}int\'egrer la dimension chinoise \`a tes r\'eflexions. 

 
L{\textquoteright}analyse et la visualisation de donn\'ees des r\'eseaux  sociaux permettent une meilleure connaissance des d\'etails de la  diffusion des contenus en ligne. N\'eanmoins, les approches couramment  utilis\'ees pour l{\textquoteright}analyse de la diffusion  d{\textquoteright}informations sur les r\'eseaux sociaux se fondent sur  des mod\`eles conceptuels tr\`es r\'educteurs  (\'emetteur-message-r\'ecepteur), faisant peu de cas des dimensions  \'enonciatives des actes de communication en ligne (fonctions, enjeux,  lieux, contexte, etc.). En nous int\'eressant \`a la diffusion des  conversations sur le service de r\'eseau social chinois \textit{Sina  Weibo}, nous avons cherch\'e ici \`a comprendre comment cette  connaissance des mod\`eles de diffusion de  l{\textquoteright}information en ligne pouvait \^etre enrichie  d{\textquoteright}apports conceptuels et m\'ethodologiques  d{\textquoteright}autres disciplines comme les th\'eories de la  communication et des syst\`emes d{\textquoteright}information, la  critique des m\'edias ainsi que la g\'eographie des technologies. Plus  particuli\`erement, nous avons entrepris de d\'evelopper un syst\`eme  d{\textquoteright}analyse et de visualisation de donn\'ees pour  explorer les dimensions s\'emantiques, conversationnelles, temporelles  et g\'eographiques de la diffusion de m\`emes Internet.   
 
 
Les particularit\'es du Web chinois nous ont d{\textquoteright}abord  permis de consid\'erer dans cette \'etude les dimensions \'economiques  et politiques des usages des r\'eseaux sociaux, avec notamment  l{\textquoteright}aspect industriel de la production de contenus web.  Toutefois, nous avons choisis de nous int\'eresser \`a un type  d{\textquoteright}usage bien particulier : les m\`emes Internet. Ces  petits objets num\'eriques connaissent souvent une grande diffusion et  offre une grande diversit\'e dans les sujets qu{\textquoteright}ils  abordent : humour, faits divers, scandales politiques, campagnes  marketing, etc.    


\subsection{Rappel des Hypoth\`eses et de la méthodologie}

\subsubsection{Hypoth\`ese 1 (m\'ethodologique) analyse enrichie de la diffusion} 

En consid\'erant un ensemble de ces m\`emes, nous pouvons identifier des r\'ecurrences dans la structure de leur diffusion en repr\'esentant sous formes de graphes multiples appel\'es \textit{topogrames} non seulement les \'echanges entre utilisateurs (conversation), mais \'egalement les structures lexicales (s\'emantique), temporelles et g\'eographiques.

\subsubsection{Hypoth\`ese 2 : r\'eseaux sociaux = mass media classiques} 

Qu{\textquoteright}il s{\textquoteright}agisse de propagande politique ou de campagnes publicitaires, la dimension strat\'egique de la diffusion de contenus fait des m\'edias un enjeu important pour tout syst\`eme \'economique et politique. Nous pensons qu{\textquoteright}il existe de grandes similarit\'es de diffusion entre les contenus majoritaires des services de r\'eseaux sociaux et ceux des m\'edias plus classiques comme la t\'el\'evision ou la radio qui sont soumis \`a des contextes de production souvent similaires.

\subsubsection{Hypoth\`ese 3: classification des discussions en ligne par leurs structures} 

L{\textquoteright}analyse de la diffusion de discussions peut permettre de d\'eterminer diff\'erents mod\`eles de conversations d{\textquoteright}apr\`es les caract\'eristiques de leurs topogrames, notamment des conversations ouvertes \`a la participation dites \textit{associatives }et d{\textquoteright}autres dites \textit{dissociatives} qui autorisent plus difficilement l{\textquoteright}\'echange. L{\textquoteright}identification de ces mod\`eles peut permettre une classification des discussions selon les modalit\'es de leurs diffusions.  

\subsubsection{Hypoth\`ese 4: milieu num\'erique et fragmentation (small worlds)} 

L{\textquoteright}observation des mod\`eles de diffusion de conversations en ligne permet de comprendre la nature dissociative ou associative de leur \textit{milieu num\'erique, }compris comme l{\textquoteright}ensemble de protocoles et outils technologiques qui les produit (Hui, 2012). Nous pensons notamment que le design actuel des services de r\'eseaux sociaux est largement dissociatif et conduit ainsi \`a une fragmentation accrue des conversations et relations en ligne (structure en small worlds), notamment \`a cause de la segmentation de l{\textquoteright}attention n\'ecessaires pour la valorisation publicitaire de son mod\`ele d{\textquoteright}affaire.  


\subsubsection{M\'ethodologie suivie} 

Afin d{\textquoteright}\'etudier ces diff\'erentes hypoth\`eses, nous avons choisi d{\textquoteright}analyser un vaste jeu \ \ de donn\'ees de 200 millions de tweets anonymis\'es comprenant \`a la fois le texte des messages ainsi qu{\textquoteright}un ensemble de m\'eta-donn\'ees dont la g\'eolocalisation par province des utilisateurs. Cet \'echantillon a \'et\'e collect\'e de mani\`ere al\'eatoire par l{\textquoteright}Universit\'e de Hong Kong pour repr\'esenter de fa\c{c}on fid\`ele l{\textquoteright}activit\'e totale de l{\textquoteright}ann\'ee 2012 sur Sina Weibo (Fu \& Chau, 2013). Notre premi\`ere analyse s{\textquoteright}est int\'eress\'e tout d{\textquoteright}abord \`a l{\textquoteright}ensemble des hashtags de l{\textquoteright}ann\'ee 2012 pour comprendre les contenus les plus discut\'es. Ensuite, nous avons s\'electionn\'e un ensemble d{\textquoteright}une dizaine de m\`emes d{\textquoteright}apr\`es leur repr\'esentativit\'e th\'ematique : comique, commercial, publicitaire, actualit\'e, faits divers et scandale politique. A l{\textquoteright}aide d{\textquoteright}un outil d{\textquoteright}analyse et de visualisation sp\'ecialement con\c{c}u pour l{\textquoteright}occasion, nous avons ensuite extraits les topogrames de ces diff\'erents m\`emes afin de comprendre et comparer les structures de leurs diffusions.

\subsection{Rappel des r\'esultats} 
\subsubsection{Analyse des hashtags (validation de l{\textquoteright}hypoth\`ese 2)} 

La premi\`ere partie de l{\textquoteright}analyse de donn\'ees portant sur les hashtags les plus mentionn\'es nous a permis de valider partiellement notre deuxi\`eme hypoth\`ese en constatant que les contenus majoritairement discut\'es sur Sina Weibo \'etaient en largement similaires \`a ceux des m\'edias classiques : divertissements, produits culturels, traffic routier, sports, etc. N\'eanmoins, nous avons aussi pu observer que les hashtags \'etaient la plupart du temps un artefact de campagnes de communication organis\'ees, refl\'etant un usage des r\'eseaux sociaux pour le marketing promotionnel politique, m\'ediatique ou commercial. Ainsi, il est impossible d{\textquoteright}infirmer que les hashtags propose un reflet de l{\textquoteright}activit\'e r\'eelle des utilisateurs.

\subsubsection{Outil de visualisation multi-graphes de la diffusion (hypoth\`ese 1)} 

La seconde \'etape de notre travail d{\textquoteright}analyse a consist\'e \`a concevoir un syst\`eme ing\'enierique permettant l{\textquoteright}\'etude et la visualisation de structures et de tendances dans les donn\'ees pour l{\textquoteright}ensemble des m\`emes s\'electionn\'es. En probl\'ematisant notre \'etude gr\^ace \`a la notion de \textit{topograme,} nous avons vu que la circulation de messages au sein du Web ne s{\textquoteright}effectuait pas seulement au sein d{\textquoteright}un r\'eseau d{\textquoteright}individus mais dans un contexte plus vaste o\`u entre en jeu la s\'emantique des mots ainsi que l{\textquoteright}espace et les lieux d{\textquoteright}utilisation. Nous avons donc proc\'ed\'e pour chaque m\`eme \`a l{\textquoteright}extraction des graphes s\'emantiques (co-occurence de mots), conversationnels (interactions entre les utilisateurs), temporels (\'evolution du volume de tweets sur une p\'eriode donn\'ee) et g\'eographiques (origine des provinces des utilisateurs lors des conversations). L{\textquoteright}\'etude de ces diff\'erents aspects de la diffusion a n\'ecessit\'e une mod\'elisation particuli\`ere des relations entre chacun de ces aspects et nous avons donc \'et\'e amen\'e \`a cr\'eer un outil d{\textquoteright}observation et d{\textquoteright}analyse permettant de visualiser sous forme de graphes multiples ces diff\'erentes lectures de la diffusion des objets digitaux.  
  
\subsubsection{Mod\`eles de m\`emes et topogrames (hypoth\`ese 3)}
 
Gr\^ace \`a l{\textquoteright}analyse des quelques \'echantillons de contenus issus de Sina Weibo, nous avons pu v\'erifier partiellement notre hypoth\`ese m\'ethodologique en lisant dans les diff\'erents topogrames des particularit\'es pour les m\`emes s\'electionn\'es. Notamment, le caract\`ere anecdotique et humoristique semble \^etre un fort vecteur de diffusion avec un succ\`es plus durable dans le temps chez les m\`emes comiques \textit{dufu} et \textit{yuanfang}. Les grands volumes de discussion entourant la diffusion des faits d{\textquoteright}actualit\'e semblent permettre \'egalement qu{\textquoteright}un m\`eme se propage dans la dur\'ee mais aupr\`es d{\textquoteright}un plus grand nombre de communaut\'es. \'Egalement, les m\`emes absurdistes ou humoristique se compose de peu de mots-cl\'es qui sont r\'eutilis\'es de mani\`ere non-d\'efinitive, permettant ainsi une grande variation et une grande appropriation par des utilisateurs m\^eme isol\'es. A l{\textquoteright}inverse un d\'ebat sur un sujet de soci\'et\'e brulant d\'emarre de mani\`ere brutale et centralise les conversations en mettant en relations des communaut\'es diverses, provoquant des concentrations de mots et d{\textquoteright}individus dans des groupes bien d\'efinis. Les discussions qui entourent une actualit\'e politique cr\'eent des communaut\'es importantes de discussions qui restent n\'eanmoins relativement distantes, discutant sans se rencontrer de fa\c{c}on plus polaris\'ee. Parmi les cas \'etudi\'es, nous avons pu identifier un contre-exemple avec l{\textquoteright}\'emission de t\'el\'evision \textit{The Voice}\textit{. }Sa diffusion en ligne morcel\'ee montre que le r\'eseau des conversations s{\textquotesingle}organise autour de peu de diffuseurs importants entour\'es de fans. Le discours est s\'emantiquement tr\`es \'etabli et inclue difficilement d{\textquoteright}autres discussions ou \'el\'ements annexes qui sont contraints \`a la marginalit\'e. Ainsi, \textit{The Voice }ne doit pas \^etre compris comme un m\`eme Internet, puisqu{\textquoteright}il ne cr\'ee pas de conversations et ne poss\`ede pas de sp\'ecificit\'es de diffusion propres aux r\'eseaux sociaux. Son topograme se caract\'erise par une tr\`es forte agr\'egation et une tr\`es faible modularit\'e (peu de groupes fortement domin\'es par des diffuseurs importants) et peut donc \^etre consid\'er\'e comme largement dissociatif. Il ne se pose pas en lieu commun du Web mais plut\^ot en lieu sp\'ecifique, clairement identifi\'e et territorialis\'e. A l{\textquoteright}inverse, la diffusion des m\`emes absurdistes ou des sujets d{\textquoteright}actualit\'e est fragment\'ee en nombreux petits groupes tr\`es int\'egr\'es et pr\'esente ainsi des qualit\'es plus associatives qui encouragent la participation. La structuration des conversations en groupes lexicaux ouverts aux associations peu communes semblent \^etre un facteur important de diffusion des m\`emes.
 
\subsubsection{Existence g\'eographique et milieu num\'erique (hypoth\`ese 4)}
 
Gr\^ace \`a l{\textquoteright}outil de visualisation interactif d\'evelopp\'e sp\'ecialement pour cette analyse, il nous a donc \'et\'e possible d{\textquoteright}observer les structures et relations mutuelles de ces graphes. Cette exploration nous a notamment permis de mettre \`a jour des caract\'eristiques sp\'ecifiques \`a certains m\`emes parmi ceux s\'electionn\'es. En s{\textquoteright}int\'eressant \`a un nombre restreint de m\`emes, notre \'etude cherchait avant tout la validation d{\textquoteright}une hypoth\`ese m\'ethodologique mais sa syst\'ematisation \`a des corpus plus diversifi\'es permettrait peut-\^etre d{\textquoteright}identifier une typologie plus g\'en\'erale des mod\`eles de \textit{topogrames} de contenus en ligne. L{\textquoteright}approche g\'eographique a n\'eanmoins donn\'e assez peu de r\'esultats, en proposant pas de structures n\'ecessairement redondantes. Ainsi, la qualification d{\textquoteright}un milieu num\'erique par la classification de ces objets digitaux selon ses topogrammes semblent une t\^ache irr\'ealisable dans cette \'etude. Les patterns g\'eographiques observables dans la diffusion des m\`emes montrent seulement une diffusion centr\'ee autour de deux p\^oles m\'ediatiques importants : Canton qui joue un r\^ole de pr\'ecurseur dans les faits d{\textquoteright}actualit\'e et para\^it {\textquotedblleft}sortir{\textquotedblright} les affaires et P\'ekin qui agit comme diffuseur en dispersant l{\textquoteright}information. Ce r\'esultat renforce n\'eanmoins l{\textquoteright}hypoth\`ese d{\textquoteright}une similarit\'e entre r\'eseaux sociaux et m\'edias traditionnels, puisque la localisation g\'eographique des entit\'es de diffusion les plus importantes sont identiques - regroup\'ees dans ces deux villes de Chine.

\begin{itemize}
    \item Canton joue un rôle d'avant-garde dans l'info. Pékin est le diffuseur principal.
    \item Absence de Taiwan des discussions politiques internes à la Chine.
    \item Concentration de l'activité dans l'Est chinois.
\end{itemize}

Egalement, notre \'etude sur les hashtags puis sur les m\`emes nous montre que les mod\`eles dissociatifs et associatifs coexistent tout deux largement sur les r\'eseaux sociaux chinois. La dimensions industrielle du fournisseur historique de contenus Sina produit un espace de discussion o\`u les contenus plus dissociatifs dominent avec une diffusion de masse proche des m\'edias traditionnels. A l{\textquoteright}inverse, la vivacit\'e des d\'ebats entourant les d\'eveloppements \'economiques, sociaux et politiques de la Chine moderne donnent lieu \`a de nombreuses tentatives d{\textquoteright}associer la population chinoise \`a ces discussions au travers des m\'edias. Le r\^ole des m\`emes \`a caract\`ere politique, marketing ou des discussions de soci\'et\'e est ici important. Les disparit\'es entre Est et Ouest sont \'egalement flagrantes, avec une domination des grandes zones urbaines de l{\textquoteright}Est chinois qui, m\^eme pond\'er\'ees, restent omnipr\'esentes sur les cartes de la diffusion. Les zones de l{\textquoteright}Ouest souvent moins urbanis\'ee semblent \^etre moins associ\'ees aux discussions, sauf dans le cas o\`u celles-ci les concernent directement (comme le Xinjiang et le Qinghai pour \textit{qiegao}).

Si les contenus diffus\'es sur Sina Weibo semblent r\'esonner en de nombreux points avec ceux des m\'edias traditionnels, nous n{\textquoteright}avons n\'eanmoins pas pu r\'eellement observ\'e le milieu num\'erique en tant que tel, \'etant donn\'e la taille \ r\'eduite de l{\textquoteright}\'echantillon de m\`emes \'etudi\'es. Egalement, l{\textquoteright}approche m\'ethodologique ne propos\'e pas une historicit\'e suffisante pour statuer des changements r\'eellement effectifs dans la structuration des relations au sein du milieu num\'erique.
 


\subsection[Validité et limites des méthodologies Big Data]{Validité et limites des méthodologies Big Data}


validité interne/externe (Tannery)


%GP 
%\`A la lumi\`ere de cela, dire si ton travail valide ta th\`ese de d\'epart (totalement, un peu, beaucoup, pas du tout) et pourquoi.  
% 
%Avec prudence ( tu n{\textquoteright}es qun gravillon qui pr\'etend s{\textquoteright}ajouter au Grand Mur de la science) . Avec quels biais \'eventuels, quelles limites, comment tu aurais pu mieux faire (apprends \`a battre ta coulpe, imagines toi \`a une s\'eance d{\textquoteright}autocritique pendant la R\'evoCul). Quelles questions nouvelles ton travail soul\`eve{}-t{}-il? 
% 
%\`A la lumi\`ere de cela, dire si ton travail valide chacune des tes hypoth\`eses, sur le m\^eme mode. (il n{\textquoteright}y a pas de honte \`a reconna\^itre \ une hypoth\`ese non v\'erifi\'ee) 
% 
%\`A la lumi\`ere de cela, dire si ton travail \ (l{\textquoteright}analyse de tes r\'esultats) valide tes choix m\'ethodologiques (les algorithmes mais aussi les outils conceptuels. 
 
Au regard des r\'esultats, nous voyons qu{\textquoteright}il reste tr\`es difficile de qualifier de fa\c{c}on syst\'ematique une conversation en ligne et qui plus est l{\textquoteright}ensemble constitu\'e par un milieu num\'erique. Le concept de topograme a permis de mettre en perspective diff\'erents aspects de la conversation pour les rendre observable et comparable et d{\textquoteright}en \'ebaucher une typologie mais la faible taille de notre \'echantillon ne nous permet pas d{\textquoteright}observer de mani\`ere empirique les structures r\'ecurrentes associ\'es \`a des conversations. L{\textquoteright}apport th\'eorique des diff\'erentes disciplines a notamment permis de probl\'ematiser notre lecture dans un contexte \'elargi en mettant en perspective les dimensions performatives et l{\textquoteright}existence g\'eographique des actes de communication. N\'eanmoins, l{\textquoteright}approche par le \textit{milieu} a \'et\'e un relatif \'echec. D{\textquoteright}une part, l{\textquoteright}usage de donn\'ees ne contenant pas d{\textquoteright}information de g\'eolocalisation en tant que telle (seulement une indication de lieu) ne permet pas une \'etude extensive des mod\`eles g\'eographiques mis en oeuvre lors de la diffusion des m\`emes. D{\textquoteright}autre part, la d\'efinition d{\textquoteright}un {\textquotedblleft}milieu{\textquotedblright} en tant que tel pose la question de ses limites et de son d\'eterminisme, qui reste non r\'esolues par cette \'etude. Comment d\'efinir un milieu s{\textquoteright}il est propre \`a un usage voire \`a une personne? La description des technologies mises en {\oe}uvre dans les actes de communication est-elle suffisante? Comment le milieu et ses protocoles existe-t-il lors des la transduction et l{\textquoteright}individuation? Cette existence est-elle alors observable? Au regrd de ses limites conceptuelles fortes, l{\textquoteright}id\'ee de milieu num\'erique gagnerait s\^urement \`a \^etre probl\'ematis\'ee autour d{\textquoteright}une d\'efinition plus pr\'ecise des relations directes existantes entre le cyber-espace dans sa forme physique (les serveurs, machines, usagers, etc.) et sa manifestation {\textquotedblleft}en ligne{\textquotedblright}. L{\textquoteright}histoire des sciences et plus particuli\`erement de la g\'eographie a montr\'e comment le concept plut\^ot d\'eterministe de milieu peut \^etre repens\'e en termes d{\textquoteright}espace, de lieu et de territoire, voire de paysages offrant des prises th\'eoriques et des approches m\'ethodologiques nouvelles.

La plasticit\'e th\'eorique du terme milieu en fait \'egalement un outil faiblement op\'erationnel du point de vue m\'ethodologique dans l{\textquoteright}\'etude de donn\'ees ou de terrain. Cette limite conceptuelle inh\'erente s{\textquoteright}exprime ici lors de la visualisation des conversations sous forme de graphe pour l{\textquoteright}analyse : l{\textquoteright}espace de la repr\'esentation sur lequel se d\'eroule la projection de graphe poss\`ede des propri\'et\'es topologiques encore ind\'efinies. Nous pouvons donc nous questionner sur la l\'egitimit\'e d{\textquoteright}une qualification de conversations et de pratiques num\'eriques selon ce type de graphes. Le choix de cet espace de repr\'esentation suppose notamment que le mod\`ele du r\'eseau comme abstraction utile pour comprendre une conversation - ce que cette recherche apr\`es d{\textquoteright}autres a essay\'e de prouver. Peut-\^etre la vision trop g\'en\'eralisante voire \'etherique induite par l{\textquoteright}usage du concept de milieu pourrait \^etre pr\'ecis\'e par une approche incarn\'ee dans des lieux, des espaces et des territoires, support structurant physiquement \`a la fois pour l{\textquoteright}\'etude mais \'egalement pour la repr\'esentation.


La disponibilit\'e des donn\'ees nous a permis ici de nous livrer \`a un genre d{\textquoteright}analyse sur les mots et les discussions encore tr\`es r\'ecent. Nous savons n\'eanmoins que ces donn\'ees elles-m\^emes sont avant tout des traces d{\textquoteright}activit\'es pass\'ees, dans lesquels de nombreuses informations sont manquantes - pensons aux aspects de g\'eolocalisation ainsi que l{\textquoteright}ensemble des informations non-verbales des actes de communication dans cette \'etude par exemple. Ainsi, l{\textquoteright}\'etude de ph\'enom\`enes particuliers d{\textquoteright}apr\`es ces r\'ealit\'es {\textquotedblleft}donn\'ees{\textquotedblleft} puis reconstruites conna\^it de multiples limites qui doivent \^etre prises en compte lors de l{\textquoteright}analyse. L{\textquoteright}exigence d{\textquoteright}une connaissance \'etendue du terrain et la n\'ecessit\'e d{\textquoteright}une discussion entre plusieurs disciplines scientifiques semblent finalement un pr\'ealable aux \'etudes utilisant l{\textquoteright}analyse de donn\'ees.

\subsubsection[Travaux \`a poursuivre]{Travaux \`a poursuivre} 
% 
%GP  
%tu conclues sur des travaux \`a poursuivre{\dots} 

La t\^ache plus large de compr\'ehension et de description du r\^ole des \'echanges en ligne d\'epasse largement les seuls r\'eseaux sociaux. Loin d{\textquoteright}\^etre une exception chinoise, le contr\^ole du caract\`ere dissociatif et associatif des nouvelles formes langagi\`eres du Web est une probl\'ematique entourant les technologies de la parole dans le monde entier. Les premiers apports m\'ethodologiques de cette \'etude demandent d\'esormais \`a \^etre poursuivis, test\'es et v\'erifi\'es sur diff\'erents terrains d{\textquoteright}\'etudes. La description des invariants et variations au sein d{\textquoteright}un corpus plus large de m\`emes ou d{\textquoteright}autres formes textuelles (e-mail, commentaires, etc.) pourrait permettre de d\'efinir les canons des nouvelles formes d{\textquoteright}\'enonciation en ligne, comme autant d{\textquoteright}arch\'etypes de la fa\c{c}on dont s{\textquoteright}\'ecrivent les lieux communs du Web. Une \'etude plus large concernant un ensemble de corpus de diff\'erents m\'edias pourrait donc \`a terme permettre de classifier les topogrames et de dresser ainsi une typologie des actes de communication publiques ou priv\'es. Une fois identifi\'e, les \'eclairages apport\'es par les topogrames sur les faits m\'ediatiques peuvent devenir \`a la fois un outil d{\textquoteright}analyse de la diffusion a posteriori mais \'egalement de mani\`ere plus strat\'egique un support pour concevoir des actes de communications. Identifier Les caract\'eristiques particuli\`eres de diffusion peut \'egalement permettre de caract\'eriser qualitativement des diffusion \`a diff\'erents moments et offrir une lecture rapide de leur nature (diffusion de masse, faits divers, artefacts de communication, etc.) pouvant s{\textquoteright}av\'erer tr\`es utiles pour la v\'erification des sources journalistiques sur les r\'eseaux sociaux ou la communication en situation de crise notamment. Les notions d{\textquoteright}impact et d{\textquoteright}influence pourraient \'egalement \^etre r\'eajust\'ee \`a l{\textquoteright}aune de facteurs param\'etriques et contextuels propres \`a des topogrames particuliers, notamment en terme de moment et de centralit\'e dans les r\'eseaux s\'emantiques, g\'eographiques ou conversationnels. Egalement, une r\'e\'evaluation du concept de milieu num\'erique pour une prise en compte plus importante des ph\'enom\`enes physiques (appartenances territoriales, r\'eseaux de lieux, etc.) offrirait une meilleure assise th\'eorique \`a cette m\'ethodologie d{\textquoteright}analyse permettant notamment \`a l{\textquoteright}analyse de donn\'ees d{\textquoteright}\^etre combin\'e \`a un travail de terrain. 