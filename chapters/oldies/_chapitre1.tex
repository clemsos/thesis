\section[Chapitre 1 : Sina Weibo et le milieu num\'erique en Chine]{Chapitre 1 : Sina Weibo et le milieu num\'erique en
Chine}
Afin de mieux définir les enjeux de la présente recherche nous allons lors de ce premier chapitre présenter le contexte de l’étude ainsi que les concepts majeurs qui y seront mobilisés. Nous débuterons ce travail en brossant un portrait de l’Internet en Chine et plus spécifiquement de Sina Weibo, service de réseau social et de microblog qui fera l’objet de cette étude. En nous appuyant sur la littérature existante, nous porterons tout d’abord un regard historique sur l’Internet en Chine, en détaillant les enjeux idéologiques, technologiques et économiques qui ont présidés à son développement. Ensuite, nous considérerons plus particulièrement l’évolution des services de réseaux sociaux en Chine. Le cas de Sina Weibo sera abordé dans le détail au travers d’observations et d’exemples, ainsi qu’une revue de la littérature qui nous permettra notamment de considérer les différences et similitudes que le service chinois entretient avec ses homologues américains Twitter et Facebook. Après cette présentation factuelle du contexte de l’étude, la seconde partie de ce chapitre introduira l’appareil conceptuel qui sera mobilisé au cours de ce travail. Considéré successivement comme fois espace, lieu et territoire, nous essaierons de comprendre comment s’articule les rapports de la Chine avec les TIC en nous appuyant sur une sélection de travaux existants. Nous proposerons enfin le concept de milieu numérique comme l’ensemble des pratiques discursives et de transduction nécessaires des objets numériques.

\subsection[L’Internet chinois : éléments de contexte  ]{L’Internet chinois : éléments de contexte}

\hypertarget{RefHeading31699228146}{}{\color{black}
La production de la recherche en sciences sociales concernant l'Internet et ses usages en Chine est largement
structur\'ee autour des int\'er\^ets politiques et \'economiques qui l'entourent. Les premi\`eres \'etudes
acad\'emiques sont issues du champ de l'information-communication et plus sp\'ecifiquement du monde anglo-saxon. Leurs
approches reposent sur le postulat qu'Internet serait un outil favorisant la d\'emocratisation. Ainsi cette
litt\'erature peut se r\'esumer en un \'echange d'arguments autour d'une alternative : l'entr\'ee de la Chine dans la
{\textquotedbl}soci\'et\'e de l'information{\textquotedbl} va de pair avec la modernisation \'economique et la
d\'emocratisation ou bien avec un contr\^ole politique accru de l'information et global de la soci\'et\'e. Ces
questions de libert\'e d'expression et du rapport entre d\'emocratie et m\'edias occupent encore aujourd'hui une large
place (MacKinnon, 2009 ; Douzet, 2007 ; Yang, 2008). En effet, la crainte des autorit\'es chinoises est grande de voir
se d\'evelopper de nouveaux canaux de circulation et de nouvelles sources d'information \'echappant \`a leur autorit\'e
traditionnelle sur les m\'edias - et contestant ainsi leur pouvoir. D'un autre c\^ot\'e, le m\'edia internet est
per\c{c}u par le gouvernement comme un agent de changement dans le processus de modernisation politique et \'economique
(dont l'ouverture de march\'es commerciaux int\'erieurs) permettant de faire circuler l'information de fa\c{c}on
optimale, en touchant l'ensemble des foyers. En somme, un Internet {\textquotedbl}sain{\textquotedbl} et ma\^itris\'e
soutiendrait le d\'eveloppement du syst\`eme de valeurs de la politique, de l'\'economie et de la culture. De
nombreuses publications se sont \'egalement int\'eress\'ees au r\^ole de l'\'economie en ligne dans la croissance
\'economique de la Chine et les liens \'etroits des entreprises web avec la censure (Dann \& Haddow, 2008).}


\bigskip

{\color{black}
Les \'etudes propres aux usages d'Internet dans le contexte chinois sont toutefois moins nombreuses ou sont
g\'en\'eralement le fruit des services marketing des entreprises locales ou internationales en qu\^ete de nouveaux
march\'es (Hwang, 2005, Bergstrom, 2012). Toutefois, on retrouve \`a la crois\'ee de ces diff\'erentes discussions des
\'etudes plus pr\'ecises qui cherchent \`a mettre en relation les dimensions \`a la fois \'economique et d'usage du web
chinois (Puel, 2009 ; Fernandez, Puel 2010 ; Shen 2012). Dans le domaine pr\'ecis de l'analyse des r\'eseaux sociaux
chinois, les publications internationales en sciences sociales restent encore peu nombreuses. Le champ de la recherche
en informatique propose n\'eanmoins plusieurs \'etudes, s'attelant notamment \`a d\'ecrire les dynamiques des
\'echanges de contenus (Asur, 2011) et les dispositifs de censure en place (Yu, 2012). }


\bigskip

{\color{black}
Afin de mieux se situer dans le large paysage de cette litt\'erature, cette premi\`ere partie rend compte des principaux
\'el\'ements historiques, politiques et \'economiques d'infra et d'infostructure de l'internet chinois.}

\subsubsection[1.1.1. Petite histoire et \'evolution de l{}'Internet chinois]{1.1.1. Petite histoire et \'evolution de
l'Internet chinois}
\hypertarget{RefHeading51699228146}{}{\color{black}
Avec plus de 560 millions d'utilisateurs connect\'es, l'Internet chinois est aujourd'hui le plus grand r\'eseau
national, d\'epassant depuis 2011 l'Am\'erique du Nord et l'Europe r\'eunies (CNNIC,
2012)\footnote{\textsf{\ }\textsf{Et alors m\^eme que seuls 40\% de ses habitants disposent \`a ce jour d'une connexion
:}\textsf{\textit{{\guillemotleft}~La collection des donn\'ees se fait gr\^ace \`a des logiciels, questionnaires en
ligne et sondages par t\'el\'ephones ({\dots}) Pour le CNNIC, un internaute est un citoyen chinois ag\'e de plus de 6
ans qui utilise Internet au moins une heure par semaine ou a utilis\'e Internet durant les 6 derniers
mois~{\guillemotright} }}\textsf{(d'apr\`es le site officiel de CNNIC, agence officielle du gouvernement).}}. Alors que
l'installation des infrastructures a d\'ebut\'e seulement en 1996, l'expansion en a \'et\'e fulgurante (Fang, 2006). 51
millions de nouveaux internautes chinois ont fait leur apparition en 2012, soit une hausse de 10\% par rapport \`a 2011
(CNNIC, 2012). }


\bigskip

{\color{black}
Comme ailleurs dans le monde, les d\'ebuts des r\'eseaux informatiques en Chine se d\'eroulent dans le contexte
universitaire avec la cr\'eation en 1994 du CERNET (\textit{China Education and Research Network}) permettant de relier
plusieurs grandes universit\'es du pays. Le 17 mai 1994, la Chine effectue sa premi\`ere connexion au r\'eseau Internet
en se reliant au \textit{Stanford Linear Accelerator Center }(SLAC) de l{}'Universit\'e de Stanford aux \'Etats-Unis.
L'ann\'ee suivante, les infrastructures backbones n\'ecessaire \`a l'installation de l'Internet \`a plus large
\'echelle sont effectu\'ees (ChinaNet, GBNet, CERNET) et la premi\`ere licence d'exploitation commerciale est
attribu\'ee, effective d\`es 1996. Le 1er janvier 1997, le Quotidien du Peuple lance sa premi\`ere version en ligne,
devenant ainsi le premier site Internet d'information officielle du pouvoir central Chinois sur Internet. Dans le
courant de l'ann\'ee, le premier FAI priv\'e chinois China InfoHighway voit le jour. En Novembre, le CNNIC (China
Internet Network Information Center) est charg\'e recueillir et publier des statistiques sur le d\'eveloppement de
l'Internet en Chine (Dai, 2007).}


\bigskip

{\color{black}
D\`es son commencement, la structuration et le contr\^ole du r\'eseau Internet est un enjeu important pour le
gouvernement de P\'ekin. Depuis le cin\'ema sovi\'etique, les dirigeants communistes ont appris \`a consid\'erer
s\'erieusement les nouveaux outils de communication, \`a la fois comme un risque et une opportunit\'e. Lors de
l'arriv\'ee d'Internet, le pays s'est engag\'e depuis plus de 20 ans dans de vastes r\'eformes (gaige kaifang) sous le
signe de l'ouverture et du changement, apr\`es avoir mis fin \`a la R\'evolution Culturelle. Les dirigeants de P\'ekin
sont bien conscients que la place forte que la Chine r\'eclame dans le paysage mondial se dessinera notamment par une
int\'egration plus accrue dans le r\'eseau mondial des TIC. En Mars 2000 alors que la population des internautes
chinois atteint 16,9 millions d'internautes, le premier ministre Jiang Zemin prononce un discours :}


\bigskip

{\color{black}
{}``Internet technology is going to change the international situation, military \ \ combat, production, culture, and
economic aspects of our daily life\ \ significantly.''}

{\raggedleft\color{black}
Jiang Zemin, Mars 2000 (Foster \& Goodman, 2000)
\par}


\bigskip

{\color{black}
Alors qu'Al Gore et l'administration Clinton d\'eploient leur strat\'egie des ``information superhighways'' aux
\'Etats-Unis, le programme d'informatisation et de d\'eveloppement des TIC (xinxihua) devient une des cl\'es du
calendrier politique et \'economique chinois. Jiang Mianheng, fils du premier ministre Jiang Zemin, est charg\'e du
d\'eploiement ce vaste projet. Revenu en 1992 apr\`es plusieurs ann\'ees pass\'ees dans les universit\'es am\'ericaines
puis dans la Silicon Valley chez HP, Jiang Manheng investit donc largement dans les infrastructures et soutient d\`es
1999 le lancement du haut-d\'ebit en Chine avec la compagnie Netcom, aujourd'hui encore deuxi\`eme op\'erateur du pays
(Dai, 2007). Fortement influenc\'e par les th\'eories de Alvin Toffler (Tsui, 2007), le gouvernement de Jiang Zemin est
bien d\'ecid\'e \`a ne pas rater la ``troisi\`eme vague'' de modernisation de l'industrie : l'informatisation. La mise
en place d'une quinzaine de ``golden projects'' informe le mouvement de strat\'egie globale qui doit dynamiser
transversalement tous les secteurs d'activit\'e jusqu'au au sein-m\^eme de l'administration chinoise. Alors que les
``golden customs'' se charge des donn\'ees du commerce ext\'erieur et que la ''golden sea'' devient un outil de
communication entre cadres du Parti et administrations locales, les multiples ``golden projects'' sont soutenus par la
volont\'e de mettre en place un ``e-government''. P\'ekin veut faire de l'Internet chinois un espace de dialogue entre
citoyens et organes du gouvernement avec notamment la mise en place de services administratifs en ligne et d'enqu\^etes
transversales sur toute la Chine, notamment sur la qualit\'e de vie dans les diff\'erentes villes. La gestion proactive
de l'Internet national offre \'egalement au Parti un vecteur unique pour la diffusion de ses id\'ees politiques.
Plusieurs \'etudes montrent comment les id\'ees et discours nationalistes sur l'Internet ont b\'en\'efici\'e d'un
soutien constant du gouvernement chinois, avec notamment comme objectifs l'acc\`es aux communaut\'es chinoises
\'emigr\'ees \`a l'\'etranger (Hugues, 2000). Finalement, le m\'edia Internet doit servir les int\'er\^ets du
gouvernement et participer \`a l'unification territoriale, \`a l'instar des autres m\'edias dits classiques comme la
t\'el\'evision. }

\subsubsection[1.1.2. Censure sur l{}'Internet chinois]{1.1.2. Censure sur l'Internet chinois}
\hypertarget{RefHeading71699228146}{}{\color{black}
Les modes d'adoption de la technologie Internet par le gouvernement de P\'ekin illustrent pourtant bien le dilemme
constant entre ouverture et protectionnisme qui tiraille la classe politique chinoise depuis la fin de la R\'evolution
Culturelle. L'ouverture au monde et la politique de r\'eformes du ``nouveau d\'epart'' s'accompagne pour la Chine de
nombreux d\'efis souvent consid\'er\'es comme ext\'erieurs, que r\'esume avec clart\'e une phrase c\'el\`ebre
attribu\'ee \`a Deng Xiaoping : {\textquotedbl}Lorsque vous ouvrez une fen\^etre pour avoir de l'air frais, vous devez
vous attendre \`a ce que quelques mouches rentrent dans la pi\`ece.{\textquotedbl} Alors que les promesses d'expansion
\'economique et politique de l'Internet sont bien au rendez-vous, les ``mouches'' entr\'ees par les fen\^etres des
navigateurs web commencent \`a s'essaimer pour faire de plus en plus bruit. }


\bigskip

{\color{black}
Depuis 1998, le Minist\`ere de la S\'ecurit\'e Publique chinois travaille \`a la conception d'un projet intitul\'e
Golden Shield qui constituerai un fichier global des citoyens chinois utilisable tant pour le contr\^ole de la
d\'emographie que le vol de v\'ehicules ou la s\'ecurit\'e aux fronti\`eres (Lyons, 2009). Lors d'un Trade Show
intitul\'e ''Security China 2000'' se d\'eroulant \`a P\'ekin en 2000, le projet est pr\'esent\'e publiquement comme
une large base de donn\'ees devant regrouper les informations administratives des citoyens et leurs activit\'e en ligne
dans le but de favoriser le travail de la police (Walton, 2001). Titanesque et complexe \`a r\'ealiser, le projet est
peu \`a peu modifi\'e pour devenir un syst\`eme de filtrage de contenus et de blocages de sites, bas\'e sur le
syst\`eme du firewall. Fang Binxing, un professeur sp\'ecialis\'e dans la s\'ecurit\'e informatique \`a l'Universit\'e
de Harbin est nomm\'e chef-ing\'enieur du projet Golden Shield. Il recrute de nombreux ing\'enieurs et avec l'aide de
l'universit\'e de Qinghua et de plusieurs entreprises occidentales (Nortel
Networks\footnote{\ \textit{{}``}\textit{Nortel Networks signs contracts valued at over USD 120 Millions during
Canadian Prime Minister Jean Chretien's visit to China''}, PR Newswire,
\url{http://www.prnewswire.co.uk/news-releases/nortel-networks-signs-contracts-valued-at-over-us-dollars-120-million-during-canadian-prime-minister-jean-chretiens-visit-to-china-156751255.html,}
consult\'e le 17 F\'evrier 2014 \`a 12:18}, Cisco\footnote{\ {}``\textit{Cisco Leak:
}\textrm{\textit{{}`}}\textit{Great Firewall}\textrm{\textit{{}'}}\textit{ of China Was a Chance to Sell More
Routers'', }\par Sarah Lai Stirland, 05.20.2008, \url{http://www.wired.com/threatlevel/2008/05/leaked-cisco-do/}
consult\'e le 17 F\'evrier 2014 \`a 12:39}...) at\`ele au d\'eveloppement technologique du projet qui devrait devenir
le syst\`eme de contr\^ole de l'Internet chinois aujourd'hui en activit\'e. Baptis\'e depuis de mani\`ere informelle
``The Great Firewall'' (GFW) par analogie avec la ``Grande Muraille de Chine'' (Great Wall), ce syst\`eme
{\guillemotleft}~sociotechnique~{\guillemotright} hors du commun est aujourd'hui consid\'er\'e comme une des plus
grandes installations d'analyse et de traitement de donn\'ees en activit\'e. A chaque seconde, GFW traite et scanne des
millions de chaines de caract\`eres issues des requ\^etes et pages vues par des centaines de millions d'internautes
(Winter, 2012). Au-del\`a de la censure automatique, GFW emploierai aujourd'hui entre 30000 et 50000
personnes\footnote{\ \textit{{\textquotedbl}What is internet censorship?{\textquotedbl}.} Amnesty International
Australia. 28 Mars 2008. Consult\'e le 17 F\'evrier 2014 \`a 15h12} : ing\'enieurs, mod\'erateurs, relecteurs,
officiers de police, etc. Ph\'enom\`ene particulier, un groupe de r\'edacteurs est notamment charg\'e d'intervenir dans
les discussions ou les forums en ligne pour faire valoir le point de vue officiel. L'adage veut que chacun des messages
qu'ils postent soient r\'emun\'er\'e 50 centimes RMB, ce qui a amen\'e les internautes chinois \`a baptiser ces
repr\'esentants de l'ordre politique en ligne ``le Parti \`a 50 centimes'' (wumao dang).}


\bigskip

{\centering \par}
\begin{center}
\includegraphics[width=5.0402in,height=2.8299in]{chapitre1-img001.jpg}
\end{center}
{\color{black}
Fig. \textit{Jingjing }et \textit{chacha }sont deux figures cr\'e\'ees par les autorit\'es chinoises pour signifier la
pr\'esence polici\`ere en ligne aux internautes. Le nom de ces veilleurs dessin\'es est une composition issue du mot
``police'' en chinois (\textit{jingcha). }Source : \url{http://jswm.newssc.org/system/2008/10/29/011233423.shtml}
consult\'ee le 17 F\'evrier 2014, \`a 15:32}


\bigskip

{\color{black}
Au-del\`a de l'activit\'e manuelle de milliers d'employ\'es, GFW op\`ere \'egalement plusieurs type de blocages sur les
contenus. Techniquement, la plupart du filtrage se d\'eroulent au niveau du fournisseur d'acc\`es avec notamment des
adresses particuli\`eres qui sont rendus inaccessible : l'adresse facebook.com ou youtube.com renvoie une erreur ``404
: Le site demand\'e n'existe pas''. Ainsi, de nombreux sites c\'el\`ebres ne sont pas accessibles (Twitter, Youtube,
Faceboo, etc.) Les autres blocages effectifs s'effectuent selon les adresses IP ou les serveurs d'attribution de noms
de domaine (DNS) menant parfois au blocage de serveurs entiers (Winter\&Lindskog, 2012). Les URLs des pages de certains
sites importants sont \'egalement filtr\'ees. Sur Wikipedia notamment, la page ``Tiananmen Square protests of 1989''
est inaccessible depuis la Chine sans que le site Wikipedia soit pour autant int\'egralement bloqu\'e. \'Egalement, des
requ\^etes contenant des mots ``interdits'' sur les moteurs de recherche peuvent conduire \`a des micro-coupures ou un
acc\`es restreint au web pendant parfois plusieurs minutes\footnote{\ Test\'e depuis Shanghai en Septembre 2013.}. La
liste des sites et mots bloqu\'es n'est pas publi\'ee par le gouvernement et l'ajout sur ses listes s'effectue a priori
sur la demande de diff\'erentes agences gouvernementales chinoises, sans notification publique. Essentiellement, il
s'agit de sites \`a caract\`ere pornographiques (la pornographie est ill\'egale en Chine), de sites li\'es aux groupes
dissidents chinois (Falung Gong, Dalai Lama, Tibet Libre...), de sites du gouvernement taiwanais et d'autres sites
revendiquant la libert\'e d'expression pour la Chine\footnote{\ Si la liste des site n'est pas publi\'ee
officiellement, une liste est n\'eanmoins maintenue par une entreprise priv\'ee sur le site
\url{http://www.greatfirewall.biz/,} consult\'e le 17 f\'evrier 2014.}.}


\bigskip

{\color{black}
Si le GFW est un outil de contr\^ole politique, il participe \'egalement largement au protectorat \'economique chinois.
L'expansion rapide du vaste march\'e de l'Internet en Chine s'est faite sous un strict contr\^ole politique et si
l'\'etat a largement financ\'e les infrastructures, GFW a \'et\'e l'un des moteurs de la croissance des grandes
soci\'et\'es du web chinois. L'absence du g\'eant YouTube a notamment permis aux acteurs locaux de la vid\'eo en ligne
de se d\'evelopper rapidement, comme Youku, aujourd'hui sujet \`a une valorisation colossale sur les march\'es
d'affaires. La situation est similaire pour les r\'eseaux sociaux. L'absence de concurrence \'etrang\`ere due \`a
l'interdiction~de Facebook en 2008 puis Twitter en 2009 a permis aux acteurs chinois de se d\'evelopper. Leur poids sur
le march\'e int\'erieur les autorise aujourd'hui \`a rivaliser avec leurs concurrents am\'ericains au niveau mondial
tant en nombre d'utilisateurs (CIC, 2012) qu'en revenus directs et indirects g\'en\'er\'es (CIW, 2012).}


\bigskip

{\color{black}
Ainsi, GFW a affect\'e l'\'economie du pays en profondeur et \`a ce titre notamment continue d'\^etre une
pr\'eoccupation premi\`ere du pouvoir politique. L'\'evolution technologique de GFW suit de pr\`es l'\'evolution des
moyens de contournement des blocages, qui sont nombreux. Ainsi, le c\'el\`ebre logiciel Tor garantissant l'anonymat sur
Internet est aujourd'hui bloqu\'e en Chine (Winter\&Lindskog, 2012) ainsi que d'autres technologies communes
d'anonymisation (comme le proxy notamment). Pourtant, il reste tr\`es facile de ``faire le mur'' (fanqiang) et
d'acc\'eder aux contenus en contournant les limites du GFW. Les solutions techniques \`a disposition sont multiples et
souvent peu couteuses \`a mettre en place ou \`a utiliser. De nombreux services commerciaux proposent de se connecter
depuis d'autres pays \`a l'aide d'un VPN (Virtual Private Network) pour un co\^ut tr\`es faible. Le VPN permet de se
connecter depuis une machine situ\'ee dans un autre pays et de b\'en\'eficier ainsi de l'acc\`es tel qu'il existe dans
le pays o\`u se situe la machine. Le flou juridique qui entoure l'existence de services commerciaux de contournement de
GFW t\'emoigne de l'int\'er\^et du gouvernement chinois \`a prot\'eger largement le march\'e int\'erieur en supprimant
l'acc\`es aux services majoritaires du web occidental, sans pour autant exercer une traque syst\'ematique de chaque
personne voulant utiliser Facebook ou Gmail. De plus, l'absence totale de moyens de contournement interdirait l'acc\`es
\`a des sources pr\'ecieuses d'informations professionnelles (notamment Twitter) et se ferai donc au prix d'une perte
d'avantages concurrentiels d\'ecisifs pour les entreprises chinoises, que les autorit\'es chinoises ne semble pas
d\'esirer. Une \'etude de l'OpenITP parue en 2013 analyse l'usage des outils de contournement de la censure aupr\`es
d'un \'echantillon de 1175 utilisateurs en Chine. Au-del\`a des solutions technologiques vari\'ees, on peut noter que
la premi\`ere raison pour contourner le blocage de l'Internet et l'utilisation des services de Google (notamment de
Gmail, dont le d\'ebit est tr\`es lent depuis la Chine), suivi de la volont\'e de ce rendre sur les sites des OSNS
am\'ericains comme Facebook et Twitter, puis de l'acc\`es aux contenus d'actualit\'e, de vid\'eo en ligne et de
mat\'eriel \`a caract\`ere pornographique. Les utilisateurs souhaitant acc\'eder \`a des contenus \`a caract\`ere
politique ou utilisant l'Internet de mani\`ere anonyme pour communiquer de fa\c{c}on plus s\'ecuris\'ee repr\'esente
moins de 10\% de la population \'etudi\'ee (OpenITP, 2013). Il est \'egalement important de noter que l'immense
majorit\'e des internautes chinois n'utilisent pas de syst\`eme de contournement des blocages de l'Internet. }

\subsection*{1.2. M\'edias sociaux en Chine : un paysage morcel\'e }
\hypertarget{RefHeading91699228146}{}{\color{black}
Alors qu'un blocage officiel s'applique donc sur les plus c\'el\`ebres services Internet am\'ericains, de nombreux
services se sont d\'evelopp\'es pour r\'epondre aux besoins et int\'er\^ets des internautes chinois. Cette situation
particuli\`ere influe sur les contenus diffus\'es et les usages possibles du r\'eseau par ses membres. Au-del\`a des
caract\'eristiques politiques et \'economiques du web chinois, d'autres ph\'enom\`enes li\'es \`a ce d\'eveloppement
technologique particulier viennent influencer la culture des entreprises web et l'\^etre-ensemble des usagers en ligne.
Dans la seconde partie de ce chapitre, nous allons pr\'esenter les diff\'erents services de r\'eseaux sociaux en ligne
(SNS) les plus utilis\'es en Chine. Nous nous attarderons sur le cas de Sina Weibo qui sera utilis\'e comme support
pour la suite de la pr\'esente \'etude. }


\bigskip

{\color{black}
Profitant de l'absence des grands noms du r\'eseau social en ligne, de nombreux services ont vu le jour sur la Toile
chinoise. L'importance du guanxi (Yu, 2008), \'el\'ement profond de la culture traditionnelle poussant chaque Chinois
\`a entretenir et exposer avec soin ses relations en soci\'et\'e, peut \'egalement avoir contribu\'e \`a cr\'eer un
terrain id\'eal pour le d\'eveloppement rapide de ces sites (Arbor, 2011). Plut\^ot morcel\'e, le paysage des SNS en
Chine offre une vari\'et\'e de services et d'acteurs qui rassemble les internautes chinois selon leurs centres
d'int\'er\^ets. Douban offre aux jeunes {\guillemotleft}~branch\'es~{\guillemotright} de partager lectures, films et
musique. Kaixin001, plus centr\'e sur les jeux, propose un espace ludique pour les trentenaires au bureau. Renren
(anciennement Xiaonei) est, quant \`a lui, un v\'eritable clone de Facebook et se focalise sur le monde \'etudiant
chinois (Renaud, 2011). Malgr\'e ces nombreux concurrents, c'est le service le service de messagerie instantan\'ee QQ
qui reste le leader incontest\'e du march\'e chinois. Aujourd'hui class\'e 8\`eme site le plus visit\'e au
monde\footnote{\ \textstyleappleconvertedspace{D'apr\`es
}Alexa\textstylereferenceaccessdate{.com}\textstylereferenceaccessdate{, consult\'e le 3 F\'evrier
2013.}\textcolor{black}{ }}, QQ d\'enombre jusqu'\`a 100 millions d'utilisateurs connect\'es
simultan\'ement\footnote{\ Voir http://im.qq.com/culture/\textstyleInternetlink{, }consult\'e le 14 F\'evrier 2013.}.
Plus qu'une simple messagerie de chat, les services de QQ sont multiples : avatar, espace personnel, blog, monnaie
virtuelle, jeux, rencontres, etc. Consid\'er\'e comme la quatri\`eme plus grande firme du web mondiale, son cr\'eateur
le g\'eant Tencent Holdings Limited a investi depuis le d\'ebut de l'Internet en Chine dans de nombreux domaines des
TIC : jeux, publicit\'e en ligne, e-commerce, etc. R\'eseau social avant l'heure, l'utilisation de la messagerie QQ est
devenu un v\'eritable ph\'enom\`ene de soci\'et\'e port\'e par la diversification de Tencent dans de multiples secteurs
sous une marque unique. Au-del\`a des jeunes et des professionnels de l'Internet, le r\'eseau QQ comptait en juillet
2011 plus de 812,3 millions de comptes actifs, faisant de lui le deuxi\`eme r\'eseau social mondial apr\`es Facebook.
Du magasin de photocopie de quartier au r\'eseau de prostitution clandestin, QQ h\'eberge les discussions quotidiennes
et fait pour ainsi dire partie int\'egrante du paysage des villes modernes. Les chinois \'echangent plus volontiers
leurs num\'eros de QQ que ceux de leurs t\'el\'ephones portables et ce mode de communication est souvent pr\'ef\'er\'e
au mail dans les \'echanges au bureau. Convergeant rapidement avec la croissance fulgurante du e-commerce en Chine, les
produits d\'eriv\'es estampill\'es QQ sont devenu une v\'eritable mode en Chine : voitures, t\'el\'ephones, boissons,
etc. Le groupe Tencent poursuit son \'evolution avec le lancement en 2008 son service de microblog Tencent Weibo, qui a
connu un v\'eritable succ\`es d\`es les premiers mois. Aujourd'hui, la firme de Shenzhen poursuit la conversion de ces
utilisateurs QQ vers sa plateforme mobile WeChat qui connait actuellement une tr\`es forte croissance, au point de voir
les autres services de microblog mis au banc par les utilisateurs. A l'origine application de messagerie \'ecrite et
vocale, Tencent continue de se diversifier en offrant d\'esormais d'utiliser son compte QQ comme moyen de paiements
pour de nombreux services du quotidien (taxis, nourritures, etc.)\footnote{\ \textit{{}``}\textit{21 million taxi rides
have been booked on WeChat in the past month''}, Tech in Asia February 12, 2014
\url{http://www.techinasia.com/wechat-21-million-taxi-rides-booked/} consult\'e le 17 F\'evrier 2014 \`a 15:36}.}

\subsubsection[1.2.1. Microblog en Chine et Sina Weibo]{1.2.1. Microblog en Chine et Sina Weibo}
\hypertarget{RefHeading111699228146}{}{\color{black}
Les sites de microblogging (en chinois weibo) permettent aux utilisateurs de poster de courts messages compos\'e de
photos ou de texte de 140 caract\`eres maximum, puis de les commenter et de les partager avec leurs lecteurs. A l'image
de Twitter, chaque utilisateur peut souscrire aux fils d'info d'autres utilisateurs afin de recevoir leurs messages et
mises \`a jour.}


\bigskip

{\color{black}
L'histoire du microblog en Chine d\'ebute en 2007 avec plusieurs services se pr\'esentant alors comme des clones de
Twitter. Le service Fanfou connait notamment un succ\`es rapide alors que de nombreux journalistes l'utilisent pour
enqu\^eter et coordonner leurs actions lors de l'arriv\'ee du SRAS ou le tremblement de terre de Wenchuan dans le
Sichuan en 2008. Le service qui conna\^it un rapide succ\`es est ferm\'e sur ordre du gouvernement en Juillet 2009,
suite aux nombreux commentaires suscit\'es par des \'emeutes s'\'etant d\'eroul\'ees \`a Urumqi dans la province du
Xinjiang. A peine un mois apr\`es cette fermeture, la firme SINA Corporation saisit l'opportunit\'e et s'installe en
lan\c{c}ant son propre service de microblog intitul\'e Sina Weibo. Sina Weibo conna\^it une croissante soutenue avec
plus de 10 millions de nouveaux inscrits par mois et devient en 2012 la plateforme de microblog la plus utilis\'ee en
Chine avec 250 millions d'utilisateurs (Milhard, 2012). Les revenus de Sina Weibo ne cessent alors de cro\^itre (+19\%
en 2012\footnote{\ Voir http://corp.sina.com.cn/chn/Annual\_Report\_2011\_Final.pdfnvestment, consult\'e le
16/02/2013.}) alors que plus de 86 millions de messages sont post\'es chaque jour sur ce service\footnote{\ D'apr\`es
Sina Corp. Earning Calls - http://phx.corporate-ir.net/phoenix.zhtml?c=121288\&p=irol-EventDetails\&EventId=4727394,
consult\'e le 16/02/2013.}. }


\bigskip

{\centering \par}
\begin{center}
\includegraphics[width=14.2291in,height=8in]{chapitre1-img002.png}
\end{center}
{\color{black}
Fig. Capture d'\'ecran de Sina Weibo, r\'ealis\'e le 9 Avril 2013 \`a 08:59}


\bigskip

{\color{black}
Figure historique de l'Internet chinois, SINA Corporation est c\'el\`ebre pour son portail sina.net et son immense
plateforme de blogs qui en font le fleuron des fournisseurs de contenus en ligne en Chine. Sp\'ecialis\'e dans
``l'infotainment'' (un m\'elange tr\`es tablo\"id d'actualit\'e et de news people), SINA est la premi\`ere compagnie
nationale chinoise a avoir \'et\'e list\'ee au NASDAQ d\`es Avril 2000. Avec son Weibo, la firme r\'eussit un coup de
force commercial et prouve une fois encore combien la censure gouvernementale est b\'en\'efique \`a l'industrie du web
chinois. N\'eanmoins, la r\'eussite de SINA et de son service de microblog ne se fait pas sans conna\^itre de nombreux
ajustements parfois chaotiques. En effet, la strat\'egie agressive d'acquisition d'utilisateurs qui pousse la
croissance Sina Weibo se fait avec la garantie pour les utilisateurs de pouvoir discuter et s'informer plus facilement
en ligne. D\`es le d\'ebut de l'ann\'ee 2010, les suppressions de comptes utilisateurs et de messages non d\'esir\'es
commencent \`a se r\'epandre dans le service. Les discussions politiques sont r\'eguli\`erement effac\'es et Sina se
voit contraint de mettre en place un syst\`eme de censure efficace sous la pression du gouvernement de P\'ekin.
N\'eanmoins, afin de continuer \`a garantir la croissance du service, la firme de P\'ekin laisse une relative libert\'e
aux utilisateurs en \'etant plut\^ot large sur la surveillance des discussions et les actions prises. Des
personnalit\'es publiques ou journalistes devenues ``weibo-stars'' mobilisent r\'eguli\`erement l'opinion publique
autour des sujets d'actualit\'e attirant souvent des millions de lecteurs et de commentaires. Plusieurs scandales
\'eclatent en ligne, mettant en cause des officiels et leur famille\footnote{\ Le fils d'un haut-cadre du Parti
arr\^et\'e ivre par la police apr\`es avoir renvers\'e 5 personnes et annonce : \textit{{}''Mon p\`ere s'appelle Li
Gang'' }et se voit imm\'ediatement lib\'er\'e, provoquant un toll\'e chez les internautes
\url{http://www.chinadaily.com.cn/china/2011-03/02/content_12099500.htm}}. Le 23 Juillet 2011, deux trains d\'eraillent
sur la ligne reliant Ningbo \`a Wenzhou qui venait d'\^etre inaugur\'ee en fanfare quelques jours auparavant, faisant
pr\`es de 40 morts et 200 bless\'es. La col\`ere gronde alors que le gouvernement tarde pendant plusieurs jours \`a
prendre la parole sur ce sujet d'actualit\'e \'epineux. Sur la toile et Sina Weibo en particulier, les discussions vont
bon train et les internautes indign\'es commentent le dernier drame du d\'eveloppement trop rapide de la Chine, o\`u se
m\^ele d\'etournement d'argent public, corruption et s\'ecurit\'e des passagers. }


\bigskip



\begin{center}
\includegraphics[width=5.8646in,height=2.9374in]{chapitre1-img003.jpg}
\end{center}
{\color{black}
Fig. Un sondage publi\'e sur Sina Weibo (traduction C. Custer, Tech in Asia, 1er Aout 2011), consult\'e le 24 F\'evrier
2014, \`a 22h12.}


\bigskip

{\color{black}
Sina Weibo d\'esactive alors la fonction de commentaires des messages. Dans les jours qui suivent, le gouvernement fait
enfin une d\'eclaration officielle sur les causes de l'accident de train puis se d\'ecide \`a agir en mettant en garde
les internautes trop audacieux de repr\'esailles \`a venir. Les messages controvers\'es sont supprim\'es, plusieurs
comptes utilisateurs sont ferm\'es, la police convoque les meneurs des discussions et d'autres mesures d'intimidation
sont men\'es aupr\`es des journalistes et des stars du weibo qui se seraient exprim\'es un peu trop directement \`a
l'\'egard du Parti. Dans le m\^eme temps, le gouvernement a du mal a se saisir de se nouvel outil. Alors que les
administrations locales, universit\'es et les m\'edias d'\'etat ont plut\^ot bien r\'eussi le virage de leur
strat\'egie de communication vers le microblog, les membres du gouvernement de P\'ekin ouvrent des comptes o\`u ils
sont parfois raill\'es, tourn\'es en ridicule et harass\'es de questions. En F\'evrier 2012, les quatre plus grosses
soci\'et\'es de microblog (dont Sina Weibo) annoncent que chaque utilisateurs est maintenant contraint de modifier son
profil pour mentionner son v\'eritable nom, pr\'enom ainsi que son num\'ero de carte d'identit\'e. Cette vell\'eit\'e
de v\'erification \'echoue et est abandonn\'ee quelques semaines plus tard face \`a la mobilisation des utilisateurs et
la difficult\'e de faire appliquer de telles mesures. Le gouvernement de P\'ekin \'edite pourtant une s\'erie de
r\`egles ``Several Regulations on Microblog Development and Administration Enacted by the Beijing Government'' dont la
plus notable sera la possibilit\'e de condamner tout ceux qui auront particip\'e \`a la diffusion d'information
consid\'er\'ees comme fausses, erron\'ee ou mensong\`eres. Face \`a la multiplication des actions gouvernementales et
\`a l'apparition d'autres plateformes, la croissance du nombre d'utilisateurs de Sina Weibo a d\'esormais stopp\'ee
pour aborder une phase de d\'eclin estim\'e \`a pr\`es de 10\% dans les deux premiers mois de
2014\footnote{\textsf{\ }\textsf{D'apr\`es le }\textcolor{black}{CNNIC cit\'e dans l'article
}\textsf{\ {}``}\textsf{\textit{China's Twitter is bleeding users'', }}\textsf{17 Janvier 2014,
}\url{http://blogs.marketwatch.com/thetell/2014/01/17/chinas-twitter-is-bleeding-users/,}\textsf{ consult\'e le 17
F\'evrier 2014 \`a 18:17}}. }


\bigskip

{\color{black}
A la lecture de l'histoire de Sina Weibo on constate l'ambivalence des actions officielles du gouvernement chinois dans
la r\'eussite \'economique des entreprises d'Internet. Si la firme SINA a b\'en\'efici\'e de prime abord d'un avantage
comp\'etitif notoire par l'\'elimination de la concurrence, elle a par la suite souffert des cons\'equences du
contr\^ole politique de l'Internet, avec notamment la perte de ses utilisateurs.}


\bigskip



\begin{center}
\includegraphics[width=11.2917in,height=2.3752in]{chapitre1-img004.png}
\end{center}
{\raggedleft\color{black}
Fig. Cours de l'action SINA au Nasdaq entre 2009 et F\'evrier 2014 - \newline
Source : Google Finance, consult\'e le 17 F\'evrier 2014 \`a 15:28 
\par}


\bigskip

\subsubsection[1.2.2. Sina Weibo, un usage plus ludique que Twitter ]{1.2.2. Sina Weibo, un usage plus ludique que
Twitter }
\hypertarget{RefHeading131699228146}{}{\color{black}
Dans son article nomm\'e \textit{A Tale of two microblogs}, Jon L. Sullivan (2013) raconte comment l'\'ev\`enement
historique de la fermeture de \textit{Twitter }en Chine a vu la communaut\'es des microbloggers chinois se scinder en
plusieurs groupes distincts : }

\liststyleWWviiiNumii
\begin{itemize}
\item {\color{black}
\textit{Twitter }rassemble une communaut\'e avide de libres discussions, souvent tr\`es politis\'ee voire radicalement
en opposition avec le gouvernement chinois.}
\item {\color{black}
\textit{Tencent Weibo} est utilis\'e par les utilisateurs de \textit{QQ}, typiquement des personnes aux revenus plus
faibles qui acc\`edent au web depuis leurs mobiles}
\item {\color{black}
\textit{Sina Weibo} est le favori des travailleurs urbains, souvent plus jeunes ou \'eduqu\'es, repr\'esentant davantage
la classe moyenne montante.}
\end{itemize}

\bigskip

{\color{black}
Dans la litt\'erature en sciences informatiques, plusieurs articles propose des comparaisons entre Twitter et Sina
Weibo. Une large analyse quantitative et comparative men\'ee avec des jeux de donn\'ees des deux services (Qi\&al.,
2012) nous apprend que le contenu de Sina Weibo est davantage corr\'el\'e avec des sentiments positifs (analys\'es
automatiquement). Les utilisateurs de Sina Weibo parlent davantage de lieux et de personnes alors que les utilisateurs
actifs sur Twitter s'int\'eressent plus aux organisations. \'Egalement, Sina Weibo conna\^it un pic d'activit\'e le
week-end alors que Twitter affiche g\'en\'eralement une baisse de r\'egime dans les fins de semaine. Ces diff\'erentes
indications exprime une tendance o\`u Sina Weibo serait davantage utilis\'e pour des activit\'es de loisir quand
Twitter se destinerai \`a un usage plus professionnel. Une \'etude s'int\'eressant aux tendances sur Sina Weibo
(Li\&al., 2013) indiquent que la majorit\'e des comptes les plus influents de Twitter ont \'et\'e v\'erifi\'es
contrairement \`a Weibo o\`u le taux est plus faible chez les grands utilisateurs. La v\'erification d'un compte se
fait par l'authentification aupr\`es du fournisseur de service afin d'attester la v\'eracit\'e de la personne utilisant
le compte. C'est un enjeu important pour les figures publiques (marques, stars, homme politiques, etc.) Cet indicateur
nous montre donc l'int\'er\^et professionnel fort qui entoure Twitter, moins pressant dans le cas de Sina Weibo o\`u
moins de personnes ont ressenties la n\'ecessit\'e de faire officialiser leurs comptes. Sur les deux services de
microblog, les utilisateurs inscrits poss\`edent un r\'eseau de relations identifiables par leur souscription aux fils
d'infos d'autres utilisateurs (follow). La relation peut \^etre inexistante (none), mutuelle (friend) ou
unidirectionnelle (follow : un utilisateur suit un autre mais n'est pas suivi par ce dernier). La comparaison
d'\'echantillons des graphes sociaux issus des deux services (Chen\&al, 2012) montrent comment les relations sur Sina
Weibo sont plus dissym\'etriques et moins r\'eciproques, refl\'etant une hi\'erarchie plus forte entre les utilisateurs
que Twitter. }


\bigskip

{\color{black}
Un autre facteur important de diff\'erentiation entre les deux services est la nature de la diffusion des contenus
post\'es sur Sina Weibo. Contrairement \`a Twitter o\`u le texte domine, la majorit\'e des posts de Weibo contiennent
des images ou des vid\'eos (Zhao, 2012). Les posts poss\'edant des contenus multim\'edia (images, vid\'eos...) sont
plus susceptibles d'\^etre diffus\'es largement et restent en moyenne actifs pour une dur\'ee plus longue (ibid.).
\'Egalement, les contenus sur Sina Weibo poss\`edent une proportion moins \'elev\'ee de retweets et de commentaires que
sur Twitter (Zhao, 2012 ; Gao, 2012). L'activit\'e de la population de Twitter est plus intense, moins tourn\'ee vers
la diffusion de masse et plus r\'eactive aux influx de nouveaux contenus. }


\bigskip

{\color{black}
Nous voyons donc que le paysage de Sina Weibo se constitue autour de stars et c\'el\'ebrit\'es qui concentrent
l'attention avec des contenus \`a la diffusion tr\`es large. Moins tourn\'es vers l'actualit\'e et la conversation que
son homologue Twitter, Sina Weibo agit comme v\'ehicule de contenus \`a grande audience, souvent publi\'es par des
personnalit\'es publiques c\'el\`ebres. Des \'etudes quantitatives montrent bien que les contenus les plus \'echang\'es
et discut\'es concernent les loisirs et divertissements, la mode, la sant\'e, etc. (Li\&al., 2013). Les messages \`a
caract\`ere humoristique (texte, images et vid\'eos) occupent \'egalement une place pr\'epond\'erante dans les
\'echanges des utilisateurs, contrairement \`a son homologue am\'ericain Twitter domin\'e plut\^ot par les sujets
d'actualit\'e (Yu\&al., 2011). Sina poursuit ainsi son r\^ole historique de leader dans le domaine de l'infotainment. }


\bigskip

{\color{black}
Pourtant la population de jeunes urbains qui soutient la croissance de son service de microblog refl\`ete aussi les
transformations en cours dans la soci\'et\'e chinoise. Les journalistes et sp\'ecialistes de l'information sont les
premiers \`a se saisir de ce nouveau m\'edia. Dans un discours \`a Stanford en 2013, le PDG de \textit{Sina }Charles
Chao explique : }


\bigskip

{\color{black}
\textit{{}``}\textit{Le plus grand changement qu}\textit{{}'}\textit{a amen\'e le microblog en Chine concerne
d}\textit{{}'}\textit{abord l}\textit{{}'}\textit{industrie des m\'edias elle-m\^eme. Aujourd}\textit{{}'}\textit{hui,
plus de 30\% des actualit\'es ont d}\textit{{}'}\textit{abord \'et\'e report\'ees sur Sina Weibo avant d'atteindre les
m\'edias \ \ traditionnels. Le r\^ole des m\'edias traditionnels a \'et\'e d\'eplac\'e vers un \ \ traitement des
informations en profondeur (in-depth reporting).''}\footnote{\ Charles Chao, PDG de Sina pendant la Stanford Graduate
School of Business China 2.0 tenue le 3 Octobre 2013. Disponible en vid\'eo
\url{http://www.youtube.com/watch?v=tlliivJKHk8}, consult\'ee le 19 F\'evrier 2014 \`a 11:23}}


\bigskip

{\color{black}
L'omnipr\'esence des supports mobiles (smartphones, tablettes)\footnote{\ Selon l'Universal Telecommunication Union,
\textit{{\guillemotleft}~la Chine d\'epasse 1 milliard d'abonnements mobile, avec 400 millions d'utilisateurs
d'Internet mobile d\'epasse ainsi les \'Etats-Unis comme leader du march\'e des smartphones~{\guillemotright}},
http://mobithinking.com/blog/china-top-mobile-market\textstyleInternetlink{ }consult\'e le 24 F\'evrier 2012.} permet
en effet des modes de traitement de l'information jusqu'ici inconnus qui bousculent les hi\'erarchies tr\`es
contr\^ol\'ees des salles de r\'edaction chinoises. Alors que la population urbaine cro\^it rapidement, le smartphone
est \textit{{}``}\textit{the first big urban purchase}\textit{{}''}\textit{ }(Wallis, 2013) pour ceux qui arrivent en
ville et repr\'esente un outil indispensable de participation \`a la soci\'et\'e. En 2008, la Chine \'etait le seul
pays en Asie o\`u les mois de 30 ans poss\'edait plus d'amis en ligne que hors ligne (Synovate, 2009). Ainsi, les
r\'eseaux sociaux jouent un r\^ole primordial dans la socialisation urbaine et viennent changer les modes d'expression.
Il est \`a noter que les sp\'ecificit\'es de l'\'ecriture chinoise rendent possible l'\'ecriture d'un court texte en
140 caract\`eres, alors qu'une telle longueur autorise seulement une courte phrase dans une \'ecriture utilisant un
alphabet latin. Alors que le lectorat chinois a perdu toute confiance dans la plupart des m\'edias traditionnels suite
\`a l'absence r\'ep\'et\'ee de courage et la d\'etention d'information cruciale dans les dossiers importants qui
animent le pays, le microblog s'installe comme une nouvelle source de confiance pour des millions de citoyens voulant
comprendre et prendre part aux changements cruciaux de la Chine moderne.}


\bigskip

{\color{black}
Un rapport de l'Institut de Journalisme Reuters \`a l'Universit\'e Oxford paru en 2013 montre comment les usages du
microblog a amen\'e des transformations dans le quotidien des journalistes chinois. Le journalisme d'investigation a
notamment connu un essor important avec le renouvellement des sources gr\^ace \`a une large diffuson en ligne des
sujets. Sina Weibo n'a pas am\'elior\'e n\'ecessairement la qualit\'e de leurs investigations, mais a par contre permis
une plus grande diss\'emination. La mobilisation des utilisateurs pour la protection des journalistes a \'egalement
jou\'e un r\^ole important ainsi que le renforcement de proc\'ed\'es de v\'erification existant depuis longtemps sur
les forums du web chinois. Tr\`es populaire dans les ann\'ees 2000, Le ``moteur de recherche de viande humaine''
(renrou sousuo) est une forme de traque d'individu en ligne r\'ealis\'e par un large nombres d'internautes a partir
d'un nom ou d'une photo. Il s'agit souvent de retrouver quelqu'un d\'esign\'e comme ``coupable'' (d'adult\`ere, de
corruption, etc) en r\'eunissant un maximum d'informations \`a travers la Toile afin d'identifier ou de localiser la
personne. Devant les d\'erapages rapides de ce type de proc\'ed\'es, les questions d'\'ethique du journalisme en ligne
sont au coeur des discussions pour les journalistes et les titres de presse. En effet, l'usage des m\'edias sociaux
permis \`a certains journalistes de faire pression sur les pouvoirs publics, amenant parfois \`a la censure de leurs
travaux, mais a \'egalement permis une tr\`es large auto-promotion pour de nombreux journalistes devenus des stars de
weibo.}


\bigskip

{\color{black}
Le contr\^ole des contenus sur Weibo est donc une r\'ealit\'e quotidienne et a \'et\'e depuis son lancement la source de
plusieurs \'etudes. Les formes les plus courantes sont : la suppression de posts, la suppression de comptes
utilisateurs et le blocage de mots-cl\'es. Le blocage de mots-cl\'es s'effectue dans le moteur de recherche interne du
site (``pas de r\'esultats'' quand vous chercher un mot bloqu\'e) et plus r\'ecemment par l'impossibilit\'e de poster
un message contenant des mots ou des adresses web bloqu\'es (Ng, 2013). La pratique de la suppression de comptes s'est
intensifi\'ee en 2013\footnote{\textit{{}``}\textit{Over 100,000 Sina Weibo Accounts Shut Down or Penalized for Govt
Rules Violations''} par Gabriela Vatu, 14 November 2013
\href{http://news.softpedia.com/news/Over-100-000-Sina-Weibo-Accounts-Shut-Down-or-Penalized-for-Govt-Rules-Violations-400289.shtml,}{\textstyleInternetlink{http://news.softpedia.com/news/Over-100-000-Sina-Weibo-Accounts-Shut-Down-or-Penalized-for-Govt-Rules-Violations-400289.shtml}}
consult\'e le 17 F\'evrier \`a 16:42} avec notamment la suppression de millions de ``zombies'' pr\'esents sur le site.
Les {}``zombies'' sont des comptes utilisateurs cr\'e\'es par des robots qui se chargent de reposter automatiquement
des contenus, souvent afin d'augmenter le trafic sur le site. Les exigences des annonceurs publicitaires de plus en
plus pr\'esents sur le site ont oblig\'es Sina Weibo \`a faire la chasse aux robots sur son site, faisant ainsi
diminuer le nombre de comptes actifs de mani\`ere significative. La firme Sina est garante aupr\`es du Minist\`ere de
la S\'ecurit\'e Publique chinois des contenus qu'elle diffuse et effectue \`a ce titre une surveillance constante pour
supprimer les messages ``non-conformes''. Lors de nos recherches, nous avons constat\'e que l'interface de Sina Weibo
garde la trace des commentaires supprim\'es par le syst\`eme d'administration. Dans les messages supprim\'es se
trouvent \`a la fois des posts d'utilisateurs ``zombies'' et les posts jug\'es incorrects par les administrateurs. Au
total, la suppression des messages s'effectue avec un taux estim\'e \`a environ 16\%, allant jusqu'\`a plus de 50\%
dans certaines provinces comme Ningxia ou le Tibet contre seulement 12\% \`a Beijing (Bamman, 2012).}


\bigskip



\begin{center}
\includegraphics[width=6.1874in,height=3.6874in]{chapitre1-img005.png}
\end{center}
{\color{black}
Figure 2 - La trace des commentaires supprim\'es par Sina est encore visible - \textit{Page 677 \`a 708, les
commentaires ont \'et\'e supprim\'ees, soit }approximativement 4\% messages supprim\'es (589 messages sur 13452, \`a
raison de 18 \`a 20 messages par page). \textit{Capture d'\'ecran effectu\'ee le 29 Janvier 2013 \`a 12:32:42,
}http://www.weibo.com/1701401324/zeoBquVKi,\textcolor[rgb]{0.0,0.5019608,0.5019608}{ }consult\'e le 16/02/2013.}


\bigskip

\subsection[1.3. Code, langage et milieu(x) num\'erique(s)]{1.3. Code, langage et milieu(x) num\'erique(s)}
\hypertarget{RefHeading151699228146}{}{\color{black}
L'espace d'expression offert par l'Internet chinois et ses services de r\'eseaux sociaux poss\`ede donc de multiples
particularit\'es et am\`ene une vari\'et\'e de pratiques, de questions et de r\'eflexions qui n\'ecessitent d'\^etre
\textsf{appr\'ehen}d\'e\textsf{ }avec \textsf{des }outils de probl\'ematisation et \textsf{d'analyse sp\'ecifiques.}
Dans la troisi\`eme et derni\`ere partie de ce chapitre introductif, nous allons donc nous pencher sur les concepts
existants dans la litt\'erature scientifique qui peuvent nous \'eclairer sur l'existence \textit{in situ} des
\textsf{objets num\'eriques}. Afin de mettre en perspective le Web chinois et l'histoire de ces objets, nous
introduirons notamment le concept de \textsf{\textit{milieu num\'erique}}\textit{, }h\'eritier de l'id\'ee complexe de
milieu que nous explorerons ci-apr\`es.}

\subsubsection[1.3.1. Lieu, espace, territoire et technologies]{1.3.1. Lieu, espace, territoire et technologies}
\hypertarget{RefHeading171699228146}{}{\color{black}
Sciences g\'eographiques, management et diffusion de l'innovation, histoire des technologies, \textit{cultural studies
}ou \'etudes ``nationales'', les \'etudes qui s'int\'eressent au relation entre technologies, espace, lieux et
territoires sont nombreuses et offrent un paysage riche o\`u se croisent de nombreuses disciplines scientifiques. Dans
cette \'etude, nous avons choisi d'interroger les objets num\'eriques afin de comprendre comment se structurent la
parole et la conversation dans le contexte unique de de l'Internet chinois. Afin d'articuler les multiples dimensions
d'analyse qui viennent informer notre r\'eflexion, il nous faut donc brosser un portrait en large de l'Internet
chinois, en le consid\'erant tour \`a tour comme un espace \textit{structurant} pour les actions des internautes qui le
pratiquent, comme un territoire \textit{sujet }aux relations de pouvoir actualis\'ees par les groupes et individus et
enfin comme un lieu \textit{habit\'e }par ceux qui y construisent chaque jour des significations communes. Nous
proposons donc ici une revue s\'elective des quelques travaux \`a m\^eme de nous apport\'e des \'eclairages pertinents
dans la vaste litt\'erature s'int\'eressant aux dimensions g\'eo-graphiques des TIC. }


\bigskip


\bigskip

{\sffamily\color{black}
Consid\'erant notamment l'influence des r\'eseaux de transports sur le d\'eveloppement des villes, certains g\'eographes
d\'efendent le r\^ole d\'ecisifs des humains sur l'espace bien avant la technologie Error: Reference source not found.
L'Internet lui-m\^eme a \'egalement fait l'objet de nombreuses \'etudes monographiques (par pays) ou comparatives, une
\'etude \`a l'\'echelle mondiale pr\'esentant en effet des probl\`emes de donn\'ees et \'evidemment d'\'echelle Error:
Reference source not found.La disponibilit\'e des donn\'ees et le r\^ole croissant de la cartographie dans l'usage
d'Internet viennent produire un {\guillemotleft}~\textit{Geoweb}~{\guillemotright} constitu\'e de donn\'ees et
m\'etadonn\'ees spatiales Error: Reference source not found. D\`es les d\'ebuts d'Internet, la carte a jou\'e sur le
web un r\^ole important d'abord en tant qu'illustration, puis plus r\'ecemment d'interface. Le d\'eveloppement de
standards comme le syst\`eme GPS et de services en ligne Error: Reference source not found, on vu appraitre de
nouvelles formes de donn\'ees produites par les utilisateurs, dites \textit{{\guillemotleft}~volunteered geographic
information~{\guillemotright}} Error: Reference source not found. }


\bigskip

{\sffamily\color{black}
Cette participation croissante des internautes dans la production de donn\'ees ouvre de nouvelles voies \`a la
d\'ecouverte des espaces en tant que construction sociale, proposant une voies d'acc\`es aux pratiques des lieux
individuelles comme de groupes. La pratique de la cartographie notamment se voit renouvell\'ee par les nombreux outils
disponibles en ligne et la possibilit\'e d'utiliser de nombreuses donn\'ees disponibles poss\'edant des marqueurs
spatiaux ou \textit{geotag }\textit{Error: Reference source not found}. }


\bigskip

{\sffamily\color{black}
Pourtant comme le d\'efendent les chercheurs du projet \textit{Floating Sheep}\footnote{\ Ce projet regroupe 5
chercheurs autour de la g\'eographie des r\'eseaux sociaux Zook, Graham, Shelton, Stephens, Poorthuis
\url{http://www.floatingsheep.org/}}\textit{ }\textit{Error: Reference source not found}, la focalisation sur le
\textit{geotag} comme principal approche m\'ethodologique Error: Reference source not found r\'eduit consid\'erablement
le champ de recherche des g\'eographes lors de l'\'etude des donn\'ees de r\'eseaux sociaux en ligne. Dans ce papier,
les chercheurs pr\'esentent un programme en 5 \'etapes pour penser \textit{{\textquotedbl}beyond the
geo~web{\textquotedbl}} :}

\liststyleWWviiiNumi
\begin{enumerate}
\item {\sffamily\color{black}
D\'epasser la simple projection des geo-tag sur une carte pour produire des visualisations \`a diff\'erentes dimensions,
notamment en incluant les diff\'erents \'el\'ements g\'eographiques pr\'esents dans chaque tweet (lieux cit\'es,
etc.).}
\item {\sffamily\color{black}
Prendre en consid\'erations les dynamiques temporelles (espace-temps de production et r\'eception)}
\item {\sffamily\color{black}
Inclure les dimensions sociales et relationnelles lors de l'analyse}
\item {\sffamily\color{black}
Prendre en compte non seulement les humains~mais aussi les automates qui cr\'eent \'egalement du contenu}
\end{enumerate}
{\sffamily\color{black}
Mettre en relation les donn\'ees des OSNS avec d'autres sources (articles de presse, recensement~, open data, etc.)}


\bigskip

{\color{black}
Ici, les lieux physiques offrent le point d'entr\'ee commun aux r\'eseaux et acteurs multiples~Error: Reference source
not found. En termes de m\'ethodologie, les acteurs des r\'eseaux num\'eriques~(humains et machines)~produisent
aujourd'hui quantit\'e de donn\'ees qui si elles semblent disparates, peuvent souvent \^etre r\'eunies et comprises par
leur attachement commun \`a un lieu : Open Data territorial, geo-localisation, GIS,
POI\footnote{\foreignlanguage{english}{\ }\foreignlanguage{english}{GIS : }\foreignlanguage{english}{\textit{Geographic
Information System}}\foreignlanguage{english}{ ; POI : }\foreignlanguage{english}{\textit{Point-Of-Interest}}}.
\ Ainsi, l'approche de la question de la relation entre lieux et TIC offre l'opportunit\'e d'un regard crois\'e,
prenant la g\'eographie comme approche pour comprendre ces donn\'ees}


\bigskip

{\sffamily\color{black}
En Chine o\`u l'urbanisation produit actuellement une vaste mobilit\'e notamment des ruraux qui arrivent en ville,
l'utilisation des r\'eseaux sociaux m\'ediatise bien souvent les choix de lieux et de strat\'egies pour l'achat
d'immobilier (Li, 2013). Au travers de groupes de discussions, les nouveaux acheteurs se regroupent pour former des
associations de d\'efenses de droits pour leurs biens, mais aussi organiser des rencontres. L'usage des r\'eseaux
sociaux rev\^et ici une importance capitale, notamment dans la recherche de groupes similaires et l'\'echange
d'exp\'eriences.}


\bigskip

{\bfseries\color{black}
Cartographie : espace, temps, r\'eseaux}

{\color{black}
\textsf{l'interm\'ediaire de la carte n'est pas chose nouvelle }\textsf{Error: Reference source not found}\textsf{ mais
l'usage nouveau d}\textsf{u GPS }\textsf{mobile notamment en fait d\'esormais un outil d'exploration en soi
}\textsf{Error: Reference source not found}\textsf{. L'arriv\'ee de l'Internet et des outils de cartographie libre
(open-source) ont \'egalement largement contribu\'e a une appropriation de cette pratique par un nombre croissant de
personnes, }\textsf{Error: Reference source not found}}


\bigskip

{\color{black}
\textsf{Loin d'\^etre fig\'e par le territoire, la carte le d\'ecrit sous un ou des angles particuliers }\textsf{Error:
Reference source not found}\textsf{. La pratique du }\textsf{\textit{mapping}}\textsf{ permet d'identifier des
dynamiques nouvelles entourant des ph\'enom\`enes particuliers. La pratique de la cartographie contributive induite par
l'apparition de services en ligne et l'usage de marqueurs sur Google Maps notamment offre d\'esormais un
}\textsf{\textit{{\guillemotleft}~miroir du monde offline~{\guillemotright}}}\textsf{ }\textsf{Error: Reference source
not found}\textsf{. Le courant dit de la n\'eog\'eographie utilise abondamment le GIS et les outils en ligne (Google
Mpas, Flickr, etc.) afin de pratiquer de nouvelles formes de {\guillemotleft}~g\'eographies
volontaires~{\guillemotright} }\textsf{Error: Reference source not found}\textsf{. Au travers de cette pr\'esence
accrue dans le r\'eseau se dessine l'enjeu non seulement de cartographier le monde, mais \'egalement de cartographier
le r\'eseau lui-m\^eme. }}


\bigskip


\bigskip


\bigskip


\bigskip

\paragraph[1.3.1.1. Code / space : l{}'espace transductif des TIC]{1.3.1.1. Code / space : l'espace transductif des TIC}
\hypertarget{RefHeading191699228146}{}{\color{black}
Dans leurs recherches autour de la g\'eographie des technologies num\'eriques, Dodge \& Kitchin ont travaill\'e \`a
d\'evelopper le concept de code comme un \'el\'ement fondateur des espaces modernes dans lesquels nous \'evoluons.
Refl\'etant l'importance croissante accord\'e aux TIC dans l'environnement urbain, ils citent un travail sur la
production automatique des espaces : ``De plus en plus, les espaces de la vie \ quotidienne nous parviennent charg\'es
de logiciels (software)'' Error: Reference source not found En effet, si le projet urbain a \'et\'e guid\'e pendant le
demi-si\`ecle dernier par l'\'eruption de la technologie automobile dans l'espace, les TIC ont d\'esormais pris le
relais avec l'id\'ee d'une ville intelligente et connect\'ee, connues sous le nom par trop pompeux de
{\guillemotleft}~smart cities~{\guillemotright} Error: Reference source not found. Dodge \& Kitchin ont donc fait du
\textit{code} une des pierres d'angles de l'appr\'ehension de l'espace dans leur travail en le d\'efinissant comme suit
: }


\bigskip

{\color{black}
{\guillemotleft}~an instruction or rule that has a single outcome determined by a binary logic \ \ (yes/no). The
combination of these indidivuals logic rules produces code \ \ \ \ (program)'' Error: Reference source not found.}


\bigskip

{\color{black}
La part croissante des TIC dans nos espaces quotidiens am\`enent les auteurs \`a envisager l'espace dans son interaction
avec le code, symbolisant la suite d'instructions machiniques et \'electroniques qui permettent \`a un espace de
remplir sa fonction. Dans un article intitul\'e Flying through code/space: the real virtuality of air travel, Dodge \&
Kitchin analyse la structure des espaces a\'eroportuaires. De l'achat des tickets jusqu'au vol des avions en passant
par la gestion des bagages, le bon fonctionnement d'un a\'eroport est enti\`erement conditionn\'e par le bon
fonctionnement des longues successions d'instructions du code. Ici, Dodge et Kitchin propose le concept de
code/space\textit{ }pour d\'ecrire ce type d'espace sp\'ecifique o\`u lorsque le code \'echoue (failure) alors le
code/space tout entier \'echoue Error: Reference source not found. L'exemple de l'a\'eroport est parlant : si le
syst\`eme de check-in des bagages ou les machines responsables du contr\^ole de s\'ecurit\'e des passagers ne
fonctionnent pas, alors l'espace a\'eroportuaire ne peut exister en tant qu'a\'eroport. L'analyse de la spatialit\'e ne
se situe alors plus dans un domaine s\'emantique ou narratif, mais plut\^ot dans les processus et op\'erations qui s'y
d\'eroulent et le code et les technologies y jouent un r\^ole primordial : }


\bigskip

{\color{black}
{\guillemotleft}~Code is employed as the solution to a problem, a particular kind of transduction \ \ is
occurring.{\guillemotright} Error: Reference source not found. }


\bigskip

{\color{black}
L'espace n'est pas un donn\'e mais s'explique plut\^ot comme : ``une forme d\textit{{}'}ontog\'en\`ese (en perp\'etuel
devenir-au-monde), l\textit{{}'e}space est une pratique; un faire ; un \'ev\`enement ({\dots}) qui ne pr\'e-existe pas
\`a son faire (doing)''. L'espace est consid\'er\'e non pas comme une production, mais comme une transduction.
Reprenant le travail de Simondon sur l'individuation par la technologie, Dodge \& Kitchin pr\'esente l'espace comme une
pratique qui comprend les actes, actions, occurrences, m\'emoires, perceptions, etc. d'un groupe d'individus s'y
trouvant. La fonction de l'espace structure et est structur\'ee par les individus et le code y est consid\'er\'e comme
une entit\'e agissante. Dans le code/space, la relation dyadique entre code et espace est sym\'etrique : l'un ne peut
aller sans l'autre. En terme simondonnien, la transduction ne peut \^etre assur\'ee sans code. Si l'exemple de
l'a\'eroport illustre bien cette n\'ecessit\'e du code dans le devenir-espace, Dodge \& Kitchin ont \'egalement
identifi\'e d'autres cat\'egories o\`u cette relation est plus t\'enue : les coded spaces, qui peuvent poursuivre leurs
fonctions m\^eme lorsque le code \'echoue ; les background coded spaces o\`u les processus de transduction induit par
l'espace ne s'appuie pas n\'ecessairement sur le code, mais propose n\'eanmoins des possibilit\'es de l'activer
(machine \'eteintes ou inactives, etc.) }


\bigskip

{\color{black}
L'analyse fonctionnelle des rapports entre espace et technologie de Dodge \& Kitchin offre \`a voir comment les TIC
peuvent \^etre un facteur transductif pour les individus se mouvant dans les espaces de leurs vies quotidiennes. Si
nous appuyons pleinement ce constat, il nous semble que le parti-pris des auteurs de consid\'erer le ``code'' comme une
abstraction incluant uniquement les instructions ou ``logiques machiniques`` ferme la porte \`a l'immense densit\'e des
activit\'es symboliques qui se jouent dans l'usage des technologies. Comment notamment consid\'erer les ``contenus'' du
web dans cette grille de lecture? Comment resituer dans une perspective historique les logiques de m\'ediation de
l'espace par les technologies de l'\'ecriture? Ils nous semble en effet que la faillite fonctionnelle
(\textit{failure}) des code/spaces pr\'ec\`edent l'arriv\'ee des technologies et s'op\`erent d\'ej\`a \`a un niveau
symbolique - la fonction de l'espace du Palais du Louvre apr\`es la chute des rois de France se voit radicalement
modifi\'ee. La \textit{transduction} op\'er\'e lors de la pratique d'un espace s'effectue donc dans un jeu
d'appropriation symbolique qui passe notamment mais pas seulement par les technologies. Ici les technologies du langage
et de l'information jouent notamment un r\^ole crucial dans l'affirmation du r\'ecit symbolique (\textit{narrative})
qui construit l'espace. L'activit\'e du code dans la structuration des code/space de Dodge \& Kitchin existe donc sous
une forme non seulement fonctionnelle mais \'egalement s\'emantique, voire phatique ou m\^eme esth\'etique comme l'a
d\'ecrit Jakobson dans ces analyses des fonctions du langage Error: Reference source not found. Au-del\`a de sa
dimension machinique, le code poss\`ede les caract\'eristiques d'une \textit{poiesis }d\'epassant l'id\'ee simple de
fonctionnalit\'e pour exister dans la complexit\'e d'une \'ecriture comme traduction du langage humain et machine.}

\paragraph[1.3.1.2. Codes, discours et territoires des technologies]{1.3.1.2. Codes, discours et territoires des
technologies}
\hypertarget{RefHeading211699228146}{}{\color{black}
Le code serai davantage a comprendre comme un mode d'expression humain \`a travers la technologie, actualisant
l'episteme d\'ecrit par Foucault dans Les Mots et Les Choses comme l'\'el\'ement fondamental de la pens\'ee d'une
\'epoque et sa consid\'eration pour le monde Error: Reference source not found. D\'efinie comme l'\'etat des
connaissances scientifiques et litt\'eraires, l{}'espiteme existe comme la somme des savoirs d'une \'epoque
pr\'esuppos\'ee traduite en un regard sur le monde. Le code exprim\'e dans de nombreux langages est une \'ecriture o\`u
s{}'expriment les savoirs d'aujourd'hui. Le code source d'une page Internet, d'un programme informatique ou d'un driver
hardware ne s'\'ecrit pas seulement en langage ``machine'' mais m\'elange langages informatiques et humains. A la fois
production savante, outil scientifique, vecteur d'expression et interfaces des savoirs, le code constitue
l'exp\'erience narrative du monde par les TIC. Poss\'edant de nombreux mots, aspects et syntaxes issus de multiples
langues humaines, les replis de l'\'ecriture informatique laisse transparaitre de tous bords leur origine humaine. Les
standards d'encodage des bases de donn\'ees comme l'Unicode deviennent presque le nouvel alphabet de notre \'ecriture
(Guichard, 2014). Une minorit\'e de personnes savantes, ``lettr\'es'' de l'informatique d\'echiffrent et \'ecrivent le
\textit{code }alors que l'immense majorit\'e se voit frapp\'ee d'illettrisme devant l'arr\^et soudain de l'ordinateur
ou la perte d'un fichier. Ainsi, la d\'efinition du \textit{{}``code'' }de Dodge \& Kitchin doit \^etre \'etendue pour
recouvrir plus largement les pratiques symboliques li\'es aux activit\'es de l'\'ecriture du code dans ces espaces.}


\bigskip

{\color{black}
Le \textit{code} ainsi red\'efinit nous ram\`ene alors \`a une lecture foucaldienne du \textit{discours }dans sa
relation intime avec le territoire (Foucault, 2004). Dans ces nombreux travaux sur la g\'en\'ealogie, Michel Foucault
cherche \`a comprendre comment les relations de pouvoir cr\'e\'ees au travers des discours port\'es sur les objets
pr\'eside \`a la production de territoires, d'interdits comme autant de sujets de ces discours. Nous d\'efinissons la
\textit{discursivit\'e }comme le processus de construction de ces discours, et ce faisant le processus de d\'efinition
des territoires de l'Internet au travers de la pratique du code. \textstyleappleconvertedspace{Historiquement,
l'Internet a \'et\'e tr\`es t\^ot sujet \`a l'appropriation par le discours de nombreux groupes actifs dans leur
volont\'e de territorialisation. }La m\'etaphore g\'eographique et spatiale a structur\'ee le vocabulaire de l'Internet
d\`es sa cr\'eation Error: Reference source not found : site, cyberspace, etc. L'Electronic Frontier Foundation se
charge de prot\'eger l'u-topie qu'est Internet avec JP Barlow qui r\'edige la fameuse \textit{D\'eclaration
d'Ind\'ependance du Cyberespace }\textit{Error: Reference source not found}\textit{. }A l'oppos\'e du spectre, les
\textit{autoroutes} de l'information in-forment le paysage comme autant de g\'eogrammes massifs Error: Reference source
not found. L'appropriation des protocoles du r\'eseau, la lutte pour la d\'efense de standards ouverts\textit{ }sont
autant d'expression qui s'ancrent dans les pratiques du discours, dont les mots \textit{free }et \textit{open
}cristallisent l'histoire de luttes de revendication des territoires de l'Internet Error: Reference source not
found\textit{. }L'autre grande m\'etaphore constitutive de l'Internet est textuelle avec ses pages, langages et
hyper\textit{textes }\textit{Error: Reference source not found}. La formule choc \textit{{\guillemotleft}~Code is
law~{\guillemotright} }\textit{Error: Reference source not found} r\'esume l'id\'ee \ que les processus textuels du
\textit{code }mettent en jeu un ensemble de r\^oles, protocoles et mises en sc\`ene qui agissent comme autant
d'autorit\'es \`a travers le discours. La jurisprudence fait~loi comme \'ecrit sur les murs des bureaux de Facebook \`a
Palo-Alto : \textit{{\guillemotleft}~Code wins arguments~{\guillemotright}}\footnote{\ Dans la Lettres aux
Investisseurs \'ecrite par M. Zuckerberg \ pour l'IPO de Facebook
\url{http://www.sec.gov/Archives/edgar/data/1326801/000119312512034517/d287954ds1.htm\#toc287954_10}}. La
territorialisation de l'Internet se fait donc au travers d'un ensemble de pratiques discursives, m\'eta-grammaire des
discours en ligne o\`u les luttes symboliques se jouent dans les pratiques discursives du code. La confrontation
symbolique au sein des territoires num\'eriques se poursuit dans les pratiques du discours, r\'eifier dans le code. Les
structures fonctionnelles et symboliques deviennent les strat\'egies de production de territoires qui se d\'efinissent
dans l'usage des langages. }


\bigskip

{\color{black}
Sur l'Internet chinois, les pratiques de censure de l'\'ecriture en sont le reflet le plus frappants. Blocage de
mots-cl\'es, d\'etournements de langage, suppression et modification de texte sont l'expression de cet affrontement de
discursivit\'es parfois antagonistes. La Grande Muraille gouvernementale scanne les masses de texte pour reconna\^itre
et stopper des mots tels que {\guillemotleft}~\textit{Printemps arabe~}{\guillemotright} ou
{\guillemotleft}~\textit{\'ev\`enements de Tian-An Men~}{\guillemotright} (McKinnon, 2009). N\'eanmoins, l'\'etat
actuel des techniques de \textit{data mining} ne permet pas encore de d\'eceler les ph\'enom\`enes langagiers comme les
jeux de mots ou l'ironie. Bien souvent, les internautes chinois choisissent l'humour pour permettre \`a leurs id\'ees
de se frayer un espace. Rev\^etant leurs masques de chat, les internautes chinois sont devenus sp\'ecialistes dans la
publication de jeux de mots, chansonnettes et petites vid\'eos d'animaux, comme autant de couperets cinglants pour
railler les officiels trop pompeux de P\'ekin. Dans la guerre de l'information que se livrent sans cesse censeurs et
internautes, les v\'eritables h\'eros sont bien souvent de simples photos truqu\'ees de crabes et de lamas. Ces blagues
num\'eriques, d'apparence bien souvent inoffensive, font chaque jour le tour de la Toile chinoise, portant en elles
toute la subversion d'internautes aspirant \`a plus de libert\'e. D\'ebut 2010, alors que pleuvaient les longs discours
pieux du Parti sur l'harmonie de la nouvelle soci\'et\'e (en chinois \textit{hexie}), on voit appara\^itre en ligne des
essaims de crabes de rivi\`ere (se pronon\c{c}ant \'egalement \textit{hexie}) couverts de cha\^ines en or criant :
\textit{{}``Vive l'harmonie'' }au volant de leur limousine. Devenus aujourd'hui une image vivante de la corruption des
hauts dignitaires du Parti, on croise r\'eguli\`erement dans les commentaires d'un article officiel un petit crabe de
rivi\`ere, comme un petit rappel post\'e par un lecteur.}


\bigskip

\begin{flushleft}
\tablefirsthead{}
\tablehead{}
\tabletail{}
\tablelasttail{}
\begin{supertabular}{|m{3.4629598in}|m{2.87956in}|}
\hline
\begin{center}
\includegraphics[width=5.2083in,height=3.9063in]{chapitre1-img006.jpg}
\end{center}
 &
~

~

{\color{black} \textbf{[6CB3?][87F9?]}\textbf{ }\textbf{(hexie)}}

{\bfseries\color{black} Crabe de riv\`ere}

{\color{black} M\`emes / humour des internautes}

~

{\color{black} \textbf{[548C?]}\textbf{[8C10?]}\textbf{ }\textbf{(hexie)}}

{\bfseries\color{black} Harmonie}

{\color{black} Discours politique sur l'harmonie}\\\hline
\end{supertabular}
\end{flushleft}
{\color{black}
Fig. -- \textit{Mot} \textit{de la semaine : Crabe de Rivi\`ere} - China Digital Times du 21 Mars 2012 - Licence
Creative Commons http://chinadigitaltimes.net/2012/03/word-of-the-week-river-crab/, consult\'e le 15 F\'evrier 2013.}


\bigskip

{\color{black}
On voit bien comment le code d\'ecrit ici un territoire sujet \`a l'autorit\'e politique sous la forme d'un syst\`eme de
\textit{data mining} cherchant \`a mettre en forme le discours. La fonction du code/space s\'emantique que forme ici
l{}'espace de l'Internet est r\'e\'ecrit par un jeu de langage et la circulation d'objets digitaux permet de
reterritorialiser cet espace en apparence r\'egit par un code strict de censure. }

\paragraph[1.3.1.3. Les lieux des technologies comme ]{1.3.1.3. Les lieux des technologies comme }
\hypertarget{RefHeading231699228146}{}{\color{black}
Dans son c\'el\`ebre livre sur les \textit{Arts de Faire, }De Certeau Error: Reference source not found consid\`ere la
ville comme un texte dont chaque pi\'eton \'enonce et r\'ev\`ele \textit{(performe)} des sens nouveaux par son
activit\'e de marcheur. Actualisant l'espace urbain par sa marche, l'habitant de la ville s'approprie des lieux qui
restent n\'eanmoins partag\'es avec d'autres. Au d\'etour des rues, le sens commun des lieux urbains se construit au
travers des multiples \'enonciations de ceux qui les habitent et les font vivre. Cette magnifique image de la po\'esie
du texte urbain met en lumi\`ere la dualit\'e que nous avons abord\'e pr\'ec\'edemment avec la \textit{transduction }de
Simondon :\textit{ }l'espace ne peut fonctionner sans les pratiques de ceux qui l'habitent. Plus encore,
l'\^etre-ensemble et le devenir-soi proc\`edent de la construction de lieux communs, \textit{poiesis }des espaces
habit\'es. Les TIC font aujourd'hui souvent partie int\'egrante des lieux que nous habitons. Graham (1998) dans son
travail sur l'\'etude des lieux et de leur rapport \`a la technologie %
%
%Autre point, je suis circonspect avec \ ton d\'ecoupage de la page 26. Ok pour le premier point. mais apr\`es je en voie pas la diff\'erence que tu fais entre les deux autres. Peix ut pr\'eciser l{}'origine de {\textquotedbl} co{}-\'evolutionnsite{\textquotedbl}? Pour moi la finalit\'e des gens que tu cites dans la 2 est bien la m\^eme que ceux que tu cites dans la 3. En revanche l{}' approche d\'eterministe est une d\'eviation efectivement bien pr\'esente chez les am\'enageurs( voir les smart cities maintenant). Mais alors cite les auteurs qui s.y rattachent. Pour la 3, ne limite pas \c{c}a \`a la sociologie des r\'eseaux.
%clemsos
%April 22, 2014 10:28 AM
identifie trois types majeurs d'approches dans la litt\'erature :}


\bigskip

\liststyleWWviiiNumiii
\begin{enumerate}
\item {\color{black}
L'approche \textit{{\guillemotleft}~substitutive~{\guillemotright}} ou \textit{{\guillemotleft}~transductive
{\guillemotright}} qui voit dans l'arriv\'ee des TIC la disparition de la valeur des lieux, dans un id\'eal de
proximit\'e utopique Error: Reference source not found\textit{ }ou un discours dystopique sur leur proche disparition
Error: Reference source not found. Ces consid\'erations sont les formes traditionnelles du d\'ebat accompagnant
l'innovation dont sont f\'erus la communication industrielle et la critique des m\'edias Error: Reference source not
found\textit{, }restant souvent purement prospectif et faisant peu de cas des usages.}
\item {\color{black}
Plus mod\'er\'ee, l'approche qualifi\'ee de \textit{{\guillemotleft}~co-\'evolutionniste~{\guillemotright} }s'origine
dans les \'etudes \'economiques et sociales et s'interroge sur la fa\c{c}on dont les technologies produiraient de
nouveaux lieux Error: Reference source not found. Prenant la forme d'une m\'ediologie de l'espace, cette approche
d\'eterministe est adapt\'ee aux \'etudes strat\'egiques pour l'urbanisme et l'implantation des t\'el\'ecommunications
mais reste peu utile \`a la compr\'ehension des ph\'enom\`enes li\'es \`a l'appropriation et l'historicit\'e des
technologies Error: Reference source not found.}
\item {\color{black}
Une derni\`ere approche plus r\'ecente se cristallise autour de l'id\'ee de relations et de r\'eseaux. Dans la
perspective d'objets mat\'eriels {\guillemotleft}~actants~{\guillemotright} Error: Reference source not found, il
s'agit de consid\'erer les lieux en termes relationnels, comme \textit{{\guillemotleft}~des moments articul\'es dans un
r\'eseau de sens et de relation sociales~{\guillemotright}} Error: Reference source not found\textit{, }produits par la
g\'eom\'etrie des individus, groupes sociaux et flux qui les entourent\textit{.} En cherchant \`a comprendre les lieux
comme des \textit{{\guillemotleft}~assemblages~{\guillemotright}} entre r\'eseaux
d'\textit{{\guillemotleft}acteurs{\guillemotright}} et de sens, on cherche \`a \'eviter l'\'ecueil des causalit\'es
directes en bannissant notamment la notion fataliste d'{\guillemotleft}~\textit{impact}~{\guillemotright}.}
\end{enumerate}

\bigskip

{\color{black}
En privil\'egiant une approche dynamique et relationnelle des lieux comme constructions sociales de sens Error:
Reference source not found, la technologie perd son r\^ole d\'eterministe de productrice d'espaces et d'usages pour
devenir une actualisation d'un espace-temps g\'eographique et historique par des groupes d'individus. Les
ph\'enom\`enes de transduction \`a l'{\oe}uvre dans les pratiques de l'Internet sont les \'etats d'un ensemble
(r\'eseau) \`a un moment donn\'e. Dans le cas de la censure \'evoqu\'e pr\'ec\'edemment, nous pouvons voir appara\^itre
clairement l'ordre du discours. Or il peut \^etre important de consid\'erer davantage les manifestations des
ph\'enom\`enes de transduction, notamment au travers des multiples actes d'\'enonciation qui font les pratiques et
usages du web. Au-del\`a des questions de territorialisation par le discours que semble refl\'eter les tensions de
contr\^ole sur l'Internet chinois, nous cherchons dans cette \'etude \`a comprendre de mani\`ere plus profonde comment
les internautes habitent leur Internet. Il ne s'agit pas d'analyser le discours en terme de relations de pouvoir mais
plut\^ot d'essayer de distinguer comment les circulations des objets num\'eriques sont structurantes pour les pratiques
des internautes, comme autant de lieux habit\'es quotidiennement.}

\subsubsection[1.3.2. Le milieu~num\'erique]{1.3.2. Le milieu~num\'erique}
\hypertarget{RefHeading251699228146}{}
\bigskip

{\color{black}
Afin de probl\'ematiser l'\'etude de la formation des pratiques du quotidien sur l'Internet, nous avons choisir
d'introduire le concept de \textit{milieu num\'erique}. Nous d\'efinirons d'abord bri\`evement ce concept avant de
pr\'esenter un regard historique sur l'id\'ee de milieu dans les sciences. Nous discuterons ensuite de l'acception
particuli\`ere que nous avons choisit de d\'efendre ici et nous verrons comment cette notion sera utile pour la suite
de notre \'etude sur les r\'eseaux sociaux en Chine.}


\bigskip

{\color{black}
L'id\'ee de milieu num\'erique \textstyleappleconvertedspace{est a priori d\'efinit par le philosophe Yuk Hui dans les
termes suivants : }}

{\color{black}
{}``The multiple networks, which are connected by protocols and standards, constitute what I call a digital milieu.''
(Hui, 2012)}


\bigskip

{\color{black}
\textstyleappleconvertedspace{L'usage de multiples interfaces et le d\'edale des r\'eseaux
}\textstyleappleconvertedspace{TIC }\textstyleappleconvertedspace{constitue ``un nouveau milieu perceptif'' (Weissberg
\& Barboza, 2007). }Notre milieu physique est aujourd'hui (d\'ec)ouvert par l'existence de notre milieu digital qui
nous aiguille: rencontres, restaurants, voyages, etc. sont bien souvent m\'ediatis\'es par l'Internet en premier lieu.
Ainsi, nous \'evoluons dans un milieu num\'erique qui agit comme support des processus de transduction et de
connaissance du monde. Le \textit{code }prend ici pleinement part \`a la construction de ce milieu digital, \`a la fois
d\'eterminant pour la production des actes de discours et ouvert \`a l'appropriation des pratiques et usages du
quotidien.}


\bigskip

{\color{black}
Historiquement, le concept de milieu se d\'etache du centre pour resituer et mettre en perspective la relation des
\^etres \`a leurs environnements sous des jours parfois contradictoires. D\'ebattue puis \'ecart\'ee mais toujours
tr\`es usit\'ee, la notion de \textit{milieu} introduit autant le d\'eterminisme d'un combat pour la survie et
l'adaptation, que la libert\'e cr\'eatrice du sujet dans un univers ouvert \`a sa volont\'e. }

\paragraph[1.3.2.1 Histoire du milieu ]{1.3.2.1 Histoire du milieu }
\hypertarget{RefHeading271699228146}{}{\sffamily\color{black}
Pour illustrer au mieux l'extraordinaire f\'econdit\'e philosophique de la notion de milieu, nous allons tout d'abord
essayer de comprendre la trajectoire de ce mot durant les si\`ecles derniers Error: Reference source not found. Sans
pour autant remonter \`a son aube \'etymologique, nous nous apercevons que le mot \textit{milieu} d\'ecrit un trajet
singulier dans le monde des sciences. Employ\'e d\`es le 16\textsuperscript{e} si\`ecle par Descartes dans son
\textit{Trait\'e de la lumi\`ere,} il repr\'esente pour Newton une mesure de distance dans l'\'ether, cette
non-mati\`ere qui structure la gravitation et fait se mouvoir les objets. D\'efini plus tard par d'Alembert dans son
encyclop\'edie comme un : \textit{{\textquotedbl}espace naturel dans lequel un corps est plac\'e, qu'il se meuve ou
non{\textquotedbl}}, le mot connaitra durant tout le XIX\`eme un large d\'eveloppement s\'emantique en s'\'etendant de
la physique \`a la biologie. Les naturalistes fran\c{c}ais de l'\'epoque affirme que le milieu n'est pas seulement
\textit{environnant} mais influe sur les \^etres vivants. Ainsi Lamarck d\`es 1809 \'ecrira dans sa \textit{Philosophie
Zoologique}: \textit{{\textquotedbl}le milieu a une grande puissance pour modifier les organes{\textquotedbl}}. Par la
suite dans l'histoire du XX\textsuperscript{\`eme} si\`ecle, nous pouvons identifier deux moments marquants pour ce
concept et structurants pour l'histoire des sciences dans son ensemble Error: Reference source not found. }

{\sffamily\color{black}
Le premier se joue sous la plume d'Auguste Comte qui, inspir\'e de la biologie naissante, articule le vital au social
dans la sociologie naissante et d\'ecrit le milieu comme \textit{{\guillemotleft}~l'ensemble des circonstances
ext\'erieures ({\dots}) n\'ecessaires \`a l'existence de chaque organisme d\'etermin\'e~{\guillemotright}
}\textit{Error: Reference source not found}. Sans sombrer pour autant dans un d\'eterminisme total, Comte introduit une
premi\`ere dialectique des rapports avec le milieu comme conditions de possibilit\'e de la vie :
\textit{{\guillemotleft}~Tout \^etre vivant ({\dots}) modifie sans cesse son milieu.~{\guillemotright}}, \'ecrit-t-il
alors\textit{. }}

{\sffamily\color{black}
Le concept qui connait un succ\`es croissant en France est peu de temps apr\`es r\'eappropri\'e par les penseurs
d'Outre-Rhin qui lui donnent alors un sens diff\'erent. S'opposant au francis\'e \textit{Der Milieu}, le g\'eographe
Ratzel introduit dans son \textit{Anthropog\'eographie} (1899) le mot \textit{Umwelt} qui se d\'emarque rapidement par
sa dimension fortement d\'eterministe. Dans un m\^eme mouvement, le physiologue et biologiste Jakob von Uexk\"ull
\'etudie dans son laboratoire la tique et se rend compte que le milieu de la tique se d\'efinit non pas par tout
l'environnement qui l'entoure mais seulement par ce qui lui est utile et appropri\'e. Le milieu devenu \textit{Umwelt
}s'oppose alors \`a l'environnement indiff\'erenci\'e et devient l'ensemble des \'el\'ements qui sont porteurs de
significations \textit{(Merkmaltr\"ager)} pour un \^etre\textit{.} }

\paragraph[1.3.2 Le milieu comme projet politique]{1.3.2 Le milieu comme projet politique}
\hypertarget{RefHeading291699228146}{}{\sffamily\color{black}
Uexk\"ull propose ainsi une \textit{{\guillemotleft}~biologie subjective~{\guillemotright}} qui \'etudierait les
relations de chaque esp\`ece avec son milieu. Dans l'Allemagne du si\`ecle d\'ebutant, Uexk\"ull cherche \`a diffuser
largement sa th\'eorie qui s'adresse pas tant aux animaux qu'\`a l'humain dont le milieu serai la Patrie
\textit{(Heimat) }\textit{Error: Reference source not found}. Refl\'etant les d\'ebats guerriers entre la
\textit{Kultur} allemande et la \textit{Civilisation} fran\c{c}aise Error: Reference source not found, se cristallise
dans l'id\'ee de Milieu une tension politique sur les relations entre nature et vivant qui devra d\'echirer l'Europe
pendant longtemps encore. Foucault dans son cours au Coll\`ege de France du 11 Janvier 1978 parle de l'influence de
l'id\'ee de milieu sur la conception du territoire pour les urbanistes du XVIII\textsuperscript{\`eme} \ si\`ecle.
Quand sous Louis XIV, les villes \'etaient construites d'apr\`es un espace con\c{c}u comme vide (voir
\textit{Richelieu} en Indre-et-Loire), la question de l'urbaniste du XVIII\textsuperscript{\`eme} est de comprendre la
ville dans son \'evolution future. L'enjeu est devenu \textit{l'adaptation} du milieu existant (\`a la fois urbain et
naturel), la transformation du \textit{donn\'e }compris comme un \'el\'ement qu'on peut venir modifier\textit{.}
L'id\'ee sous-jacente de milieu ouvre la possibilit\'e de l'appropriation de la nature. En terme foucaldien, cette
nouvelle bio-politique se fonde sur la territorialisation du milieu comme nouveau centre des enjeux de pouvoir.}

{\sffamily\color{black}
L'\`ere industrielle r\'ealise ce projet d'une adaptation \`a la fois \textit{au }et \textit{du }milieu vu comme tension
n\'ecessaire de l'\'evolution, passage oblig\'e vers la civilisation. Alors que l'humain s'est vu d\'eplac\'e dans son
r\^ole central par l'astronomie galil\'eenne puis l'\'evolution darwinienne, l'id\'ee de milieu pose comme enjeu majeur
du vivant la maitrise de l'environnement - la lutte pour ne pas \^etre maitris\'e. Poursuivi par la psychanalyse de
Freud qui introduit l'Autre au sein du sujet, l{}'entreprise de d\'ecentrement de l'humain vers son milieu se joue
d\`es l'abord dans les termes de la vie ou de la mort. Plus tard, Lacan identifiera la transition de la petite enfance
\`a l'enfance par le \textit{{\guillemotleft}~}\textit{stade du miroir~{\guillemotright}} comme moment o\`u l'enfant
diff\'erencie enfin le Milieu (\textit{Umwelt}) du Soi (\textit{InnenWelt}) Error: Reference source not found\textit{.
}Ce {\textquotedbl}passage au milieu{\textquotedbl} est donc d'une importance capitale puisque s'y joue la constitution
de l'\^etre.}

\paragraph[1.3.3 D\'esu\'etude et renouveau du milieu]{1.3.3 D\'esu\'etude et renouveau du milieu%
%tu dois am\'eliorer la partie sur les milieux : tu passes en revue beaucoup de disciplines mais tu oublies la g\'eographie, cad les relations milieu/ espace. C{}' est curieux, car tu t{}' appuies avant sur les g\'eographes anglo{}-saxons. Un temps la g\'eographie a \'et\'e la science des milieux ( Vidal de la Blache) : le milieu faisait les hommes (d\'eterminisme naturel), etc...
%Puis, influenc\'e par l{}' ext\'erieur, elle est devenue {\textquotedbl} nouvelle g\'eographie{\textquotedbl} : nouveau paradigme avec l{}'espace, le territoire, le lieu, et m\^eme le g\'eon voire les chor\`emes ( Brunet). Bref elle se dotait d{}'outils scientifiques et de techniques (g\'eomatique,{}'...). Puis Internet! Et ses m\'etaphores g\'eographiques.
%Alors l\`a, patatras! Qui des vieux outils? Comment d\'ecrire, visualiser, analyser ce qui s{}' y produit et ses realtions avec le vieil espace : notre bon plancher des vaches?
%Certains r\'eintroduisent la notion de milieu, d{}'autres ignorent tout cela, certains encore utilisent les outils classiques et de multiples acceptions de cyberespace ( ou autre terme approchant) fleurissent. Il te faudrait pr\'esenter m\^eme succintement ces termes montrant la relation \`a l{}'espace de l{}'Internet ( synapse ou dispositifs socio{}-techniques en font partie). Ma HDR en parle un peu( cherche Batty ou Zook comme auteur). Mais fais attenttion tout auteur utilise ses propres notions et cherchent \`a faire pipi dessus pour marquer son propre territoire. Il n{}' y a pas vraiment de consensus. 
%clemsos
%April 22, 2014 10:28 AM
}
\hypertarget{RefHeading311699228146}{}{\sffamily\color{black}
Cet engouement du si\`ecle dernier semble aujourd'hui quelque peu enray\'e et l'id\'ee de milieu tend \`a dispara\^itre
des domaines qui l'ont faite. En g\'eographie tout d'abord, elle est de moins en moins enseign\'ee Error: Reference
source not found peu \`a peu remplac\'ee par la notion d'environnement. L'\'ecologie, en ce sens, a largement
recentr\'e le milieu sur le r\^ole d\'eterminant de l'Homme par l'introduction du concept \textit{d'environnement}
Error: Reference source not found. Les cons\'equences de ce passage de l'id\'ee de nature \`a celle de milieu restent
profondes, notamment dans le droit civil o\`u nous sommes pass\'es d'un rapport du {\guillemotleft}~droit
impos\'e~{\guillemotright} de la nature au {\guillemotleft}~droit n\'egoci\'e~{\guillemotright} du milieu Error:
Reference source not found. L'approche du milieu comme \'el\'ement unificateur des sciences est \'egalement un sujet
toujours en discussion. M\'eta-r\'eflexion, la discussion sur le {\textquotedbl}milieu acad\'emique{\textquotedbl}
donne lieu \`a d'int\'eressants \'echanges Error: Reference source not found qui interrogent notamment la d\'efinition
trop abrupte des disciplines scientifiques et leur herm\'eticit\'e. Parfois nomm\'e \textit{m\'esologie}, l'\'etude du
milieu se donne pour mission de r\'econcilier des pratiques diverses de la biologie \`a la sociologie en imaginant une
\'etude par le milieu Error: Reference source not found. Les {\oe}uvres de Deleuze \& Guattari, dont la m\'esologie se
revendique, parlait d\'ej\`a de la pratique d'une philosophie du milieu : \textit{{\textquotedbl}Partir au milieu, par
le milieu, entrer, sortir, non pas commencer ni [FB01?]nir, [{\dots}] renverser l'ontologie, destituer le fondement,
annuler [FB01?]n et commencement.[{\dots}] C'est que le milieu n'est pas du tout une moyenne, c'est au contraire
l'endroit o\`u les choses prennent de la vitesse.{\textquotedbl}} Error: Reference source not found }

{\color{black}
La philosophie quant \`a elle n'a cess\'e de s'interroger sur cette question du milieu. La r\'eflexion sur la technique
et les technologies s'est notamment saisie \`a bras le corps de cette notion, avec notamment l'id\'ee de \textit{milieu
technique} par Friedman et Leroi-Grouhan Error: Reference source not found. Tout geste (du plus banal au plus rare)
s'effectuerait dans un milieu technique qui le rend possible. Gilbert Simondon dans son livre \textit{Du mode
d'existence des objets techniques} continue cette r\'eflexion avec ce qu'il appelle le \textit{milieu associ\'e} : }

{\color{black}
\textit{{\guillemotleft}~m\'ediateur de la relation entre les \'el\'ements techniques fabriqu\'es et les \'el\'ements
\ \ naturels au sein desquels fonctionne l'\^etre technique. (...) C'est ce milieu associ\'e \ \ qui est la condition
d'existence de l'objet technique invent\'e.~{\guillemotright} }}

{\raggedleft\color{black}
Error: Reference source not found. 
\par}


\bigskip

{\color{black}
Simondon probl\'ematise le milieu associ\'e comme vecteur de \textit{l'individuation,} o\`u se produit la rencontre
entre objets et individus pour chacun les actualiser. Poursuivant ce travail, Stiegler comprend les technologies de
l'information comme un milieu essentiellement social, comprenant \`a la fois ce\textit{ }qui est autour de l'individu
(environnement) et entre les individus (medium)\textit{ }\textit{Error: Reference source not found}\textit{.} Sans sa
lecture critique des industries culturelles, Stiegler formule l'id\'ee qu'un milieu est \textit{associatif} s'il permet
l'individuation. A l'inverse, certains milieux seraient \textit{dissociatifs} car ils ne permettraient pas le devenir
individu, \`a l'image des mass media qui divisent producteurs et consommateurs de symboles. La dynamique industrielle
des deux derniers si\`ecles a entra\^in\'e une massification des ph\'enom\`enes culturels, cr\'eant un milieu technique
consid\'er\'e comme largement dissociatif car d\'e-r\'ealisant les individualit\'es (Simondon, 1989). N\'eanmoins, le
renouveau technologique port\'e par l'apparition des technologies num\'eriques ouvre aujourd'hui une page nouvelle pour
l'individuation en offrant un milieu extr\^emement associatif, fond\'e pour ainsi dire sur le lien. Voyant un nouvelle
\^age des Lumi\`eres Error: Reference source not found, Stiegler con\c{c}oit l'Internet comme un milieu qui ne serait
pas structurellement dissociatif et pourraient donc recr\'eer de nouvelles formes plus horizontales d'\'economie
symbolique o\`u existent davantage de symboles partag\'es. }

\paragraph[1.3.4. Milieu num\'erique et individuation]{1.3.4. Milieu num\'erique et individuation}
\hypertarget{RefHeading331699228146}{}{\color{black}
Loin d'\^etre d\'efini comme un ensemble d'entit\'es abstraites, nous abordons ici le milieu num\'erique dans une
perspective incarn\'ee : les \textit{objets num\'eriques}. S'opposant aux objets naturels et techniques, les objets
num\'eriques sont \`a comprendre dans leurs relations mat\'erielles avec le \textit{code} et temporelles par leur
production et archivage dans les m\'emoires des donn\'ees du web. Le milieu num\'erique dans lequel chacun \'evolue se
pr\'esente donc sous la forme des diff\'erents liens qui unissent les objets num\'eriques et leurs actualisations. Nous
devons concevoir l'\'etude du milieu \textit{{}``en tant que trajectivit\'e'' }(Watsuji, 2011), en observant les
trajets d\'ecris par ces objets num\'eriques au sein de milieux num\'eriques. L'espace des possibles offre un lieu pour
une individuation non-restreinte fond\'ee sur l'appropriation de ces objets num\'eriques et nous allons donc chercher
\`a comprendre quels sont les dimensions et expressions d\'ecelables par l'analyse de ces objets num\'eriques. Les
travaux des m\'ediologues nous ont notamment montr\'e comment de nouvelles technologies du langage et de l'\'ecriture
permettent le d\'eveloppement de nouveaux modes d'expression et de pens\'ee (Debray, 1994). }


\bigskip

{\color{black}
Associateur ou dissociateur, l'acc\`es et le protocole sont les enjeux politiques du milieu num\'erique (Hui, 2012). Les
discursivit\'es qui s'appliquent sur les objets num\'eriques conditionnent leurs existences en tant qu'\'el\'ement de
milieu, enceintes dans les protocoles qui g\`erent leurs acc\`es. McKenzie dans son livre \textit{Hacker Manifesto}
(2004) effectue une lecture marxiste de l'\'economie des objets num\'eriques et identifie une classe
``\textit{vectorialiste}{}`` qui poss\`ede les moyens de faire se mouvoir une information en cr\'eant des liens se
dirigeant vers elle. Cette capacit\'e est pour McKenzie un enjeu \'economique accapar\'e par les mass media gr\^ace au
contr\^ole de la technologie. Le hack, entendu comme l'activit\'e permettant de s'approprier ce pouvoir gr\^ace au
savoir-faire technologique, permettrai de court-circuiter les objectifs de la ``classe vectorialiste'' pour proc\'eder
\`a la cr\'eation ou la diffusion d'objets num\'eriques soi-m\^eme. Plus g\'en\'eralement, le \textit{hack }devient le
propre du d\'etournement d'usage, l'action qui fait qu'un objet est utilis\'e \`a des fins pour lesquelles il n'a pas
n\'ecessairement \'et\'e con\c{c}u. Cette dynamique produite par le milieu associ\'e permet l'appropriation d'un objet
par tout un chacun pour son individuation, alors m\^eme que l'objet n'en poss\`ede pas n\'ecessairement la
propri\'et\'e. Les dynamiques d'innovation issues du logiciel libre dans les communaut\'es d'utilisateurs et
d'inventeurs en Chine donnent un bel exemple de processus o\`u l'adaptation devient force d'appropriation puis
d'individuation (Lindtner, 2012). Ainsi, nous essayons de comprendre comment la circulation de ces objets num\'eriques
et leur appropriation (notamment par le \textit{hack}) informe et exprime le milieu num\'erique auquel ils
appartiennent. }


\bigskip


\bigskip

{\color{black}
Pour mener \`a bien cette \'etude du milieu num\'erique en Chine et afin de pouvoir \'etudier de fa\c{c}on empirique les
dynamiques \`a l'{\oe}uvre dans et autour de ces objets num\'eriques, nous avons choisi de nous int\'eresser \`a un
objet particuli\`erement pr\'esent sur les r\'eseaux sociaux : les m\`emes Internet. }


\bigskip


\bigskip

{\color{black}
\textsf{Ici de nouveaux acteurs font irruption dans le traitement des donn\'ees, mettant \`a jour un autre enjeu~pour la
visualisation : }\textsf{\textit{l'interface}}\textsf{. Nous retrouvons alors la
2}\textsf{\textsuperscript{\`eme}}\textsf{ cat\'egorie d'usage de Bertin (le traitement de donn\'ees par l'image pour
la compr\'ehension), a ceci pr\`es que l'image devient ainsi un support sur lequel agir }\textsf{Error: Reference
source not found}\textsf{. Manovich dans un article r\'ecent montre comment l'interface d\'efinie comme
}\textsf{\textit{{\guillemotleft}~the ways to represent (`format') and control the signal.~{\guillemotright}
}}\textsf{devient un }\textsf{\textit{software}}\textsf{ sous l'influence notamment des normes et objets concus par les
manufactureurs }\textsf{Error: Reference source not found}\textsf{. Ce formatage nouveau de l'information induit des
changements dans la pratique de la lecture qui, toujours selon Manovich, s'apparenterai davantage \`a de la
reconnaissance de }\textsf{\textit{pattern, }}\textsf{symbolis\'e par l'usage de l'ic\^one et du menu en design
d'interface. Ainsi si l'interface contraint la lecture, la prise en compte des formes narratives (les
}\textsf{\textit{pattern de Manovich)}}\textsf{ prend une grande importance quand il s'agit de concevoir une
visualisation d'information. L'usage des signes graphiques doit se faire avec une connaissance des usages de
l'interface, afin de recr\'eer la }\textsf{\textit{coop\'eration textuelle}}\textsf{ des r\^oles de lecteur et de
designer/auteur n\'ecessaire pour la production un sens }\textsf{Error: Reference source not found}\textsf{. }}


\bigskip


\bigskip
\end{document}
