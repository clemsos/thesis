\chapter*{Conclusion}

% principal statement achievement
\newthought{Le dispositif d'analyse} que nous avons présenté ici nous a permis d'observer la diversité des contenus circulant sur le réseau social chinois \textit{Sina Weibo}. Nous avons pu voir les différents aspects sémantiques, conversationnels et spatio-temporels entrant en jeu dans la construction des objets numériques. Le regard porté sur ces échanges quotidiens nous a  mené à la définition d'une première typologie des faits de communication d'après leurs \textit{topogrammes}. Cette modélisation schématique a également permis de mettre à jour la tension entre énonciation et discours qui s'articulent dans la production de différentes formes médiatiques.

% Your original contribution
\subsubsection{Topogrammes et régimes d'expression en ligne}
Le \textit{mème Internet}, proche de la blague en ligne, procède d'une énonciation spontanée concernant de nombreux petits groupes très actifs. A l'inverse, certaines constructions comme les campagnes par \textit{hashtags} semblent vouloir territorialiser des espaces d'expression entiers par des stratégies d'actions planifiées. Une analyse préliminaire portant sur les contenus représentés par les hashtags les plus utilisés dans le corpus a notamment montré les grandes similarités existantes entre contenus majoritaires des réseaux sociaux et ceux d'autres media de masse comme la télévision (divertissement, loisirs, produits culturels, etc.)

Nous avons pu également constaté que les modèles géographiques de diffusion semblaient suivre les chemins observables pour les médias traditionnels en Chine. Les actualités plus controversées sont davantage portées par des médias issus de Canton alors que Pékin joue un rôle central dans les annonces officielles. Certains médias et télévisions locales rayonnent largement dans le  divertissement, sans pour autant apparaitre dans d'autres domaine de la vie médiatique chinoise. Les mèmes comiques quant à eux semblent suivre des parcours plus atypiques, avec des foyers d'adoption localisés et une diffusion peu prévisible.

Les figures produites par notre dispositif l'analyse de données nous ont également permis de voir différents modes de construction des discussions. La structure sémantique entourant une émission de divertissement télévisée se 
caractérise par une architecture convenue autour de mots-clés pré-définis. Le conversation se centralise autour de peu de diffuseurs qui posent les questions, suggèrent les émotions et crée ainsi une activité très intense pendant un temps très bref. A l'inverse, un mème absurdiste se construit autour de jeux de mots et son champ lexical se constitue donc d'une grande diversité de mots, tous accolés à un ou deux mots-clés qui sont le centre d'intérêt. Les faits divers et les débats sur l'actualité regroupent quant à eux un grand nombre d'utilisateurs qui discutent de façon plutôt ordonnée en petits groupes.

De nombreux modèles de diffusion coexistent donc sur les réseaux sociaux chinois. Chacun d'entre eux peut être abordé par le prisme de régimes d'expressions particuliers. La diffusion fortement organisée et planifiée des campagnes publicitaires semble pouvoir s'analyser grâce à une lecture foucaldienne de l'ordre du discours. Les mèmes, quant à eux,  ne procèdent pas d'un régime discursif établi et bénéficieraient d'être approchés comme actes d'énonciation. Les discussions entourant les scandales et faits d'actualité peuvent être envisagées comme un espace de négociation de l'ordre du discours par l'énonciation.

% What pre-existing views were challenged
Nous pouvons constater que les phénomènes observables sur les réseaux sociaux sont difficilement envisageables dans une cohérence d'ensemble. Leur cohérence peut certes être abordée par l'étude des plate-formes et des outils, souvent riches en enseignements. Néanmoins, une étude des contenus et des usages doit prendre en compte la diversité des pratiques et des régimes d'expression en présence. Sur Internet comme dans la vie, on ne parle pas de la même façon à un inconnu, un ami, un collègue ou un représentant d'une marque de vêtement. Ces différences ne sont pas seulement des subtilités de l'analyse, mais bien des composantes structurelles de l'existence de ces conversations, comme le montre ici les différents topogrammes. 

Ainsi, il paraît hasardeux de vouloir qualifier a priori les objets numériques et à plus forte raison les milieux numériques. Si la notion de milieu numérique nous a permis de problématiser les relations entre usagers, protocoles et producteurs de contenus, il semble néanmoins de définir précisément son champ d'application. Les topogrammes portent peu de traces de l'actualisation du milieu. Des études plus spécifiques et localisées seraient plus à même de vérifier l'opérabilité du concept de \textit{milieu numérique} dans différents contextes. 

% social mass media
\subsubsection{Les médias sociaux de masse} % (fold)

La forte contextualisation de cette étude autour du cas \textit{Sina Weibo} nous a néanmoins permis de considérer le rôle du contexte économique et politique dans la production et la circulation des contenus. Nous avons vu comment les faits divers et les scandales politiques faisaient couler beaucoup d'encre en mobilisaient particulièrement l'attention dans les grandes villes. Cette vivacité des débats entourant les développements économiques, sociaux et politiques de la Chine moderne trouve un reflet dans les échanges en ligne. Les multiples contradictions et questions amenées par l'urbanisation croissante du pays trouvent des expressions quotidiennes, notamment au travers de la satire et de la dérision. Les innombrables déclinaisons des mèmes comiques produits par une myriade de petits groupes communiquant peu entre eux sont souvent teintés d'un humour noir, exutoire et passe-nerfs des angoisses modernes des internautes chinois. 

Cette nécessité de s'exprimer et de communiquer se corrèle de façon directe à une stratégie économique et politique de développement du secteur industriel des médias. La plate-forme de microblog chinoise joue bien souvent le rôle d'intermédiaire dans les échanges parfois houleux entre gouvernement et utilisateurs entourant les questions de société. Le protectionnisme et les investissements d'Ètat permettent de soutenir la croissance de services Web commercialement très profitables. Les industries de contenus comme \textit{Sina} trouvent une continuité directe dans le microblog et poursuivent leur intégration avec les magazines ou la télévision. Cette diversification du rôle de diffuseur vers le reste de la chaîne de production de l'information est à la fois une tradition des grands groupes médiatiques, mais également une nécessité permettant de réduire les coûts de contrôle de l'information. En effet, l'instabilité de la qualité du service engendrée par le contrôle politique est importante et amène à des pertes non-négligeables pour les entreprises Web. 

Les phénomènes de contrôle des contenus, souvent regroupés un peu rapidement à propos de la Chine sous le nom de ``censure'',  ont des équivalents directs pour la plupart des services Web dans le monde. La poussée actuelle pour davantage de régulations des usages et de la publication en ligne font de \textit{Sina Weibo} pas seulement un exemple exotique, mais bien un précédent intéressant.


% Analyse de données et modèle de diffusion complexe
\subsubsection{Limites et opportunités de l'analyse de données}

L'analyse d'un vaste ensemble de données nous a permis de construire une typologie enrichie des diverses formes typiques de communication en ligne. Un outil de visualisation interactif montrant la circulation des mots et phrases entre les utilisateurs sur le territoire chinois nous a permis d'observer les topogrammes de plusieurs contenus choisis. Le choix des exemples étudiés ainsi que celui des outils utilisés a été effectué lors d'un processus expérimental qui forge le cœur de ce travail. De nombreuses expérimentations et tentatives parfois infructueuses nous ont amené au développement d'un outil finalement utilisable.

La difficulté d'identifier des contenus intéressants dans une large masse de données a notamment été un des problèmes importants. L'approche algorithmique de détection par des méthodes dites de \textit{machine learning} n'a pas pu aboutir, nécessitant trop de ressources de calcul pour un résultat peu sûr. La seconde approche utilisant comme support les hashtags n'a pas permis de découvrir des mèmes intéressants. Néanmoins, nous avons pu mettre à jour d'autres aspects intéressants et complémentaires contenus dans notre jeu de données d'origine. Enfin, l'indexation et la recherche plein-texte a offert une solution convenable, nécessitant néanmoins une recherche parallèle dans la littérature académique et secondaire.

L'ensemble de ces tentatives plus ou moins réussies montre que le travail d'analyse de données s'effectue de manière itératif et non-définitive. Alors que la recherche en informatique et algorithmique s'intéresse aux possibilités d'analyse,  l'existence d'un résultat est la condition sine qua non de la pratique du \textit{data mining}. Cette recherche d'un élément intéressant dans les données ne peut s'effectuer sans posséder d'une part une connaissance aigüe de l'objet et du terrain étudié, et d'autre part une maitrise des outils et langages nécessaires à sa réalisation. 

L'apparition de nouveaux langages dans l'écriture des' processus de la recherche scientifique est un élément de changement important qui nécessite de nouveaux efforts et ajustements. Nous avons mis l'accent sur la disponibilité des données et du code produits dans ce travail dans le but d'une ré-utilisation pour des terrains futurs\footnote{Le code est disponible sur \url{http://github.com/topogram}, consulté le 4 Juillet 2014 à 16h15}. 

% limites et travaux futurs
Le présent travail s'est donné pour tâche d'établir les fondements méthodologiques et conceptuels qui permettent de mener à bien une étude plus systématique des formes de circulation des contenus sur Internet. Les relations entre régimes d'expression et topogrammes définies dans ce premier travail nécessitent d'être affinées et approfondies. Nous avons notamment fait attention à ne pas généraliser trop rapidement des analyses portant sur le terrain bien particulier de \textit{Sina Weibo}. Un projet de recherche post-doctorant financé par l'Agence National de la Recherche (ANR) autour des modèles d'innovation ouvertes en Chine
\footnote{
    Appel à projet \textit{Sociétés innovantes} (INOV) en 2013 : \textit{OPIMPUC - OPen Innovation : Models and Places in Urban China}. Voir l'annonce officielle \url{http://www.agence-nationale-recherche.fr/en/anr-funded-project/?tx_lwmsuivibilan_pi2\%5BCODE\%5D=ANR-13-SOIN-0006}, consulté le 5 Juillet à 17h12
} permettra d'ouvrir le champ de réflexions et de poursuivre ce travail sur un terrain. Également, nous pourrons confronter les résultats observés par l'analyse de données et celles des topogrammes à des éléments qualitatifs issus d'interviews ou de rencontres. 

Au travers de ce travail, nous pouvons donc saisir l'importance d'établir un champ d'études pluri-disciplinaire autour de l'analyse de données afin d'en affiner les méthodes et de mieux en saisir les objets.
