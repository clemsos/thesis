\chapter*{Conclusion}

% principal statement
\newthought{De multiples modèles} décrivent la diffusion des différents objets numériques circulant sur le réseau social chinois \textit{Sina Weibo}. 
% achievement
Grâce à un outil d'analyse et de visualisation de données, nous avons pu observer comment cette circulation se construit selon différents aspects sémantiques, conversationnels et spatio-temporels.

% Your original contribution
L'étude des figures produites montrent que les actes de communication en ligne s'articulent dans une tension entre énonciation et discours qui produit des formes médiatiques distinctes. Le \textit{mème Internet}, proches de la blague en ligne, semblent procéder d'une énonciation très spontanée de petits groupes entre eux. A l'inverse, certaines constructions comme les campagnes par \textit{hashtags} semblent vouloir territorialiser des espaces d'expression entiers par des actions planifiés.

% What pre-existing views were challenged
Les phénomènes observables sur les réseaux sociaux sont donc difficilement être envisageables dans une cohérence d'ensemble. Leur cohérence peut être abordée d'une part par la spécificité des outils mais doit également prendre en compte la diversité des pratiques et des régimes d'expression en présence.

% social mass media
Le cas de \textit{Sina Weibo}, entre protectionnisme économique et contrôle politique, illustre la dimension industrielle et stratégique que revêt la diffusion de contenus en ligne. Notre analyse montre qu'une grande partie des contenus sur \textit{Sina Weibo} reproduit ceux des médias traditionnels, comme la télévision par exemple. La plate-forme de microblog chinoise jouent également un rôle d'intermédiaire dans les échanges parfois houleux entre gouvernement et utilisateurs dans les débats de société.

% Analyse de données et modèle de diffusion complexe
L'analyse d'un vaste ensemble de données nous a permis de construire une lecture enrichie des diverses formes typiques de communication en ligne. 
Notre outil de visualisation interactif nous a notamment permis de constituer différents \textit{topogrammes} représentants les mots et phrases circulant entre les utilisateurs sur le territoire chinois. 

% classification des discussions en ligne selon les structures de leurs diffusions
Les campagnes promotionnelles en ligne avec l’exemple de \textit{The Voice} notamment apparaissent architecturées autour d'un champ lexical défini par un réseau faits de peu d'utilisateurs dominants. Les faits d'actualité se caractérise par une diffusion mobilisant un grand nombre de groupes de discussions échangeant des idées plutôt structurées pendant un temps très court. Les faits divers et les scandales politiques mobilisent plus particulièrement l'attention dans les grandes villes, faisant couler beaucoup d'encres et reflétant ainsi la vivacité des débats entourant les développements économiques, sociaux et politiques de la Chine moderne. A l’inverse, les discussions des mèmes comiques sont fragmentées en une myriade de petits groupes très intégrés qui communiquent peu entre eux mais procède à d'innombrables déclinaisons lexicales caractéristiques des jeux de mots.

% régimes d'expression
De nombreux modèles de diffusion coexistent donc sur les réseaux sociaux chinois. Chacun d'entre eux peut être abordé par le prisme de régimes d'expressions particuliers. La diffusion fortement organisée et planifiée des campagnes publicitaires semblent pouvoir s'analyser grâce à une lecture foucaldienne de l'ordre du discours. Les mèmes quant à eux semblent regrouper avant tout des petits groupe de personnes autour de blagues et bénéficierai plus facilement d'une lecture comme actes d'énonciation, ne procédant pas d'un régime discursif établi. Les discussions entourant les scandales et faits d'actualité pourraient quant à eux être envisagés comme un espace de négociation de l'ordre du discours par l'énonciation.

% limites et travaux futurs
Les distinctions quelques peu artificielles qui définissent cette première typologie des régimes d'expression en ligne nécessitent d'être affinée et approfondie. Le présent travail s'est donné pour tâche d'établir les fondements méthodologiques et conceptuels qui permettrait de mener à bien une série d'études plus systématique et diversifiées. Ainsi, il a pris la forme de la conception d'un outil ré-utilisable sur différents objets et corpus. La faible opérabilité du concept de \textit{milieu numérique} et les résultats moyens de l'approche cartographique incitent à poursuivre vers des études plus spécifiques et localisées.Également, ces premiers résultats témoignent de l'importance d'établir un champ d'études pluri-disciplinaire pour une meilleure compréhension des méthodes et objets de l'analyse de données. 


% conclusion should be able to stand on its own and provide a justification and defence of the thesis
% Tie together, integrate, and synthesize the various issues raised in the discussion sections, while reflecting the introductory thesis statement(s) or objectives

% Make a clear and concise statement of the original contribution to knowledge found in your thesis
% indicate the importance of the subject discussed 
% restate the research questions as presented in the introduction chapter 
% reinforces the importance of the study and its findings. 
% summarize what was said in the different chapters
% What you researched
% A deduction made on the basis of the main body (i.e. Concluding statements)
% reaffirms the thesis statement, reaches a final judgment
% future direction for further researchs
% theoretical and policy implications 
% the practical implications of your results
% future work, les travaux à poursuivre, Recommend direction and areas for future research
% final section reminding readers of the original contribution and significance of your research to your field