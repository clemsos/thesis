\chapter*{Conclusion}

\newthought{Sina Weibo}


Nous avons donc procédé pour chaque mème à l{\textquoteright}extraction des graphes sémantiques (co-occurence de mots), conversationnels (interactions entre les utilisateurs), temporels (évolution du volume de tweets sur une période donnée) et géographiques (origine des provinces des utilisateurs lors des conversations). L{\textquoteright}étude de ces différents aspects de la diffusion a nécessité une modélisation particulière des relations entre chacun de ces aspects et nous avons donc été amené à créer un outil d{\textquoteright}observation et d{\textquoteright}analyse permettant de visualiser sous forme de graphes multiples ces différentes lectures de la diffusion des objets digitaux.  


Gr\^ace à l{\textquoteright}outil de visualisation interactif développé spécialement pour cette analyse, il nous a donc été possible d{\textquoteright}observer les structures et relations mutuelles de ces graphes. Cette exploration nous a notamment permis de mettre à jour des caractéristiques spécifiques à certains mèmes parmi ceux sélectionnés.


L{\textquoteright}approche géographique a néanmoins donné assez peu de résultats, en proposant pas de structures nécessairement redondantes. Ainsi, la qualification d{\textquoteright}un milieu numérique par la classification de ces objets digitaux selon ses topogrammes semblent une t\^ache irréalisable dans cette étude. 


\lipsum[4-10]