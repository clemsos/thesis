\chapter*{Conclusion}

% principal statement  achievement

\newthought{Les objets numériques} circulant sur le réseau social chinois \textit{Sina Weibo} se construisent selon différents aspects sémantiques, conversationnels et spatio-temporels.

% Your original contribution
L'étude des figures produites par notre dispositif d'analyse et de visualisation de données nous a montré que les actes de communication en ligne s'articulent dans une tension entre énonciation et discours qui produit des formes médiatiques distinctes. Le \textit{mème Internet}, proche de la blague en ligne, procède d'une énonciation spontanée concernant de nombreux petits groupes. A l'inverse, certaines constructions comme les campagnes par \textit{hashtags} semblent vouloir territorialiser des espaces d'expression entiers par des actions planifiées.

% What pre-existing views were challenged
Les phénomènes observables sur les réseaux sociaux sont donc difficilement envisageables dans une cohérence d'ensemble. Leur cohérence peut être abordée d'une part par l'étude des plate-formes et outils, mais elle doit d'autre part prendre en compte la diversité des pratiques et des régimes d'expression en présence.

% social mass media
La stratégie industrielle \textit{Sina Weibo}, entre protectionnisme et contrôle politique, illustre bien les enjeux de la diffusion de contenus en ligne. Prolongements actuels des industries culturelles, les contenus de \textit{Sina Weibo} reproduisent largement ceux des médias traditionnels, comme la télévision par exemple. La plate-forme de microblog chinoise joue également un rôle important d'intermédiaire dans les échanges parfois houleux entre gouvernement et utilisateurs entourant les questions de société.

% Analyse de données et modèle de diffusion complexe
L'analyse d'un vaste ensemble de données nous a permis de construire une typologie enrichie des diverses formes typiques de communication en ligne. Différents \textit{topogrammes} ont pu être produits grâce à l'outil de visualisation interactif, montrant la circulation des mots et phrases entre les utilisateurs sur le territoire chinois. 

% classification des discussions en ligne selon les structures de leurs diffusions
Les campagnes promotionnelles en ligne, avec l’exemple de \textit{The Voice}, apparaissent architecturées autour d'un champ lexical défini par un réseau d'utilisateurs dominants de taille réduite. La diffusion des faits d'actualité rassemble un grand nombre de groupes d'utilisateurs échangeant des idées plutôt structurées pendant un laps de temps relativement court. Les faits divers et les scandales politiques mobilisent plus particulièrement l'attention dans les grandes villes, faisant couler beaucoup d'encre et reflétant ainsi la vivacité des débats entourant les développements économiques, sociaux et politiques de la Chine moderne. A l’inverse, les discussions des mèmes comiques sont fragmentées en une myriade de petits groupes très intégrés qui communiquent peu entre eux mais procèdent à d'innombrables déclinaisons lexicales caractéristiques des jeux de mots.

% régimes d'expression
De nombreux modèles de diffusion coexistent donc sur les réseaux sociaux chinois. Chacun d'entre eux peut être abordé par le prisme de régimes d'expressions particuliers. La diffusion fortement organisée et planifiée des campagnes publicitaires semble pouvoir s'analyser grâce à une lecture foucaldienne de l'ordre du discours. Les mèmes, quant à eux,  ne procèdent pas d'un régime discursif établi et bénéficieraient d'être approchés comme actes d'énonciation. Les discussions entourant les scandales et faits d'actualité peuvent être envisagées comme un espace de négociation de l'ordre du discours par l'énonciation.

% limites et travaux futurs
Ces distinctions quelques peu artificielles définissent cette première typologie des régimes d'expression en ligne et nécessitent d'être affinées et approfondies. Le présent travail s'est donné pour tâche d'établir les fondements méthodologiques et conceptuels qui permettent de mener à bien leur étude systématique. Ainsi, il a pris la forme de la conception d'un outil ré-utilisable sur différents objets et corpus. La faible opérabilité du concept de \textit{milieu numérique} et les résultats moyens de l'approche cartographique incitent à poursuivre en se dirigeant vers des études plus spécifiques et localisées. Ces premiers résultats témoignent de l'importance d'établir un champ d'études pluri-disciplinaire pour une meilleure compréhension des méthodes et objets de l'analyse de données. 


% conclusion should be able to stand on its own and provide a justification and defence of the thesis
% Tie together, integrate, and synthesize the various issues raised in the discussion sections, while reflecting the introductory thesis statement(s) or objectives

% Make a clear and concise statement of the original contribution to knowledge found in your thesis
% indicate the importance of the subject discussed 
% restate the research questions as presented in the introduction chapter 
% reinforces the importance of the study and its findings. 
% summarize what was said in the different chapters
% What you researched
% A deduction made on the basis of the main body (i.e. Concluding statements)
% reaffirms the thesis statement, reaches a final judgment
% future direction for further researchs
% theoretical and policy implications 
% the practical implications of your results
% future work, les travaux à poursuivre, Recommend direction and areas for future research
% final section reminding readers of the original contribution and significance of your research to your field