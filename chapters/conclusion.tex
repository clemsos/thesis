\chapter*{Conclusion}
% conclusion should be able to stand on its own and provide a justification and defence of the thesis
% Tie together, integrate, and synthesize the various issues raised in the discussion sections, while reflecting the introductory thesis statement(s) or objectives

% Make a clear and concise statement of the original contribution to knowledge found in your thesis
% indicate the importance of the subject discussed 
% restate the research questions as presented in the introduction chapter 
% reinforces the importance of the study and its findings. 
% summarize what was said in the different chapters
% What you researched
% A deduction made on the basis of the main body (i.e. Concluding statements)
% reaffirms the thesis statement, reaches a final judgment
% future direction for further researchs
% theoretical and policy implications 
% the practical implications of your results
% future work, les travaux à poursuivre, Recommend direction and areas for future research
% final section reminding readers of the original contribution and significance of your research to your field



% principal statement
\newthought{L'observation} de différents objets numériques circulant sur le réseau social Sina Weibo nous a permis de constater la multiplicité des modèles de diffusion mis en œuvre. 

% achievement
Le développement d'un outil d'analyse et de visualisation de données nous  nous a notamment procurer des lectures croisées des différents aspects de cette circulation sous forme de graphes sémantiques, conversationnels et spatio-temporels.

% Your original contribution
Les actes de communication en ligne semblent s'articuler selon des régimes d'expression fort différents dont la plupart existent depuis bien avant l'Internet. En dégageant une typologie générale de ces usages, nous pouvons resituer ces formes particulières dans des perspectives théoriques existantes.

% What pre-existing views were challenged
Les phénomènes observables sur les réseaux sociaux ne peuvent donc difficilement être envisagés dans une cohérence d'ensemble. Leur cohérence peut être abordée par la spécificité des outils mais doit d'abord prendre en compte la diversité des pratiques qui les font exister.

% social mass media
Le cas de Sina Weibo et les spécificités du contexte politique et économique chinois nous ont notamment permis de montrer comment l'activité historique de Sina comme fournisseur de contenu influence très notablement la nature des contenus échangés par les utilisateurs. Une analyse portant sur les hashtags a notamment que la majorité des contenus s'apparentaient largement à ceux des médias traditionnels comme la télévision ou les magazines. Ainsi, les hashtags doivent être compris comme des artefacts représentatifs d'usages très planifiés. 

% Analyse de données et modèle de diffusion complexe
Notre approche expérimentale portant sur un vaste ensemble de données nous a permis de construire une lecture enrichie de plusieurs formes typiques de communication en ligne. Notre outil de visualisation interactif nous a notamment permis d'observer comment les mots et phrases circulaient entre les utilisateurs sur le territoire chinois. Cette représentation graphique a mis à jour certaines caractéristiques spécifiques sous la forme de figures que nous appelons \textit{topogrammes}.

% classification des discussions en ligne selon les structures de leurs diffusions
Les campagnes promotionnelles en ligne avec l’exemple de \textit{The Voice} notamment sont architecturées autour d'un champ lexical défini par un réseau faits de peu de diffuseurs importants. La diffusion entourant les faits d'actualité se caractérise par un grand nombre de groupes de discussions échangeant des idées plutôt structurées. Les faits divers et les scandales politiques mobilisent particulièrement l'attention dans les grandes villes, faisant couler beaucoup d'encres et reflétant ainsi la vivacité des débats entourant les développements économiques, sociaux et politiques de la Chine moderne. A l’inverse, les discussions des mèmes comiques sont fragmentées en une myriade de petits groupes très intégrés qui communiquent peu entre eux. La grande diversité lexicale observable reflète les innombrables déclinaisons humoristiques caractéristiques des jeux de mots.

% régimes d'expression
Si cette typologie reste partielle et porte seulement sur un échantillon restreint de contenus, nous pouvons voir que de nombreux modèles de diffusion coexistent sur les réseaux sociaux chinois. Chacun d'entre eux peut être abordé par le prisme de régimes d'expressions particuliers. La diffusion fortement organisée et planifiée des campagnes publicitaires semblent pouvoir s'analyser grâce à une lecture foucaldienne de l'ordre du discours. Les mèmes quant à eux semblent regrouper avant tout des petits groupe de personnes autour de blagues et bénéficierai plus facilement d'une lecture comme actes d'énonciation, ne procédant pas d'un régime discursif établi. Les discussions entourant les scandales et faits d'actualité pourraient quant à eux être envisagés comme un espace de négociation de l'ordre du discours par l'énonciation. 

% limites et travaux futurs
Les distinctions quelques peu artificielles qui définissent cette première typologie des régimes d'expression en ligne nécessitent d'être affinée et approfondie. Le présent travail s'est donné pour tâche d'établir les fondements méthodologiques et conceptuels qui permettrait de mener à bien une série d'études plus systématique et diversifiées. Ainsi, il a pris la forme de la conception d'un outil ré-utilisable sur différents objets et corpus. La faible opérabilité du concept de \textit{milieu numérique} et les résultats moyens données par l'approche cartographique incitent à poursuivre vers davantage des cas plus spécifiques et localisés. \'Egalement, il montre l'importance d'établir un champ d'étude pluri-disciplinaire pour une meilleure compréhension des méthodes et résultats obtenues par l'analyse de données. 


Une meilleure compréhension des nouvelles formes langagières du Web nées de la rencontre entre médias traditionnels et nouvelles technologies de la parole constitue aujourd'hui un enjeu théorique, pratique pour les peuples du le monde entier. 