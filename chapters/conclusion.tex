\chapter*{Conclusion}

% principal statement achievement
\newthought{Le dispositif d'analyse} que nous avons présenté ici nous a permis d'observer la diversité des contenus circulant sur le réseau social chinois \textit{Sina Weibo}. Nous avons pu aborder les différents aspects sémantiques, conversationnels et spatio-temporels entrant en jeu dans la construction des objets numériques. Le regard porté sur ces échanges quotidiens nous a mené vers une première typologie des faits de communication d'après leurs \textit{topogrammes}. Cette modélisation schématique a également permis de mettre à jour la tension entre énonciation et discours qui s'articule dans la production de différentes formes médiatiques.

% Your original contribution
\subsubsection{Topogrammes et régimes d'expression en ligne}
Le \textit{mème Internet}, proche de la blague en ligne, procède d'une énonciation spontanée. Son topogramme montre des petits groupes discutant très activement sans pour autant converser entre eux. Les mèmes absurdistes sont avant tout des jeux de mots et leur champ lexical se constitue donc une grande variété de déclinaisons autour d'un ou deux mots-clés situés au centre de l'intérêt. Suivant des parcours géographiques atypiques, ils se constituent autour des foyers d'adoption très localisés et offre finalement une diffusion peu prévisible. Parmi les différents contenus observés, les mèmes comiques semblent être les plus tenaces, apparaissant régulièrement même longtemps après leur première apparition.

A l'inverse, certaines constructions comme les campagnes par \textit{hashtags} semblent vouloir territorialiser des espaces d'expression entiers par des stratégies d'actions planifiées. L'enjeu de la communication stratégique en ligne et performatif, il s'agit de ``faire dire'' plutôt que de simplement exprimer. Le champ sémantique entourant une émission de divertissement télévisée se caractérise par une architecture convenue autour de mots-clés attendus. La conversation se centralise autour de peu de diffuseurs qui posent les questions, suggèrent les émotions et crée ainsi une activité très intense pendant un temps très bref. Une analyse portant sur les hashtags les plus utilisés dans le corpus a notamment montré les grandes similarités existantes entre les contenus majoritaires des réseaux sociaux et ceux d'autres media de masse comme la télévision (divertissement, loisirs, produits culturels, etc.)

Les faits divers et les débats sur l'actualité regroupent quant à eux un grand nombre d'utilisateurs qui discutent de façon plutôt ordonnée en petits groupes indépendants. Leurs diffusions semblent également suivre les modèles géographiques observés traditionnellement en Chine. Les actualités plus controversées sont portées par des médias plus libéraux de Canton alors que Pékin joue un rôle de diffuseur central dans les annonces officielles. Certains médias et télévisions locales rayonnent largement dans des domaines particuliers - comme le divertissement - sans pour autant occuper une place prédominante ailleurs dans la vie médiatique chinoise.

De nombreux modèles de diffusion coexistent donc sur les réseaux sociaux chinois. Chacun d'entre eux peut être considéré selon un régime d'expression particulier. La diffusion fortement organisée et planifiée des campagnes publicitaires satisfaire une lecture foucaldienne des relations de pouvoir établies par le discours. Les discussions entourant les scandales et les faits d'actualité peuvent être envisagées comme un espace de négociation de l'ordre du discours par l'énonciation. Les mèmes, quant à eux, ne procèdent pas d'un régime discursif établi et bénéficieraient d'être envisagés comme des actes d'énonciation. 

% What pre-existing views were challenged
Nous constatons que les phénomènes observables sur les réseaux sociaux sont très hétérogènes et difficilement envisageables dans leur ensemble. Leur cohérence peut être abordée par l'étude des plate-formes et des outils, souvent riches en enseignements. Néanmoins, une étude des contenus et des usages doit prendre en compte la diversité des pratiques et des régimes d'expression en présence. Sur Internet comme dans la vie, on ne parle pas de la même façon à un inconnu, un ami, un collègue ou un représentant d'une marque de vêtement. Ces différences ne sont pas seulement des subtilités de l'analyse, mais bien des composantes structurelles de l'existence de ces conversations, comme nous l'avons vu à partir des différents topogrammes.

Ainsi, il paraît hasardeux de vouloir qualifier a priori les objets numériques. La notion de \textit{milieu numérique} nous a permis de problématiser les relations entre usagers, protocoles et producteurs de contenus. Néanmoins, il semble difficile de définir précisément son champ d'application. Les topogrammes portent peu de traces de l'actualisation d'un milieu qui constitué de protocoles et témoignent plutôt de la diversité des usages possibles. Des études plus spécifiques et localisées seraient plus à même de vérifier l'opérabilité du concept de milieu numérique. 



% social mass media
\subsubsection{Les médias sociaux de masse} % (fold)

La forte contextualisation de cette étude autour du cas \textit{Sina Weibo} nous a permis de considérer le rôle du contexte économique et politique dans la production et la circulation des contenus. Nous avons vu comment les faits divers et les scandales politiques faisaient couler beaucoup d'encre en mobilisaient particulièrement l'attention dans les grandes villes. Ces échanges sont un reflet de la vivacité des débats entourant les développements économiques, sociaux et politiques de la Chine moderne. Les multiples contradictions et questions amenées par l'urbanisation croissante du pays trouvent des expressions quotidiennes, notamment par la satire et la dérision. Les mèmes comiques sont souvent teintés d'un humour noir, exutoire et passe-nerfs des angoisses modernes des internautes chinois. 

Cette nécessité de s'exprimer et de communiquer se corrèle de façon directe à une stratégie économique et politique de développement industriel du secteur médiatique. Le protectionnisme et les investissements d'Ètat soutiennent activement la croissance de services Web commercialement très profitables. En contrepartie, \textit{Sina Weibo} doit bien souvent jouer le rôle d'intermédiaire dans les échanges parfois houleux entre gouvernement et utilisateurs. En s'orientant vers le divertissement, les industries de contenus des magazines ou la télévision trouvent des débouchés directs dans le microblog. L'intégration du diffuseur au reste de la chaîne de production de l'information est à la fois une tradition des grands groupes médiatiques, mais également une nécessité pour réduire les coûts engendrés par les régulations toujours plus drastiques de la publication d'information. En effet, l'instabilité de la qualité du service due aux décisions politiques amène des pertes non-négligeables chez les entreprises Web. Ce manque-à-gagner peut être compensé par une diversification, permettant également un contrôle accru de la production d'information en amont. 

Ces phénomènes de contrôle des contenus, souvent regroupés un peu rapidement sous le nom de ``censure'',  ont des équivalents directs pour la plupart des services Web dans le monde. La tendance législative en Europe tend actuellement à établir un cadre réglementaire pour les fournisseurs et distributeurs publiant des informations en ligne. A ce titre, \textit{Sina Weibo} en fait  pas seulement figure d'exemple exotique, mais établit dans ce domaine un précédent tant par sa réussite économique que son modèle de coopération gouvernemental.


% Analyse de données et modèle de diffusion complexe
\subsubsection{Limites et opportunités de l'analyse de données}

L'analyse d'un vaste ensemble de données nous a permis de construire une typologie enrichie des diverses formes typiques de communication en ligne. Un outil de visualisation interactif montrant la circulation des mots et phrases entre les utilisateurs sur le territoire chinois nous a permis d'observer les topogrammes de plusieurs contenus choisis. Le choix des exemples étudiés et des outils utilisés a  nécessité un processus expérimental qui constitue le cœur de ce travail. De nombreuses expérimentations et tentatives parfois infructueuses nous ont mené vers le développement d'un outil finalement utilisable.

La difficulté d'identifier des contenus intéressants dans une large masse de données a notamment été un des problèmes importants. L'approche algorithmique de détection par des méthodes dites de \textit{machine learning} n'a pas abouti, nécessitant trop de ressources de calcul pour un résultat peu sûr. La seconde approche utilisant comme support les hashtags n'a pas permis de découvrir des mèmes intéressants. Néanmoins, nous avons pu mettre à jour d'autres aspects intéressants et complémentaires contenus dans notre jeu de données. Enfin, l'indexation et la recherche plein-texte a offert une solution satisfaisante, nécessitant néanmoins une recherche parallèle dans la littérature académique et secondaire.

L'ensemble de ces tentatives plus ou moins réussies montre que le travail d'analyse de données s'effectue de manière itérative et non-définitive. Tâtonnements, tentatives et hésitations ont jalonnés ce travail. L'exploration progressive du jeu de données grâce à des travaux préliminaires ont façonnés les réflexions méthodologiques et théoriques finalement produites ici.Contrairement à la recherche en informatique et algorithmique qui s'intéresse aux possibilités et capacités d'analyse, la pratique de l'analyse de données est conditionnée par un résultat extérieur au processus lui-même. L'ajustement des étapes du traitement et de la visualisation doit donc refléter sincèrement l'approche première de l'objet étudié. Ainsi, une connaissance aigüe du terrain doit se corréler d'une maitrise des outils et langages nécessaires à la réalisation de l'étude. Cette rencontre se produit dans la conjugaison de langages informatiques et humaines afin de formuler et d'écrire un travail juste. La disponibilité des données et du code en donne le texte intégral et permet une ré-utilisation de tous les éléments produits ici\footnote{Le code et les données sont disponibles depuis \url{http://github.com/topogram}, consulté le 4 Juillet 2014 à 16h15}. 

% limites et travaux futurs
Le présent travail s'est donné pour tâche d'établir les fondements méthodologiques et conceptuels qui permettent de mener à bien une étude plus systématique des formes de circulation des contenus sur Internet. Les relations entre régimes d'expression et topogrammes définies dans ce premier travail nécessitent d'être affinées et approfondies. Nous avons notamment fait attention à ne pas généraliser trop rapidement des analyses portant sur le terrain bien particulier de \textit{Sina Weibo}. Un projet de recherche post-doctorale financé par l'Agence National de la Recherche (ANR) autour des modèles d'innovation ouvertes en Chine
\footnote{
    Appel à projet \textit{Sociétés innovantes} (INOV) en 2013 : \textit{OPIMPUC - OPen Innovation : Models and Places in Urban China}. Voir l'annonce officielle \url{http://www.agence-nationale-recherche.fr/en/anr-funded-project/?tx_lwmsuivibilan_pi2\%5BCODE\%5D=ANR-13-SOIN-0006}, consulté le 5 Juillet à 17h12
} permettra de poursuivre les réflexions amorcées dans ce travail sur un autre terrain. Également, nous pourrons confronter les résultats observés par l'analyse de données sous forme de topogrammes aux expériences transcrites lors d'interviews ou de méthodes d'exploration plus qualitatives. 

Ce travail continuera de questionner les objectifs, limites et objets dont peuvent se saisir le vaste champ d'études pluri-disciplinaire ouverts par l'analyse de données et l'usage des langages informatiques en contexte scientifique.
