\chapter*{Conclusion}
% conclusion should be able to stand on its own and provide a justification and defence of the thesis
% Tie together, integrate, and synthesize the various issues raised in the discussion sections, while reflecting the introductory thesis statement(s) or objectives


% le sujet de la discussion
\newthought{La diversité} des objets numériques circulant le réseau social Sina Weibo nous a permis d'observer des modèles de diffusion fort différents

% indicate the importance of the subject discussed 
Les mèmes


% restate the research questions as presented in the introduction chapter 
créer un outil pour l'observation des actes de communication en ligne

% reinforces the importance of the study and its findings. 
L'étude de ces différents modèles nous a notamment permis de comprendre comment ils pouvaient s'articuler selon des régimes d'expression différents.

définir des préceptes méthodologiques et conceptuels permettant de mobiliser des théories issus de plusieurs champs scientifiques.


\subsubsection[Travaux et contributions de cette recherche]{Travaux et contributions de cette recherche}

% summarize what was said in the different chapters
Sina Weibo
Les mèmes
Le milieu numérique
topogramme

L'analyse d'un vaste ensemble de données nous a permis d'observer différents aspects intéressants de la diffusion des objets digitaux.

Un outil capable de procurer lectures multiples grâce à la visualisation sous forme de graphes sémantiques (co-occurence de mots), conversationnels (interactions entre les utilisateurs), temporels (évolution du volume de tweets sur une période donnée) et géographiques (origine des provinces des utilisateurs lors des conversations). 

La modélisation
L'extraction des graphes

% synthesis of empirical finding as answers to research questions

Hypothèses :

\textit{La majorité des contenus circulant sur les réseaux sociaux s'apparentent largement à ceux des médias traditionnels}

Les hashtags

Dans le cas de Sina Weibo, l’espace de discussion est structuré par un fournisseur historique de contenus de masse (Sina). A l’inverse, la vivacité des débats entourant les développements économiques, sociaux et politiques de la Chine moderne donnent lieu à de nombreuses tentatives d’associer la population chinoise à ces discussions au travers des médias. Le rôle des mèmes à caractère politique, marketing ou des discussions de société est ici important. Loin d’être une exception chinoise, le contrôle du caractère dissociatif et associatif des nouvelles formes langagières du Web est une problématique entourant les technologies de la parole qui traversent les médias du monde entier.


\textit{Les échanges en ligne sont à comprendre comme des actes d'énonciation, dont les mèmes sont l'expression la plus actuelle.}
\textit{L'analyse de la diffusion ne doit pas se contenter de l'analyse conversationnelle, mais doit mobiliser les modèles étudiant l'énonciation} 

L’industrie des contenus mass-media produisant souvent des contenus plus dissociatifs est très mature sur les réseaux sociaux. 

régimes multiples d'énonciation

outil

Le topogramme d’un fait médiatique considéré comme dissociatif se caractérise par une très forte agrégation et une très faible modularité (peu de groupes fortement dominés). Dans notre étude, l’exemple The Voice montre un réseau très structuré autour de peu de groupes. A l’inverse, la diffusion des mèmes absurdistes ou des sujets d’actualité est fragmentée en nombreux petits groupes très intégrés, et présente ainsi des qualités plus associatives en encourageant à la participation. Néanmoins, une structure trop associative conduit à l’asphyxie de ces groupes par la réduction de l’espace dans son réseau. D’après notre étude sur les hashtags, nous pouvons dire que les deux modèles coexistent largement sur les réseaux sociaux chinois. 



\textit{L'analyse et la visualisation de données permette d'établir une classification des discussions en ligne selon les structures de leurs diffusions}.

topogrammes

Gr\^ace à l{\textquoteright}outil de visualisation interactif développé spécialement pour cette analyse, il nous a donc été possible d{\textquoteright}observer les structures et relations mutuelles de ces graphes. Cette exploration nous a notamment permis de mettre à jour des caractéristiques spécifiques à certains mèmes parmi ceux sélectionnés.



\textit{La circulation des contenus sur les réseaux sociaux accroît la fragmentation du milieu numérique}.

Reste encore pas mal de travail mais l'analyse de données comme opportunité pour établir un champ commun pour des analyses pluri-disciplinaires.




% emphasis of the main original contributions



\subsubsection[Limites]{Limites} 
% discusses the issues
% à la lumière de cela, dire si ton travail valide ta thèse de départ (totalement, un peu, beaucoup, pas du tout) et pourquoi.
% Highlight the study limitations : A statement about the limitations of the work

L{\textquoteright}approche géographique a néanmoins donné assez peu de résultats, en proposant pas de structures nécessairement redondantes. Ainsi, la qualification d{\textquoteright}un milieu numérique par la classification de ces objets digitaux selon ses topogrammes semblent une t\^ache irréalisable dans cette étude. 


% A deduction made on the basis of the main body (i.e. Concluding statements)
% reaffirms the thesis statement, reaches a final judgment



\subsubsection[Travaux à poursuivre]{Travaux à poursuivre} 
% END (future direction and direction of further research). 

% theoretical and policy implications 
% the practical implications of your results
% future work, les travaux à poursuivre, Recommend direction and areas for future research

