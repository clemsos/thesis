\section{Résultats sur un échantillon de mèmes}
\label{sec:results-memes}

Après avoir présenté dans le détail le fonctionnement du dispositif d'étude des mèmes que nous avons créé, nous allons désormais présenter un ensemble de résultats obtenus par l'analyse d'un ensemble de mèmes choisis. Afin de procéder à la sélection de mèmes, nous avons donc choisi d{\textquoteright}utiliser la typologie des catégories de mèmes identifiées dans la littérature \ref{fig:typologie-memes} et d{\textquoteright}en systématiser l{\textquoteright}usage sur notre corpus. Ainsi, pour chaque catégorie, nous avons effectué une recherche m\^elant des sources parlant des mèmes et de l{\textquoteright}Internet chinois (journaux, blogs, encyclopédies en ligne, sites ressources), notre propre expérience du web chinois et notre corpus de données disponibles. 

\begin{table}[h!]
  \begin{tabulary}{\textwidth}{c|C}

    \textbf{Nom} &
    \textbf{Description}\\
    \hline \\[-1.5ex]
    \textit{S}\textit{extape}\textit{ } &
    Un scandale d{\textquoteright}adult\`ere concernant un homme politique
    dans la ville de Chongqing dans le centre de la Chine.\\[2ex]

    \textit{The Voice } &
    La premi\`ere saison d{\textquoteright}une \'emission de
    t\'el\'e-crochet musical en Chine.\\[2ex]

    \textit{Qiegao } &
    Un d\'ebat de soci\'et\'e sur la condition du peuple Uyghur suivant un
    fait divers autour d{\textquoteright}une rixe lors
    d{\textquoteright}une vente de g\^ateau.\\[2ex]

    \textit{Dufu} &
    Un m\`eme comique mettant en sc\`ene un po\`ete chinois dans des
    situations burlesques.\\[2ex]

    \textit{Biaoge} &
    Un scandale entourant la collection de montres de marques prestigieuses
    d{\textquoteright}un homme politique de la province du Shaanxi \\[2ex]

    \textit{Moyan} &
    L{\textquoteright}attribution du Prix Nobel de litt\'erature \`a
    l{\textquoteright}\'ecrivain chinois Mo Yan\\[2ex]

    \textit{Yuanfang} &
    La reprise d{\textquoteright}une citation d{\textquoteright}une s\'erie
    t\'el\'evis\'ee sous forme humoristique\\[2ex]

    \textit{Ccp} &
    D\'ebats entourant le 18\`eme Congr\`es du Parti Communiste Chinois\\[2ex]

  \end{tabulary}
  \caption{D\'enomination et description des m\`emes \'etudi\'es}
\end{table}

Pour constituer cette liste, nous avons dans un premier temps identifié pour chaque catégorie de mème plusieurs évènements web de l{\textquoteright}année 2012 sur Sina Weibo comme autant de candidats pour représenter l{\textquoteright}ensemble de notre typologie. La classification de l{\textquoteright}importance des évènements web rev\^et une nature très différente selon les différentes sources. Une revue de la littérature web sur les {\textquotedblleft}évènements marquant du web social en 2012{\textquotedblright} montre que les contenus à caractère politique et polémique sont considérés comme plus importants sur les sites à audience majoritairement occidentale\footnote{Global Voices,\url{http://globalvoicesonline.org/2012/12/07/top-10-chinese-internet-memes-of-2012/}, consulté le 22 Avril 2014 à 12:10} \footnote{ Wall Street Journal http://blogs.wsj.com/chinarealtime/2012/12/19/the-top-10-chinese-internet-memes-of-2012/,consulté le 22 Avril 2014 à 12:10 } alors que les sites plus spécialisés sur la Chine\footnote{ Danwei \url{http://www.danwei.com/chinas-hottest-styles-of-2012/} consulté le 22 Avril 2014 à 12:12} prennent davantage en considération les phénomènes médiatiques commerciaux. Encore une fois, nous voyons comment les réseaux sociaux chinois sont représentés comme des phénomènes politiques, parfois au détriment de leur existence comme média à part entière. 

Pour une meilleure commodit\'e de lecture, un nom a \'et\'e attribu\'e \`a chacun des m\`emes \'etudi\'es qui permettra de l{\textquoteright}\'evoquer facilement tout au long du chapitre qui suit. Une pr\'esentation plus d\'etaill\'ee des m\`emes identifiés est disponible en annexe de cette th\`ese (Figure \ref{fig:memelist}). Nous allons donc maintenant consid\'erer les diff\'erents aspects de chacun des m\`emes s\'electionn\'es auparavant afin de mieux comprendre comment diff\'erentes lectures peuvent nous informer sur les diff\'erents aspects de chacune de leur diffusion. 


\subsection[Structures temporelles]{Structures temporelles}
Dans un premier temps, nous avons choisi de consid\'erer les diff\'erentes structures temporelles des m\`emes. Les figures ci-dessous repr\'esentent chronologiquement le volume de messages \'echang\'es par heure dans chacun des jeux de donn\'ees. La dur\'ee totale des \'echantillons ci-dessous est de 3 semaines. La diminution r\'eguli\`ere de la quantit\'e de messages observ\'es dans les graphes correspond \`a la baisse d{\textquoteright}activit\'e pendant les nuits. 

D{\textquoteright}apr\`es ces figures, nous pouvons constater diff\'erentes formes repr\'esentatives : la {\textquotedblleft}breaking news{\textquotedblright} du m\`eme \textit{qiegao} se voit nettement par un d\'epart abrupt de la courbe (Fig. \ref{fig:time-qiegao}) \`a un niveau directement tr\`es \'elev\'e, suivi d{\textquoteright}une retomb\'ee rapide de l{\textquoteright}attention en quelques jours .  

La discussion plus informative autour du proc\`es de l{\textquoteright}homme politique de Chongqing et sa {\textquotedblleft}sextape{\textquotedblright} (Fig. \ref{fig:time-sextape}) croit doucement pour conna\^itre un pic rapide et tr\`es fort de discussion avant de retomber rapidement \'egalement.  

L{\textquoteright}\'emission de t\'el\'evision The Voice est, elle, absolument \'ev\`enementielle et son graphe montre bien le rythme des \'emissions accompagn\'ees de tr\`es peu de discussions entre les \'emissions (Fig. \ref{fig:time-voice}). N\'eanmoins, nous voyons que le volume est nettement plus important avec un pic de plus de 2000 tweets dans une m\^eme heure le soir de la premi\`ere \'emission, d\'epassant tr\`es largement les trois autres m\`emes pr\'esent\'es.  

Le graphe du m\`eme absurdiste {\textquotedblleft}dufu{\textquotedblright} connait une lente mont\'ee d{\textquoteright}attention puis un plateau de plusieurs jours. La blague semble durer et m\^eme conna\^itre un regain d{\textquoteright}int\'er\^et une semaine plus tard et \^etre r\'eguli\`erement mentionn\'e par la suite (Fig. \ref{fig:time-dufu}).

\begin{figure}[ht]
    \centering
    \subfloat[sextape]{
      \includegraphics[width=6.0087in,height=1.6697in]{figures/chap4/chapitre4-img1.png}
      \label{fig:time-sextape}
    }
    \newline
    \subfloat[dufu]{
      \includegraphics[width=6.0087in,height=1.6697in]{figures/chap4/chapitre4-img2.png}
      \label{fig:time-dufu}
    }
    \newline
    \subfloat[qiegao]{
      \includegraphics[width=6.0087in,height=1.6697in]{figures/chap4/chapitre4-img3.png}
      \label{fig:time-qiegao}
    }
    \newline
    \subfloat[The Voice]{
      \includegraphics[width=6.0087in,height=1.6697in]{figures/chap4/chapitre4-img4.png}
      \label{fig:time-voice}
    }
    \caption{
      Graphe temporel repr\'esentant le volume de messages \'echang\'es sur une dur\'ee de 3 semaines    
    }
\end{figure}
\clearpage
  
Proportionnellement, nous constatons que le m\`eme absurdiste est celui qui retient le plus l{\textquoteright}attention, le plus longtemps (Fig. \ref{fig:time-dufu}). A l{\textquoteright}inverse la campagne en ligne de The Voice n{\textquoteright}est qu{\textquoteright}un artefact de l{\textquoteright}\'emission (Fig. \ref{fig:time-voice}) mais draine une audience tr\`es importante pendant un temps restreint. 

\begin{figure}[h!]
    \centering
    
  \includegraphics[width=6.0087in,height=1.6697in]{figures/chap4/chapitre4-img5.png}
  
  \caption{
   Graphe temporel repr\'esentant le volume de messages \'echang\'es  autour du m\`eme \textit{yuanfang} entre le 7 Octobre et le 16 D\'ecembre 2012.
  }
  \label{fig:time-yuanfang}
\end{figure}

La tendance du m\`eme comique \`a durer se v\'erifie \'egalement avec
\textit{yuanfang} (Fig. \ref{fig:time-yuanfang}) qui continue d{\textquoteright}\^etre mentionn\'e
tr\`es r\'eguli\`erement sur une p\'eriode de plusieurs mois (ici du 7
Octobre au 16 D\'ecembre 2012).

\begin{figure}[h!]
    \centering
    \includegraphics[width=6.0087in,height=1.6697in]{figures/chap4/chapitre4-img6.png}
    \caption{
      Graphe temporel repr\'esentant le volume de messages \'echang\'es autour du m\`eme \textit{moyan} entre le 8 et le 21 Octobre 2012.
    }
    \label{fig:time-moyan}
\end{figure}

A l{\textquoteright}inverse, les informations type {\textquotedblleft}news{\textquotedblright} ont une dur\'ee de vie tr\`es courte, m\^eme lorsqu{\textquoteright}elles sont tr\`es d\'ebattues. Dans le cas de Moyan, on voit un tr\`es fort volume d{\textquoteright}activit\'e (3000 messages par heure au plus haut) mais une disparition quasi-totale des d\'ebats au bout en moins d{\textquoteright}une dizaine de jours (Fig. \ref{fig:time-moyan}).

\subsection[Structures conversationnelles]{Structures conversationnelles}

Les conversations sur Sina Weibo sont faites d{\textquoteright}\'echanges entre les utilisateurs sous la forme de commentaires, de promotion d{\textquoteright}un message ({\textquotedblleft}chuanfa{\textquotedblright} \'equivalent du {\textquotedblleft}retweet{\textquotedblright} de Twitter) ou de r\'eponse au message. Ces diff\'erentes interactions forment dans leur ensemble une large conversation o\`u les utilisateurs interagissent, se parlent, se r\'epondent, discutent en utilisant les diff\'erentes fonctionnalit\'es de la plateforme. En simplifiant ces \'echanges sous la forme de relation dirig\'ee entre deux utilisateurs (\textit{{\textquotedblleft}A interagit avec B{\textquotedblright}}), nous pouvons reconstituer le r\'eseau des interactions que nous visualisons ensuite sous la forme d{\textquoteright}un graphe. Dans ce réseau conversationnel, les individus sont symbolisés par des points au sein des communautés notés par différentes couleurs sélectionnées pour leur cohérence et leur visibilité en groupe \citep{Lin2013}. La taille des cercles indique leur centralité dans le réseau global (total des connections entrantes et sortantes). 


\begin{figure}[h!]
    \centering
    \subfloat[Sextape]{
      \includegraphics[width=3.0705in,height=2.7913in]{figures/chap4/chapitre4-img7.png}
      \label{fig:users-sextape}
    }
    \subfloat[The Voice]{
      \includegraphics[width=3.0705in,height=2.7913in]{figures/chap4/chapitre4-img8.png}
      \label{fig:users-voice}
    }
    \newline
    \subfloat[Qiegao]{
      \includegraphics[width=3.0705in,height=2.7913in]{figures/chap4/chapitre4-img9.png}
      \label{fig:users-qiegao}
    }
    \subfloat[Dufu]{
      \includegraphics[width=3.0748in,height=2.7913in]{figures/chap4/chapitre4-img10.png}
      \label{fig:users-dufu}
    }
    
  \caption{  
    Graphe conversationnel pour chacun des 4 m\`emes (pour une dur\'ee de 3 semaines)
  }
\end{figure}
% \clearpage

Les graphes de conversation permettent d{\textquoteright}identifier des sp\'ecificit\'es et de lire plusieurs indicateurs. Ils rendent notamment bien compte de la taille des conversations, tr\`es variables, ou plut\^ot de la dimension des interactions. On voit que les discussions autour d{\textquoteright}aspects politiques et soci\'etaux connaissent beaucoup plus d{\textquoteright}\'echanges, alors que les deux autres semblent moins importantes. Dans le cas de \textit{The Voice} (Fig. \ref{fig:users-voice}), la conversation est tr\`es structur\'ee autour de peu d{\textquoteright}acteurs qui centralisent les d\'ebats. Le m\`eme absurdiste \textit{dufu} (fig. \ref{fig:users-dufu}) semble \^etre compos\'e d{\textquoteright}utilisateurs tr\`es distants qui ne communiquent entre eux que faiblement ou par petits groupes. La conversation \textit{sextape }autour du fait divers politique se constitue en groupes structur\'es mais plus distants, repouss\'es vers les bords du graphe, montrant l{\textquoteright}existence de nombreux foyers de discussions. A l{\textquoteright}inverse, la discussion de soci\'et\'e \textit{qiegao}\textit{ }autour des peuples du Xinjiang (fig. \ref{fig:users-qiegao}) semble constitu\'e autour d{\textquoteright}un vaste foyer de discussions o\`u viennent s{\textquoteright}agr\'eger de nombreux utilisateurs tr\`es actifs. 

\subsection[Structures s\'emantiques]{Structures s\'emantiques}

La dimension lexicale des contenus web pr\'esente une autre approche int\'eressante \`a explorer pour mieux comprendre comment se structure l{\textquoteright}activit\'e lors de la diffusion des m\`emes. La repr\'esentation classique du nuage de mots s{\textquoteright}int\'eresse principalement \`a la quantit\'e de mots en laissant toutefois de cot\'e l{\textquoteright}aspect relationnel de l{\textquoteright}existence des mots qui est pourtant le fondamental de l{\textquoteright}activit\'e symbolique et expressive. Ainsi, nous avons plut\^ot choisi de nous int\'eresser aux r\'eseaux de mots afin de comprendre comment l{\textquoteright}activit\'e en ligne se construit autour de signifi\'es particuliers. 

Les relations entre les mots représentant les co-occurences dans un même message. La proximité des mots lors des discussions est matérialisée par un graphe disposant les mots par groupe grâce à un algorithme de force. La taille des mots indique le nombre fois qu'ils ont été cités dans les conversations. Les couleurs représentent les communautés détectées par l'algorithme de Louvain \cite{Blondel2008} (voir section \ref{sec:viz} pour plus de détails). Un des biais importants \`a prendre en compte dans la lecture de ces graphes est le fait que les corpus desquels ils sont extraits ont \'et\'e constitu\'es gr\^ace \`a une recherche par mots-cl\'es dans un moteur d{\textquoteright}indexation. Ainsi, les mots les plus pro\'eminents sont ceux qui sont \`a l{\textquoteright}origine de la constitution du corpus. 

\begin{figure}[h!]
    \centering
    \subfloat[Dufu]{
      \includegraphics[width=3.1661in,height=1.7594in]{figures/chap4/chapitre4-img11.png}
      \label{fig:words-dufu}
    }
    \subfloat[Yuanfang]{
      \includegraphics[width=3.1835in,height=1.7697in]{figures/chap4/chapitre4-img12.png}
      \label{fig:words-yuanfang}
    }
    
  \caption{
    Graphe s\'emantique de co-occurences des mots pour les m\`emes dufu et yuanfang   
  }
\end{figure}

En regardant ces graphes, nous constatons tout d{\textquoteright}abord que les graphe des m\`emes absurdistes (fig. \ref{fig:words-dufu} \& \ref{fig:words-yuanfang})se compose de peu de mots centraux tr\`es fortement connect\'es et entour\'e d{\textquoteright}une myriade de mots de faible densit\'e, correspondant s\^urement aux myriades de d\'eclinaisons qui font le propre de la blague en ligne.  

\begin{figure}[h!]
    \centering
    \subfloat[Sextape]{
      \includegraphics[width=3.1461in,height=1.7504in]{figures/chap4/chapitre4-img13.png}
      \label{fig:words-sextape}
    }
    \subfloat[biaoge]{
      \includegraphics[width=3.2335in,height=1.7961in]{figures/chap4/chapitre4-img14.png}
      \label{fig:words-biaoge}
    }
    
  \caption{
    Graphe s\'emantique de co-occurences des mots pour les m\`emes Sextape et Biaoge
  }
\end{figure}


Les r\'eseaux s\'emantiques de la discussion concernant les faits divers politiques (fig. \ref{fig:words-sextape} \& \ref{fig:words-biaoge}) sont plus structur\'es. Ils sont organis\'es en communaut\'es de mots et de sens plut\^ot bien d\'efinies : une communaut\'e d\'ecrit les d\'etails de l{\textquoteright}affaire (noms de lieux et de personnes, verbes d{\textquoteright}actions), l{\textquoteright}autre se compose d{\textquoteright}adjectifs discutant le caract\`ere houleux de ces histoires, la derni\`ere est faite de mots plus anecdotiques.

\begin{figure}[h!]
    \centering
    \subfloat[ccp]{
      \includegraphics[width=3.0843in,height=1.7134in]{figures/chap4/chapitre4-img15.png}
      \label{fig:words-ccp}
    }
    \subfloat[The Voice]{
      \includegraphics[width=3.0752in,height=1.7122in]{figures/chap4/chapitre4-img16.png}
      \label{fig:words-voice}
    }
    
  \caption{
    Graphe s\'emantique de co-occurences des mots pour les m\`emes CCP et The Voice
  }
\end{figure}

 La structure lexicale qui entoure \textit{The Voice }et \textit{CCP }semble tr\`es architectur\'ee, avec des associations plus convenues. Pour \textit{The Voice} (fig. \ref{fig:words-voice}), nous sommes sans surprise en pr\'esence du champ lexical du chant (voix-chant, chant-chanson...) entour\'e par de nombreuses \'emotions et sentiments (arc-en-ciel-amour, trac-courage, etc.). Nous voyons \'egalement que de tr\`es nombreux mots ont \'et\'e propuls\'es sur les bords du graphe car peu reli\'es entre eux ou avec le reste du r\'eseau. Ils sont d{\textquoteright}ailleurs reconnus algorithmiquement comme une seule communaut\'e. Il int\'eressant de constater qu{\textquoteright}on y trouve notamment le vocabulaire propre de l{\textquoteright}\'emission, qui apparemment ne s{\textquoteright}int\`egre pas aux discussions. 

Dans le cas du congr\`es du parti communiste chinois (fig. \ref{fig:words-ccp}), on retrouve \'egalement cette organisation s\'emantique tr\`es hi\'erarchique autour du mot central {\textquotedblleft}parti{\textquotedblright}, avec un ensemble de signifiants issus du programme politique (construction, d\'eveloppement, loi, science, etc.). Un autre \'el\'ement pr\'edominent s{\textquoteright}articule autour du caract\`ere {\textquotedblleft}nouveau{\textquotedblright} (r\'evolution, changement, futur...). Comme dans le cas de \textit{The Voice}, nous sommes en pr\'esence d{\textquoteright}un langage st\'er\'eotyp\'e qui donne lieu \`a un graphe tr\`es structur\'e selon des axes de communication d\'efinis. Proportionnellement, une grande partie des discussions est situ\'e aux fronti\`eres du graphe, tr\`es peu reli\'ee avec le reste de l{\textquoteright}activit\'e.

\begin{figure}[h!]
    \centering
    \subfloat[qiegao]{
      \includegraphics[width=2.9248in,height=1.6252in]{figures/chap4/chapitre4-img17.png}
      \label{fig:words-qiegao}
    }
    \subfloat[moyan]{
      \includegraphics[width=2.978in,height=1.6551in]{figures/chap4/chapitre4-img18}.png
      \label{fig:words-moyan} 
    }
    
  \caption{
    Graphe s\'emantique de co-occurences des mots pour les m\`emes Qiegao et Moyan
  }
\end{figure}


\textit{Qiegao} (fig. \ref{fig:words-qiegao}), quant \`a lui, poss\`ede une structure tr\`es centralis\'ee articul\'ee autour d{\textquoteright}un mot-cl\'e pr\'ecis qui relie plusieurs groupes s\'emantiques (communaut\'es de mots par couleurs) autour de lui. On retrouve cette caract\'eristique d{\textquoteright}ensemble dans le graphe de \textit{moyan} (fig. \ref{fig:words-moyan}) m\^eme si la structuration des communaut\'es s\'emantiques est plus accentu\'ee - on peut penser que cela est d\^u aux plus grands volumes de messages \'echang\'es.

\subsection[Dynamiques g\'eographiques des discussions entre provinces]{Dynamiques g\'eographiques des discussions entre provinces}

Nous avons propos\'e dans cette recherche le terme de milieu num\'erique afin notamment de probl\'ematiser les rapports existants entre les pratiques de la discussion en ligne sur Sina Weibo et l{\textquotesingle}espace de la Chine urbaine moderne. Afin de poursuivre notre interrogation, nous allons d\'esormais observer comment les topologies de diffusion en ligne peuvent se corr\'eler avec la diffusion g\'eographique. La diffusion d{\textquotesingle}un m\`eme suit-elle la hi\'erarchie urbaine classique ou bien la g\'eographie urbaine de l{\textquotesingle}Internet? Existe-t-il des mod\`eles de diffusion selon les types de contenus? Quels sont les \textit{patterns }g\'eographiques d\'ecrits pas la circulation des contenus? 

Pour tenter de r\'epondre \`a ces diff\'erentes questions, nous disposons de donn\'ees concernant l{\textquoteright}origine des utilisateurs, d{\textquoteright}apr\`es les provinces dans leur profil Sina Weibo. En mettant en relations ces informations de profils avec les diff\'erentes dimensions d\'efinies pr\'ec\'edemment, nous allons proc\'eder \`a l{\textquoteright}analyse de la diffusion g\'eographique des m\`emes. Le réseau des conversations projetée sur les cartes présentent les échanges entre les utilisateurs par province, pondérée par la population totale de l'échantillon. L{\textquoteright}\'epaisseur des traits exprime le volume d{\textquoteright}\'echanges en pourcentage du total sur la p\'eriode observ\'ee.

\begin{figure}[h!]
  \begin{minipage}[c]{.45\linewidth}
    \centering
    \includegraphics[scale=.3]{figures/chap4/chapitre4-img19.png}
    \caption{
      The Voice: Interaction des utilisateurs par province entre le 9 et le 29 Juillet 2012
    }
    \label{fig:geo-voice-t1}
  \end{minipage}
  \begin{minipage}[c]{.45\linewidth}
    \centering
    \includegraphics[scale=.3]{figures/chap4/chapitre4-img20.png}    
    \caption{
      The Voice: Interaction des utilisateurs par province entre le 16 et le 17 Juillet 2012
    }
    \label{fig:geo-voice-t2}
  \end{minipage}
\end{figure}

Dans le cas de \textit{The Voice} (fig. \ref{fig:geo-voice-t1}), nous pouvons voir que l{\textquoteright}interaction est structur\'e autour d{\textquoteright}un axe fort entre la province de Zhejiang et P\'ekin. Cela s{\textquoteright}explique assez simplement par le fait que l{\textquoteright}\'emission est diffus\'e par Zhejiang T\'el\'evision dont le compte officiel joue un r\^ole important dans la diffusion et la promotion de la discussion. En isolant uniquement le soir de la premi\`ere diffusion (fig. \ref{fig:geo-voice-t2}), on voit que cette structure s{\textquoteright}accentue encore davantage et que les interactions se structurent entre le Zhejiang P\'ekin et Shanghai avec quelques autres axes plus minoritaires.

\begin{figure}[h!]
    \centering
    \includegraphics[scale=.35]{figures/chap4/chapitre4-img21.png}
    \caption{
      Qiegao: Interaction des utilisateurs par province entre le 19 Novembre et le 15 D\'ecembre 2012
    }
    \label{fig:geo-qiegao-t0}
\end{figure}

A l{\textquoteright}inverse, nous voyons que pour \textit{Qiegao} (fig. \ref{fig:geo-qiegao-t0}) les patterns semblent beaucoup plus diversifi\'es. Guangzhou joue un r\^ole de diffuseur important et beaucoup d{\textquoteright}informations converge vers Shanghai et P\'ekin. De nombreuses conversations se d\'eroulent \'egalement dans l{\textquoteright}ouest de la Chine. En effet, la discussion concerne ici les peuples Uyghur et on constate que les habitants du Xinjiang participent davantage \`a la conversation que dans d{\textquoteright}autres m\`emes.  




\begin{figure}[h!]
    \centering
    \subfloat[J1: 3 D\'ecembre 2012]{
      \includegraphics[width=2in,height=1.61in]{figures/chap4/chapitre4-img22.png}
      \label{fig:geo-qiegao-t1}
    }
    \subfloat[J2-3: 4-5 D\'ecembre 2012]{
      \includegraphics[width=2in,height=1.61in]{figures/chap4/chapitre4-img23.png} 
      \label{fig:geo-qiegao-t2}   
    }
    \newline
    \subfloat[J4-6: 6-8 D\'ecembre 2012]{
      \includegraphics[width=2in,height=1.61in]{figures/chap4/chapitre4-img24.png}
      \label{fig:geo-qiegao-t3}
    }
    \subfloat[J7+: 9-16 D\'ecembre 2012]{
      \includegraphics[width=2.8724in,height=1.7961in]{figures/chap4/chapitre4-img25.png}  
      \label{fig:geo-qiegao-t4}
  
    }
    
  \caption{
    Qiegao: Interaction des utilisateurs par province (Jours)
  }
\end{figure}

En consid\'erant l{\textquoteright}\'evolution du m\`eme dans le temps,
nous voyons qu{\textquoteright}il \'emane d{\textquoteright}abord de
Canton (fig. \ref{fig:geo-qiegao-t1}), puis se diffuse \`a P\'ekin et Shanghai (fig. \ref{fig:geo-qiegao-t2}) qui semble jouer un r\^ole d{\textquoteright}amplificateur en le diffusant plus largement (fig. \ref{fig:geo-qiegao-t4}), notamment vers l{\textquoteright}Ouest (fig. \ref{fig:geo-qiegao-t3}). 


\begin{figure}[h!]
    \centering
    
    \subfloat[J1: 22 Novembre 2012]{
      \includegraphics[width=1.9587in,height=1.2248in]{figures/chap4/chapitre4-img26.png}
      \label{fig:geo-sextape-t1}
    }
    \subfloat[J2-5: 23-25 Novembre 2012]{
      \includegraphics[width=1.9386in,height=1.2114in]{figures/chap4/chapitre4-img27.png}
      \label{fig:geo-sextape-t2}
    }
    \subfloat[J5+: 26 Nov-15 D\'ec 2012]{
      \includegraphics[width=2.0012in,height=1.2512in]{figures/chap4/chapitre4-img28.png}
      \label{fig:geo-sextape-t3}
    }    
  \caption{
    Sextape: Interaction des utilisateurs par province (Jours)
  }
\end{figure}

Avec une plus grande libert\'e de ton et une plus grande latitude dans leur propos, les m\'edias cantonais sont bien souvent instigateurs d{\textquoteright}affaires importantes alors que les m\'edias p\'ekinois, plus conservateurs joue plut\^ot un r\^ole de diffuseur. 

\begin{figure}[h!]
    \subfloat[J1-3 : 26-28 Aout 2012 ]{
      \includegraphics[width=1.9004in,height=1.1878in]{figures/chap4/chapitre4-img29.png}
      \label{fig:geo-biaoge-t1}
    }
    \subfloat[J4-5 : 29-30 Aout 2012 ]{
      \includegraphics[width=1.9004in,height=1.1878in]{figures/chap4/chapitre4-img30.png} 
      \label{fig:geo-biaoge-t2}
    }
    \subfloat[J5+ : 1-3 Septembre2012]{
      \includegraphics[width=1.9004in,height=1.1878in]{figures/chap4/chapitre4-img31.png} 
      \label{fig:geo-biaoge-t3}
    }
  \caption{
    Biaoge: Interaction des utilisateurs par province
  }
\end{figure}

On retrouve \'egalement ce pattern pour d{\textquoteright}autres faits de soci\'et\'e comme \textit{sextape} ou \textit{biaoge}. Cela illustre bien le r\^ole particulier des m\'edias du sud de la Chine et plus sp\'ecialement celui de Guangzhou (fig. \ref{fig:geo-sextape-t1} \& \ref{fig:geo-biaoge-t1}). 

\begin{figure}[h!]
    \subfloat[ Qiegao]{
      \includegraphics[width=1.9016in,height=1.9016in]{figures/chap4/chapitre4-img33.png}
    }
    \subfloat[Sextape]{
      \includegraphics[width=1.9016in,height=1.9016in]{figures/chap4/chapitre4-img34.png}
    }
    \subfloat[Biaoge]{
      \includegraphics[width=1.9016in,height=1.9016in]{figures/chap4/chapitre4-img35.png}
    }
  \caption{
    R\'epartition des utilisateurs par province (en \% du total)
    \label{fig:users-share}
  }
\end{figure}


En observant la r\'epartition des utilisateurs par province pour chacun de ces trois m\`emes (fig. \ref{fig:users-share}), on remarque que Beijing, Canton, Shanghai ont une place importante, ainsi que les personnes situ\'ees \`a l{\textquoteright}\'etranger (Haiwai) et le Zhejiang. N\'eanmoins, le reste des provinces (repr\'esent\'es ici par {\textquotedblleft}Others{\textquotedblright} en gris) constitue une part importante des \'echanges.

% \begin{figure}
%   \includegraphics[width=0.0012in,height=0.0012in]{figures/chap4/chapitre4-img32.png}
% \end{figure}

\begin{figure}[h!]
    \centering
    \subfloat[J1: 12 Octobre 2012]{
      \label{fig:geo-moyan-t1}
      \includegraphics[width=1.9004in,height=1.1878in]{figures/chap4/chapitre4-img36.png}
    }
    \subfloat[J2: 13 Octobre 2012]{
      \label{fig:geo-moyan-t2}
      \includegraphics[width=1.9004in,height=1.1878in]{figures/chap4/chapitre4-img37.png}
    }
    \subfloat[J3+: 14-21 Octobre 2012]{
      \label{fig:geo-moyan-t3}
      \includegraphics[width=1.9004in,height=1.1878in]{figures/chap4/chapitre4-img38.png}
    }
    \caption{Moyan : Interaction des utilisateurs par province   }
\end{figure}

Dans le cas de nouvelles plus officielles comme pour \textit{Moyan }notamment, nous pouvons voir que l{\textquoteright}information s{\textquoteright}origine g\'en\'eralement de P\'ekin (fig. \ref{fig:geo-moyan-t1}) et se diffuse vers le sud et le centre (fig. \ref{fig:geo-moyan-t2}). 

\begin{figure}[h!]
    \centering
    \includegraphics[width=2.6043in,height=2.6043in]{figures/chap4/chapitre4-img39.png}
  \caption{
    R\'epartition des utilisateurs par province pour Moyan (en \% du total)\\
  }
  \label{fig:geo-moyan-pie}
\end{figure}

La composition des utilisateurs (fig. \ref{fig:geo-moyan-pie}) montre bien que les d\'ebats autour de Moyan sont tr\`es largement domin\'es par P\'ekin avec plus de 50\% de l{\textquoteright}activit\'e impliquant au moins un utilisateur identifi\'e \`a P\'ekin.

Dans le cas des m\`emes absurdistes et comiques, il est int\'eressant de constater qu{\textquoteright}ils ne se construisent pas n\'ecessairement sur les patterns que nous avons pu observer auparavant.

\begin{figure}[h!]
    \centering
    \subfloat[dufu]{
      \includegraphics[width=2.9252in,height=1.828in]{figures/chap4/chapitre4-img40.png}
      \label{geo-dufu}
    }
    \subfloat[yuanfang]{ 
      \includegraphics[width=2.9252in,height=1.828in]{figures/chap4/chapitre4-img41.png}
      \label{geo-yuanfang}
    }
    \caption{
        Interaction des utilisateurs par province
    }
\end{figure}


On voit notamment que le m\`eme \textit{dufu } (fig. \ref{geo-dufu}) d\'ebute dans la r\'egion
du Hubei alors que le Liaoning joue un r\^ole cl\'e dans la diffusion de \textit{yuanfang} (fig. \ref{geo-yuanfang}). 

\begin{figure}[h!]
    \centering
    \subfloat[(Dufu) J1-2 : 23 au 25 Mars 2012 ]{
      \includegraphics[width=2.9252in,height=1.828in]{figures/chap4/chapitre4-img42.png}
      \label{geo-dufu-t1}
    }
    \subfloat[(Dufu) J3+ : 25 Mars au 15 Avril]{
      \includegraphics[width=2.9713in,height=1.8571in]{figures/chap4/chapitre4-img43.png}
      \label{geo-dufu-t2}
    }
    \newline
    \subfloat[(Yuanfang) 1-2: 15 au 17 Octobre 2012]{
      \includegraphics[width=2.9252in,height=1.828in]{figures/chap4/chapitre4-img44.png}
      \label{geo-yuanfang-t1}
    }
    \subfloat[(Yuanfang) J3+: 18 Oct au 16 D\'ecembre 2012]{
      \includegraphics[width=2.9252in,height=1.828in]{figures/chap4/chapitre4-img45.png}
      \label{geo-yuanfang-t2}
    }
    
    \caption{
        Evolution des interactions des utilisateurs par province pour Dufu et Yuanfang
    }
\end{figure}


En consid\'erant les graphes de temps, on remarque \'egalement que si P\'ekin est bien pr\'esent, les \'echanges des premiers jours se font principalement avec le Hubei (fig. \ref{geo-dufu-t1}) et le Yunnan (fig. \ref{geo-yuanfang-t1}), deux provinces typiquement peu actives. Dans un second temps, l'activité se concentre sur les régions cotières (fig. \ref{geo-yuanfang-t2}).

En s{\textquoteright}int\'eressant aux \'echanges qui se d\'eroulent au sein de chaque province (d{\textquoteright}un utilisateur situ\'e dans une province vers un autre utilisateur situ\'e dans la m\^eme province) (fig. \ref{geo-same-yuanfang} \& \ref{geo-same-dufu}),nous constatons que les \'echanges internes sont \'egalement importants au sein des villes principales (Canton, P\'ekin et Shanghai).

\begin{figure}[h!]
    \centering
    \subfloat[dufu]{
      \includegraphics[width=2.9252in,height=1.828in]{figures/chap4/chapitre4-img46.png}
      \label{geo-same-dufu}
    }
    \subfloat[yuanfang]{
      \includegraphics[width=2.9252in,height=1.828in]{figures/chap4/chapitre4-img47.png}
      \label{geo-same-yuanfang}
    }
    \caption{
        Interaction entre utilisateurs de la m\^eme province
    }
\end{figure}

Néanmoins, de nombreuses villes prennent aussi part \`a la discussion (fig. \ref{geo-pie-yuanfang} \& \ref{geo-pie-dufu}). 


\begin{figure}[h!]
    \centering
    \subfloat[dufu]{
      \includegraphics[scale=0.5]{figures/chap4/chapitre4-img48.png}
      \label{geo-pie-dufu}
    }
    \subfloat[yuanfang]{
      \includegraphics[scale=0.5]{figures/chap4/chapitre4-img49.png}
      \label{geo-pie-yuanfang}
    }
    \caption{
        R\'epartition des utilisateurs par province (en \% du total)
    }

\end{figure}


\subsection{Communaut\'es, clusters et groupes de provinces}

Un r\'ecent article du d\'epartement d{\textquoteright}urbanisme de l{\textquoteright}Universit\'e de Nanjing \citep{Zhen2013} analyse les relations entre le r\'eseau urbain et le r\'eseau qu{\textquoteright}il est possible d{\textquoteright}inf\'erer au sein des relations friends/followers entre les utilisateurs sur Sina Weibo . Il y est notamment montr\'e que le r\'eseau des relations entre utilisateurs vient renforcer la hi\'erarchie classique du syst\`eme urbain en acc\'el\'erant l{\textquoteright}agglom\'eration selon des mod\`eles spatiaux pr\'e-existants comme celui le pattern des \textit{{\textquotedblleft}Three majors }\textit{and four smalls{\textquotedblright} }montrant aussi une diff\'erence tr\`es marqu\'e entre l{\textquoteright}Ouest et l{\textquoteright}Est. 

\begin{figure}[h!]
    \centering
    
    \begin{description}
    \item[Three Majors]
      \begin{enumerate}
      \item Pearl River Delta (Guangzhou, Shenzhen)
      \item Beijing-Tianjin-Hebei region (Beijing)
      \item the Yangtze River Delta (Shanghai, Hangzhou, Nanjing)
      \end{enumerate}

    \item[Four Smalls]
      \begin{enumerate}
      \item Chengdu-Chongqing region (Chengdu, Chongqing)
      \item Hercynian region (Fuzhou, Xiamen)
      \item Wuhan (central) region (Wuhan, Changsha)
      \item Northeast China (Shenyang, Harbin, Changchun)
      \end{enumerate}

    \end{description}

   \caption{
      Le mod\`ele urbain chinois du Three majors and four smalls d{\textquoteright}apr\`es \cite{Zhen2013}
    }
\end{figure}

Nous avons donc d\'ecid\'e de comparer ces r\'esultats obtenus depuis le r\'eseau des followers sur Sina Weibo avec les ph\'enom\`enes observ\'es lors de la diffusion de contenus. En effet, si on peut affirmer que le r\'eseau {\textquotedblleft}d{\textquoteright}amis{\textquotedblright} structure les canaux de diffusion de Sina Weibo, ces canaux ne sont pas n\'ecessairement sujets \`a une utilisation fr\'equente et donc \`a l{\textquoteright}actualisation de ces structures. 

Afin d{\textquoteright}observer les patterns form\'es par la diffusion des m\`emes, nous continuons d{\textquoteright}utiliser le r\'eseau des interactions entre provinces des utilisateurs. Afin de d\'etecter les communaut\'es de discussion et identifier les provinces ayant interagit le plus ensemble dans chaque cas, nous soumettons le r\'eseau d{\textquoteright}interactions \`a un algorithme de Louvain \citep{Blondel2008}. Sur les cartes, les couleurs repr\'esentent les diff\'erentes communaut\'es calcul\'ees depuis le r\'eseau d{\textquoteright}interactions. 

\begin{figure}[h!]
    \centering

    \subfloat[sextape]{
      \includegraphics[width=3.198in,height=2.0004in]{figures/chap4/chapitre4-img50.png}
      \label{geo-clusters-sextape}
    }
    \subfloat[TheVoice]{
      \includegraphics[width=3.3024in,height=2.063in]{figures/chap4/chapitre4-img51.png}
      \label{geo-clusters-voice}
    } 
    \newline
    \subfloat[Qiegao]{
      \includegraphics[width=3.1878in,height=1.9795in]{figures/chap4/chapitre4-img52.png}
      \label{geo-clusters-qiegao}
    } 
    \subfloat[Dufu]{
      \includegraphics[width=3.5315in,height=2.2087in]{figures/chap4/chapitre4-img53.png}
      \label{geo-clusters-dufu}
    }

    \caption{
        Communaut\'e de provinces dessin\'ees par les \'echanges entre utilisateurs autour de chaque m\`eme 
    }

\end{figure}


\'A la premi\`ere lecture de ces cartes, nous constatons en effet que les provinces de l{\textquoteright}Ouest sont nettement moins engag\'ees que celles de la moiti\'e Est de la Chine. Ce fait refl\`ete que la population des utilisateurs de Weibo est concentr\'ee en majorit\'e dans les grandes villes en d\'eveloppement de la c\^ote et du centre \cite{Fu2013}.  

Le m\`eme absurdiste \textit{dufu} poss\`ede une diffusion plus concentr\'ee (fig. \ref{geo-clusters-dufu}) alors que les discussions politiques semblent regrouper des utilisateurs d{\textquoteright}origines plus diverses (fig. \ref{geo-clusters-sextape} \& \ref{geo-clusters-qiegao}). Nous remarquons \'egalement que Taiwan est absente des discussions plus en lien avec la politique et la soci\'et\'e de Chine continentale, alors qu{\textquoteright}elle est bien pr\'esente dans le cas des m\`emes absurdistes et de l{\textquoteright}\'emission de loisir (fig. \ref{geo-clusters-voice}).  

Devant la diversit\'e de contenus propos\'es par ces cartes, peu de pattern sont d\'ecelables a priori et la circulation des contenus ne semble pas suivre les patterns observ\'es dans les relations friends / followers. 

Ces premi\`eres observations nous montre qu{\textquoteright}il est possible de mieux comprendre les logiques et dynamiques de diffusion des contenus, mais il est n\'eanmoins plus p\'erilleux de consid\'erer les usages du r\'eseau pour en tirer des conclusions sur les dynamiques g\'eographiques d{\textquoteright}ensemble d{\textquoteright}un territoire. De plus, nous ne disposons que de tr\`es peu d{\textquotesingle}aspects {\textquotedbl}g\'eo-localis\'es{\textquotedbl} dans ces donn\'ees (pas de g\'eotag sur chaque message \`a proprement parler) et nous savons que la province d{\textquoteright}origine du profil ne peut \^etre consid\'er\'e comme une source absolument fiable. Les utilisateurs sont en effet libres de la remplir selon leur bon vouloir. De plus, cette information est \'ecrite une fois pour toute et n{\textquoteright}est pas n\'ecessairement mise \`a jour lors des d\'eplacements ou m\^eme d\'em\'enagements des utilisateurs. 

\subsection{Dimensions g\'eographiques des graphes socio-s\'emantiques}

Face \`a ces diverses r\'eserves, nous voyons qu{\textquoteright}il est important de recentrer l{\textquoteright}\'etude sur les modes de diffusion et leurs dimensions socio-s\'emantiques. Afin de continuer notre exploration des dynamiques conversationnelles entourant les \'echanges en ligne, nous allons donc nous int\'eresser au croisement de ces diff\'erentes dimensions en consid\'erant les corr\'elations entre les mots, les utilisateurs et les provinces \`a diff\'erents niveaux. 

Tout d{\textquoteright}abord, nous nous proposons d{\textquoteright}observer la distribution des provinces d{\textquoteright}origine des communaut\'es d{\textquoteright}utilisateurs les plus importantes. En s\'electionnant les communaut\'es les plus centrales pour chaque graphe, nous pouvons mieux comprendre comment se r\'epartissent les {\textquotedblleft}influenceurs{\textquotedblright} sur le territoire pour chacun des m\`emes choisis ici.

\begin{figure}[h!]
    \centering
    \includegraphics[width=5.9996in,height=2.5004in]{figures/chap4/chapitre4-img54.png}
    \caption{
        Communaut\'e de provinces dessin\'ees par les \'echanges entre utilisateurs autour de Sextape
    }
    \label{fig:sextape-users-pie}
\end{figure}

Les communaut\'es form\'ees lors de la discussion autour des m\`emes \`a teneur plus politique pr\'esentent une grande diversit\'e de provenances. On voit dans le cas de \textit{biaoge} (fig. \ref{fig:biaoge-users-pie}) et \textit{sextape} (fig. \ref{fig:sextape-users-pie}) que de nombreuses provinces sont repr\'esent\'ees.

\begin{figure}[h!]
    \centering     
    \includegraphics[width=5.9996in,height=2.5004in]{figures/chap4/chapitre4-img55.png}
    \caption{
        Communaut\'e de provinces dessin\'ees par les \'echanges entre utilisateurs autour de Biaoge
    }
    \label{fig:biaoge-users-pie}
\end{figure}

\begin{figure}[h!]
    \centering
    \includegraphics[width=5.9996in,height=2.5004in]{figures/chap4/chapitre4-img56.png}
    \caption{
        Communaut\'e de provinces dessin\'ees par les \'echanges entre utilisateurs autour de The Voice
    }
    \label{fig:voice-users-pie}
\end{figure}

\'A l{\textquoteright}inverse, les communaut\'es d{\textquoteright}utilisateurs les plus impliqu\'ees dans \textit{The Voice } (fig. \ref{fig:voice-users-pie}) ou \textit{Moyan} (fig. \ref{fig:voice-users-pie}) sont pour la plupart localis\'ees \`a Beijing. Par rapport \`a la cartographie pr\'ec\'edente qui montrait les relations fortes avec le Zhejiang pour \textit{The Voice} (fig. \ref{fig:voice-users-pie}), nous voyons ici que la province est tr\`es faiblement repr\'esent\'ee dans les communaut\'es majoritaires. 

\begin{figure}[h!]
  \centering
   \includegraphics[width=5.9996in,height=2.5004in]{figures/chap4/chapitre4-img57.png}
    \caption{
        Communaut\'e de provinces dessin\'ees par les \'echanges entre utilisateurs autour de Moyan
    }
    \label{fig:moyan-users-pie}
\end{figure}

Le m\`eme \textit{dufu} (fig. \ref{fig:dufu-users-pie}) montre quant \`a lui un pattern diff\'erent o\`u les communaut\'es sont plus organis\'ees plus r\'egionalement, avec moins de diversit\'e d{\textquoteright}origine des utilisateurs engag\'es dans les discussions.

\begin{figure}
    \centering
    \includegraphics[width=5.9996in,height=2.5004in]{figures/chap4/chapitre4-img58.png}
    \caption{
        Communaut\'e de provinces dessin\'ees par les \'echanges entre utilisateurs autour de Dufu
    }
    \label{fig:dufu-users-pie}
\end{figure}


Une dimension importante du m\`eme absurdiste semble \^etre l{\textquoteright}organisation des conversations en communaut\'es plut\^ot local (utilisateurs de m\^eme province). Une premi\`ere analyse des communaut\'es les plus centrales dans le graphe de \textit{yuanfang} ne nous permet de retrouver ce pattern. Pourtant, en nous int\'eressant aux communaut\'es de seconde et troisi\`eme importance, nous voyons que la plupart se constituent autour de 1 ou 2 provinces seulement.  

Pour continuer notre exploration de l{\textquoteright}existence g\'eographique des relations socio-s\'emantiques dans nos m\`emes, nous allons maintenant nous int\'eresser \`a une seconde dimension qui est celle de la distribution des mots par province. Nous s\'electionnons pour chaque m\`eme les mots les plus centraux du graphe s\'emantique et d\'ecomposons leur usage afin de comprendre la diversit\'e ou l{\textquoteright}unicit\'e de l{\textquoteright}origine des utilisateurs. 

\begin{figure}[h!]
    \centering
    \includegraphics[width=6.0087in,height=3.3386in]{figures/chap4/chapitre4-img59.png}
    \caption{
      biaoge : distribution des citations par provinces pour le mot wan signifiant 10000, une quantit\'e ici quantit\'e d{\textquoteright}argent pour une montre.
    }
    \label{fig:biaoge-words-pie-wan}
\end{figure}
 



\begin{figure}[h!]
  \centering
  \includegraphics[width=6.0087in,height=3.3386in]{figures/chap4/chapitre4-img60.png}
  \caption{
     biaoge : distribution des citations par provinces pour le mot biao (montre)
  }
  \label{fig:biaoge-words-pie-biao}
\end{figure}


Dans le cas du m\`eme biaoge (fig. \ref{fig:biaoge-words-pie-biao} \& \ref{fig:biaoge-words-pie-biao}), nous voyons que les mots principaux
portent une forte diversit\'e et confirme l{\textquoteright}hypoth\`ese
d{\textquoteright}une diffusion g\'eographique plus large de la
discussion.


\begin{figure}[h!]
    \centering

    \subfloat[\textit{zhen}, signifiant {\textquotedblleft}vrai{\textquotedblright}]{
      \includegraphics[scale=0.3]{figures/chap4/chapitre4-img61.png}
    }
    \subfloat[\textit{chang}, signifiant {\textquotedblleft}chanter{\textquotedblright}]{
      \includegraphics[scale=0.3]{figures/chap4/chapitre4-img62.png}
    }
    \newline
    \subfloat[\textit{shengyin}, signifiant "la voix"]{
      \includegraphics[scale=0.5]{figures/chap4/chapitre4-img63.png}
    }
    \caption{
      The Voice : distribution des citations de mots par provinces 
    }
    \label{fig:voice-words-pie}
\end{figure}


\'A l{\textquoteright}inverse, les mots-cl\'es de \textit{The Voice} (fig. \ref{fig:voice-words-pie}) sont
nettement domin\'es par des communaut\'es d{\textquoteright}utilisateurs de P\'ekin ou du Zhejiang qui semblent r\'eellement fixer les termes de la discussion.

\begin{figure}
    \centering
    \subfloat[\textit{zuijin}, signifiant {\textquotedblleft}ces derniers temps{\textquotedblright}]{  
      \includegraphics[scale=0.3]{figures/chap4/chapitre4-img65.png}
    }
    \subfloat[\textit{mang}, signifiant {\textquotedblleft}occupé{\textquotedblright}]{  
      \includegraphics[scale=0.3]{figures/chap4/chapitre4-img66.png}
    }
    \newline
    \subfloat[\textit{dufu}, le nom d'un lettré chinois célèbre]{  
      \includegraphics[scale=0.5]{figures/chap4/chapitre4-img64.png}
    }
    \caption{
      Dufu : distribution des citations de mots par provinces 
    }
    \label{fig:biaoge-words-pie}
\end{figure}

Dans le cas de dufu, les trois mots les plus importants \textit{dufu}, \textit{zuijin} et \textit{mang} qui forment la phrase-m\`eme \textit{dufu est tr\`es occup\'e r\'ecemment }sont tous tr\`es fortement reli\'e \`a la r\'egion de Canton (fig. \ref{fig:biaoge-words-pie}). Cette information, pas forc\'ement pro\'eminente dans les analyses pr\'ec\'edentes, montrent que le d\'eveloppement de ce m\`eme est largement le fait d{\textquoteright}utilisateurs de la r\'egion de Canton.