\chapter*{Introduction}

% 1. contexte : planter le décor

\newthought{Comprendre les modèles régissant les interactions en ligne} est aujourd'hui une question centrale. Managers, employés, industriels, entrepreneurs, politiques, publicitaires ou activistes, tous s'interrogent sur les nouvelles formes de conversations de l'Internet. Les services de réseaux sociaux en ligne sont notamment le lieu d'exploration et d'expérimentations donnant naissance à des phénomènes nouveaux.

Contenus humoristiques circulant rapidement sur la Toile, les \textit{mèmes Internet} sont notamment une des formes les plus typiques des échanges en ligne. Réunissant en un temps très court un grand nombre d'utilisateurs, 
les mèmes et leurs modèles de diffusion restent encore largement méconnus. 

L'échelle et les nouvelles facilités de communication offertes par les réseaux sociaux ont notamment vu ces objets numériques comiques et atypiques investir différents champs des discussions en ligne : blagues potaches {Knobel2007}, campagne marketings \cite{Flor2000}, activisme \cite{Lotan2011},  faits de société \cite{Mina2012} ou même communication politique \cite{Walker2012}.

Friand de sarcasmes et d'humour en tout genre, l'Internet chinois n'est pas exempt de ce phénomène. Le service de microblog \textit{Sina Weibo}, fleuron de l'industrie du web en Chine, s'est largement développé ces dernières années. La réactivité nouvelle de ce média au sein d'un environnement médiatique chinois très contrôlé \cite{MacKinnon2009, Douzet2007} a offert aux internautes chinois une réactivité d'un genre nouveau. 

L'étude des changements apportés par l'apparition de ce service en Chine s'est souvent focalisée sur les enjeux politiques, exprimés par une tension entre  censure gouvernementale \cite{Ng2014} et pratiques de contournement des utilisateurs \cite{Yang2008}. La complexité des relations entre pouvoir politique, industries culturelles, enjeux médiatiques et usage quotidien de cette plate-forme exige néanmoins une lecture plus nuancée.

Ce travail cherche à examiner les différents régimes d'expression et de discours qui régissent les multiples usages de \textit{Sina Weibo}. En nous intéressant autant aux mèmes absurdistes qu'aux campagnes commerciales, nous souhaitons dresser un portrait des modèles de diffusion de différents contenus afin de mieux saisir leurs spécificités.

\subsubsection{Modèles théoriques de la diffusion sur les réseaux sociaux}

% vision globale
Les méthodes d'analyse des activités sur les services de réseaux sociaux en ligne un véritable essor et promet de devenir une des ressources méthodologiques importantes pour de multiples disciplines scientifiques \cite{Nettleton2013}. La compréhension du rôle de ces pratiques de communication sur l'Internet. L'usage de la visualisation de données permettant notamment d'observer les formes de la diffusion des conversations sur le Web \cite{Russell, Adams2012}. Ce type de travaux explorant les dynamiques entourant les discussions en ligne sont notamment un domaine très actif \cite{Nettleton2013} 
 
% la logique sous-jacente du travail 
Pourtant, les assises théoriques utilisés pour la modélisation et l'analyse des conversations d'après des données restent encore peu élaborées. Notamment, la visualisation sous forme de graphes propose d'approcher les conversations en ligne selon un modèle émetteur-récepteur basique pourtant largement invalidé. Rhétorique, discours, énonciation, espaces, code et lieux de paroles, nous considérons dans ce travail différentes lectures et approches théoriques apportées par les théories de la communication et des systèmes d'information, la géographie des technologies et les études sur le langage.

Notre réflexion s'articulera autour de l'idée de \textit{milieu numérique}, construit des protocoles et interfaces technologiques, physiques et institutionnelles du web. L'actualisation de ce milieu numérique par la circulation de formes spécifiques de contenus font partie dessine des motifs particuliers. Mèmes humoristiques, campagnes publicitaires ou scandales politiques peuvent se comprendre selon des structures sémantiques, topologiques et spatio-temporelles particulières, que nous nommons \textit{topogrammes}.

\subsubsection{Outil d'observation et de visualisation de données}

% objet de la recherche 
Afin d'observer dans le détail les topogrammes de différents type d'objets numériques, nous nous proposons donc de concevoir un outil spécialement d'analyse et de visualisation de données. En effet, les outils actuellement disponibles proposent des approches méthodologiques souvent disparates permettant seulement une observation fragmentée des phénomènes de diffusion en ligne. 

% raison d'être de la recherche
Les études existantes dans ce domaine \cite{Leskovec1996, Adamic2014} s'appuient en effet sur l'analogie mécaniste du virus à l'origine de l'idée de \textit{mème} \cite{Dawkins1984, Blackmore2001} pour approcher la diffusion des contenus Internet. Cette naturalisation de l'objet numérique pêchent fige un cadre conceptuel qui pêche à restituer les multiples phénomènes d'appropriation entourant les usages des plate-formes en ligne.

% objectifs
En réunissant un ensemble d'outils déjà existants au sein d'un seul et même dispositif d'analyse et surtout au sein d'un seul et même espace de représentation, nous souhaitons proposer une lecture concomitante des différents aspects de la diffusion.

% l'approche
La conception d'un outil d'analyse et de visualisation situe donc notre travail de rechercheà la croisée d'une approche ingénierique de développement technologique et d'une réflexion méthodologique sur l'application possible de théories existantes à l'analyse de données.

% méthodologie
Afin de mener à bien ce travail, nous allons procéder à l'analyse d'un vaste corpus de plus de 200 millions de messages représentant l'activité de Sina Weibo durant l'année 2012 \cite{Fu2013}. 

En utilisant les technologies de l'analyse des réseaux conversationnels \cite{Weng2013}, du traitement automatique du langage et de la visualisation de données, nous proposons ici de concevoir un outil capable d'extraire puis de donner à voir les dynamiques entourant différents types de contenus circulant sur le plate-forme web Sina Weibo.


\subsubsection{Le cas des mèmes sur Sina Weibo}

% suppositions / postulats
Les spécificités du contexte politico-économique chinois nous amènerons à discuter du rôle de l'industrie médiatique dans la formation des pratiques en ligne. Le cas spécifique des mèmes sur Sina Weibo apportera une lecture fortement contextualisée, nécessaire pour mener à bien une analyse de large volume de données.

% limites et faiblesses de l'expérience
L'identification d'une typologie des modèles de diffusion sur Internet sera donc conditionnée à ce seul exemple. La discussion portera donc sur un nombre limité de mèmes, pas nécessairement représentatif des contenus du Web sur d'autres plate-formes. De plus, les aléas de la disponibilité des données issues des réseaux sociaux et la complexité de la mise en place de méthodes de traitement adaptées seront autant de difficultés supplémentaires pour mener à bien cette étude. 

% prédire l'impact qu'aura votre recherche si tout fonctionne comme prévu
Néanmoins, l'approche pluri-disciplinaire offre l'opportunité d'une réflexion plus large sur les usages et enjeux qui entourent la circulation des contenus sur les Internets. 

La discussion théorique permettra de développer des outils pour analyser les dynamiques multiples entourant la diffusion sur les réseaux sociaux. La production de visualisations, de graphes et de cartes offrira notamment la possibilité d'une observation détaillée puis d'une lecture critique des actes de communication en ligne.

% son domaine d'application
En observant les formes et régimes d'expression entourant les échanges, nous pourrons notamment formuler des recommandations pratiques et méthodologiques pour la création et l'analyse de la circulation des objets digitaux. Ce dispositif technologique sera conçu pour être ré-utilisable sur d'autres terrains, domaines et plate-formes.

\subsubsection{Plan de la thèse}

% l'évolution de sa réalisation
% appréhender le contenu de la thèse.

La première partie de cette thèse présentera l'exemple historique du développement de l'Internet en Chine et plus particulièrement du service de microblog \textit{Sina Weibo}. Nous proposerons le concept de {milieu numérique} pour problématiser les rapports existants entre usages et individuation dans le contexte d'Internet.

La seconde partie discutera la notion de \textit{mème Internet} en proposant une première typologie des différents usages que recouvrent ce mot mystérieux. Sur la base de cette typologie, nous introduirons l'idée de \textit{topogramme} afin de décrire les structures observables lors de la diffusion en ligne.

La partie suivante présentera nos réflexions épistémologiques et méthodologiques relatives à l'analyse et au traitement des données issues des réseaux sociaux. Nous discuterons la possibilité d'identifier des mèmes dans le vaste jeu de données grâce à un protocole expérimental mobilisant différentes approches algorithmiques.

La dernière partie présentera dans le détail les choix retenus lors du développement de notre outil d'analyse et de visualisation. Un échantillon d'une douzaine de mèmes sélectionnés permettra de présenter et discuter les figures obtenues grâce à cet outil et de dresser des conclusions provisoires. \'Egalement, nous nous interrogerons sur la portée et la validité des méthodes d'analyse de données en Science Humaines au regard des résultats de la présente recherche.

% Nous allons maintenant voir comment les outils conceptuels en communication existants peuvent être mobilisés dans le cadre de l'analyse de données pour mieux comprendre comment différents contenus circulent sur la Toile.


% A la fin de l'introduction, le lecteur devrait savoir exactement quel est votre but avec ce document. 
% se termine à la question de recherche et à l'énonciation de la thèse ou l'hypothèse. 

%l'hypothèse ou de l'énonciation de la thèse devrait finaliser l'introduction.

% légitimez votre travail en tant qu'élément essentiel de la recherche dans ce domaine
