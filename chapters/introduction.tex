\chapter*{Introduction}

% 1. contexte : planter le décor

\newthought{Parler, partager, commenter, discuter, écrire} sont les actes quotidiens qui font circuler les informations sur l'Internet. Du silicium des serveurs aux écrans des téléphones, des millions de messages se frayent chaque jour des chemins insoupçonnés à travers les multiples plate-formes du Web. Soutenues par l'appétit d'une industrie florissante, ces nouvelles formes de conversation viennent interroger et renouveler les pratiques médiatiques, politiques, scientifiques et managériales.

Les \textit{mèmes Internet}, petites blagues circulant rapidement sur la Toile, réunissent autour d'eux un grand nombre d'utilisateurs en un temps très court. Satire politique, action commerciale ou simple blague potache, la vélocité et le pouvoir fédérateur de ces simples photos légendées ne cessent de surprendre. Les modèles de leur diffusion, encore largement méconnus, ont jusqu'ici largement été compris par analogie à ceux du virus biologique. Pourtant, l'observation des tensions entourant l'appropriation de ces puissantes instances médiatiques donne à voir une réalité bien différente.

Ces objets numériques atypiques ont également pénétré les fenêtres des navigateurs Internet en Chine, portés par le sarcasme et l'humour. Le service de microblog \textit{Weibo} lancé par le portail \textit{Sina} a connu un dès son lancement en 2008 un succès fulgurant. La réactivité de ce service de publication instantanée a démultiplié les espaces de conversation en ligne. Rassemblant aujourd'hui plusieurs centaines de millions d'utilisateurs, cette plateforme a amené de nouvelles pratiques de la discussion au sein d'un environnement médiatique chinois traditionnellement très contrôlé \citep{MacKinnon2009, Douzet2007, Yang2008}

La croissance rapide de \textit{Sina Weibo} a été largement soutenue par la politique du gouvernement central de Pékin. Le protectionnisme strict appliqué au secteur des industries culturelles et des Technologies de l'Information et de la Communication (TIC) a notamment permis à la firme de se développer dans un environnement non-concurrentielle. Ses homologues américains \textit{Twitter} et \textit{Facebook} ont en effet été bannis du paysage chinois, rendus inaccessibles depuis l'intérieur du territoire \citep{Sullivan2012}. La valorisation importante sur les marchés d'affaires du titre \textit{Sina}\footnote{D'après \url{http://finance.yahoo.com/echarts\?s=SINA+Interactive\#symbol=SINA;range=5y}, consulté le 5 Juillet 2014 à 11h10} témoigne du succès économique et commercial de l'entreprise. Néanmoins, l'évolution des régulations et utilisations du service lui-même témoigne des tensions constantes entre agenda gouvernemental, désir d'expression des utilisateurs et objectif de rentabilité.

% objectifs
La présente recherche se propose d'examiner les différents régimes d'expression et de discours régissant les usages de \textit{Sina Weibo} grâce à l'étude des contenus à forte circulation - dont les mèmes. Les travaux concernant cette plateforme se polarisent souvent autour des actions de marketing, des stratégies de censure gouvernementale \citep{Ng2013a} ou des tactiques de contournement développées par les utilisateurs \citep{Yang2014}. Nous souhaitons ici dépasser ce clivage en considérant la complexité des relations entre pouvoir politique, industries culturelles et pratiques quotidiennes afin de proposer une lecture plus contrastée \citep{Fernandez2010}.

Pour ce faire, nous dresserons un portrait de différents contenus web sur \textit{Sina Weibo} comprenant mèmes absurdistes, scandales politiques, débats de société et campagnes commerciales. Un outil d'analyse et de visualisation de données spécialement conçu pour l'occasion nous permettra d'observer dans le détail les interactions entre plate-formes, mots, lieux et utilisateurs lors de leur diffusion.


\subsubsection{Modèles théoriques de la diffusion sur les réseaux sociaux}

% vision globale
L'analyse des activités humaines grâce aux informations disponibles en ligne promet de devenir une ressource méthodologique importante dans de multiples champs scientifiques \citep{Schreibman2007}. Stockées en des lieux et formats pas toujours aisément accessibles, ces fragments s'amoncèlent pour former d'innombrables bases de données. Parfois désignées par le mot-valise de \textit{Big Data} \citep{Lohr2012a}, ces traces matérielles pourraient permettre une forme nouvelle d'archéologie des phénomènes humains. L'accès et le traitement de cette mémoire technologique nécessitent un renouvellement de l'écriture scientifique, utilisant à la fois les langages humains et informatiques \citep{ Guichard2014}. La fiabilité des méthodes et la pertinence des découvertes issue de ce vaste champ d'expérimentations restent encore largement à construire \citep{Boyd2011}.

Les données produites par l'usage des services de réseaux sociaux offrent notamment une opportunité nouvelle pour l'étude des pratiques de communication \citep{Zook2007,Nettleton2013,Manovich2011}. La disponibilité en grande quantité de matériels retraçant les échanges quotidiens les plus divers augure de nouvelles formes d'investigation et d'analyse. Depuis la sociométrie de \cite{Moreno1938}, le modèle des réseaux a progressivement structuré la lecture des phénomènes sociaux \citep{Latour1999a, Castells1989}. Guidé par les théories de la complexité \citep{Morin1990}, cette science des réseaux émergente \citep{Brandes2013} trouve un terrain de prédilection dans l'analyse des activités sur les réseaux sociaux en ligne. L'observation des dynamiques entourant la diffusion des conversations en ligne fait notamment l'objet de nombreux travaux d'une grande variété méthodologique : modélisation statistique \citep{Steyer2006}, cartographie \citep{Conover2013,Eisenstein2012},  simulation \citep{Tubaro2010}, etc. 

% la logique sous-jacente du travail
L'analyse des données conversationnelles issues des réseaux sociaux souffre cependant d'assises théoriques encore peu élaborées. La visualisation de conversations sous forme de graphes d'utilisateurs est notamment très répandue dans les travaux sur la communication en ligne. Cette schématisation modélise pourtant les échanges entre individus selon un schéma émetteur-récepteur très basique et largement décrié \citep{Proulx2000}. Les réflexions théoriques et méthodologiques sur l'analyse communicationnelle des données issues de réseaux sociaux émanent le plus souvent de disciplines connexes comme la géomatique \citep{Crampton2013, Leetaru2013} ou l'informatique \citep{Brodka2013, Russel2011}. Notre travail s'efforce à situer ces pratiques dans une perspective historique. Par conséquent, nous avons été amené à faire dialoguer des approches conceptuelles diverses (espaces, territoires, lieux, rhétorique, discours, énonciation, code...) issues de champs disciplinaires multiples (communication, gestion et systèmes d'information, géographie des technologies, histoire du langage, etc.)

% raison d'être de la recherche
Notre réflexion est conduite autour de l'idée de \textit{milieu numérique}, construit des protocoles et interfaces technologiques, physiques et institutionnels du web \cite{Hui2012}. L'actualisation de ce milieu numérique par la circulation de formes spécifiques de contenus dessine des motifs particuliers que nous nommons \textit{topogrammes}. Ces topogrammes sont connaissables par l'observation des propriétés des structures sémantiques, topologiques et spatio-temporelles représentant le mouvement des informations dans le réseau. Mèmes humoristiques, campagnes publicitaires ou scandales politiques, chaque contenu peut être abordé par la forme particulière de son topogramme, expression de son ou ses milieux numériques.


\subsubsection{Outil d'observation et de visualisation de données}

% suppositions / postulats
Le terme de \textit{mème}, défini initialement comme une unité de diffusion culturelle, trouve son origine dans une analogie avec le gène biologique \citep{Dawkins1989, Blackmore2001}. La dimension évolutionniste sous-jacente du mot et l'absence de domaines d'application l'ont cantonné pendant longtemps aux marges de la culture scientifique \citep{Jouxtel2014}. Remis au goût du jour avec l'Internet, les mèmes ont depuis été le sujet de plusieurs études, utilisant souvent le virus comme modèle idéal de la diffusion des contenus sur les réseaux sociaux \citep{Leskovec2005, Adamic2014}. L'idée d'une transmission mécanique par contact représente peu la réalité entrant en jeu dans l'appropriation d'objets technologiques \citep{Orlikowski1993} ou informationnels \citep{Certeau1980}. De plus, naturaliser des éléments de culture commune en entités autonomes nous semble non seulement réducteur mais aussi dangereux \citep{Elias1975}.

% méthodologie % approche
Ainsi, nous nous éloignons de la description virale du mème pour observer en détail les topogrammes de différents objets numériques en concevant un dispositif de visualisation de données. À la croisée de l'ingénierie et de la réflexion méthodologique, nous utilisons les technologies de l'analyse des réseaux conversationnels \citep{Weng2012}, du traitement automatique de la langue chinoise \citep{Xue2003} et de la visualisation d'information \citep{Cairo2012}. Nous sélectionnons des contenus de différents types sous la forme de jeux de messages extraits d'un vaste corpus de 200 millions d'interactions retranscrivant l'activité de \textit{Sina Weibo} durant l'année 2012 \citep{Fu2013}. Notre dispositif permet ensuite de montrer différents aspects de leur circulation au sein d'un seul et même espace de représentation.

% son domaine d'application
Cette réflexion s'ancre dans le contexte politico-économique chinois avec le cas particulier de \textit{Sina Weibo}. Nous discuterons du rôle de l'industrie médiatique et des enjeux politiques agissant dans la formation des pratiques en ligne. Nous chercherons également à formuler des recommandations pratiques et méthodologiques concernant la création et l'analyse d'objets digitaux à forte diffusion. Notre dispositif technologique est spécialement conçu avec pour finalité la ré-utilisation sur d'autres terrains, domaines et plate-formes.


\subsubsection{Plan de la thèse}

% l'évolution de sa réalisation
% appréhender le contenu de la thèse.
La première partie de cette thèse présentera l'exemple historique du développement de l'Internet en Chine et plus particulièrement du service de microblog \textit{Sina Weibo}. Nous introduirons le concept de \textit{milieu numérique} pour problématiser les rapports existants entre protocoles et usages dans le contexte d'Internet.

La seconde partie discutera la notion de \textit{mème Internet} en proposant une première typologie des différents usages exprimés par ce mot mystérieux. Nous introduirons l'idée de \textit{topogramme} afin de décrire les structures observables lors de la diffusion en ligne.

La partie suivante introduira des réflexions épistémologiques et méthodologiques relatives à l'analyse et au traitement de larges jeux de données issues des réseaux sociaux. Nous discuterons la possibilité d'identifier des mèmes grâce à un protocole expérimental mobilisant différentes approches algorithmiques.

La dernière partie présentera dans le détail les choix retenus lors du développement de notre outil d'analyse et de visualisation. Un échantillon d'une douzaine de mèmes sélectionnés permettra de présenter et discuter les figures obtenues grâce à cet outil. Nous établirons des catégories d'après les modèles de circulation des contenus sur \textit{Sina Weibo}. \'Egalement, nous nous interrogerons sur la portée et la validité des méthodes d'analyse de données en sciences humaines au regard des résultats de la présente recherche.
