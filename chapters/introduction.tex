\chapter*{Introduction}

% 1. contexte : planter le décor

\newthought{}
Les nouvelles formes de discussions sont un enjeu central pour l{\textquoteright}industrie et le politique, mais aussi au quotidien dans les relations aux travail ou entre amis.  


Les services de réseaux sociaux en Chine

En Chine notamment, la tension constante entre enjeux politiques (Ng, 2014), impératifs économiques (Dai, 2007) et inventivité des utilisateurs du web (Yang, 2008) font des services de réseaux sociaux en ligne un espace d{\textquoteright}énonciation unique. %Unique en quoi ? 

Fleuron de l{\textquoteright}industrie du web en Chine, le service de microblog \textit{Sina Weibo }propose notamment la possibilité d{\textquoteright}une expression pour les utilisateurs gr\^ace à une réactivité d{\textquoteright}un genre nouveau au c{\oe}ur d{\textquoteright}un environnement médiatique chinois très contrôlé (MacKinnon, 2009 ; Douzet, 2007). 

Les \textit{mèmes }Internet, souvent envisagés par leurs contenus humoristiques {\textquotedblleft}viraux{\textquotedblright}, se propagent notamment très rapidement en véhiculant des messages les plus divers : faits de société (Lotan et al., 2011 ; Mina, 2012), blagues potaches (Knobel \&Lankshear, 2007), campagnes marketings (Flor, 2000) ou m\^eme communication politique (Walker, 2012).


\subsubsection{Outils d'analyse des réseaux sociaux}

% vision globale
Les méthodes d{\textquoteright}analyse des activités en ligne plus particulièrement sur les réseaux sociaux,  un véritable essor et promet de devenir une des ressources méthodologiques importantes pour de multiples disciplines scientifiques \cite{Nettleton2013}. La compréhension du rôle de ces pratiques de communication sur l{\textquoteright}Internet 

L{\textquoteright}usage de la visualisation de données permettant d{\textquoteright}observer les formes de la diffusion des conversations sur le Web (Russell, ; Adams, 2012). Ce type de travaux explorant les dynamiques entourant les discussions en ligne sont notamment un domaine très actif \cite{Nettleton2013} 
 
% la logique sous-jacente du travail
les contenus produits par les utilisateurs en ligne comme des actes d{\textquoteright}énonciation (Austin, ..; De Certeau, 1983) et à ce titre considérant les aspects sémantiques, relationnels, territoriaux et temporels des conversations sur les réseaux sociaux. 

Géo, code/space, réflexions au-delà d'une vision hors-sol.
énonciation/discours/discursivités


% objet de la recherche 

Dans ce travail, nous allons chercher à comprendre comment les procédés méthodologiques d{\textquoteright}analyse de données sont capables d{\textquoteright}appréhender aborder et observer ces objets numériques complexes au moyens de l'analyse de données.


%Il faut préciser en quoi cette question des formes est importante 
%Et préciser l eneju d étudier la question de l énonciation que tu introduis juste après 
% son domaine d'application. 
% reject null hypothesis (H0) : a hypothesis which the researcher tries to disprove, reject or nullify. 

\subsubsection{Importance de la recherche}
% 2  importance : 
% raison d'être de la recherche, vos objectifs et votre méthodologie, prédire l'impact qu'aura votre recherche si tout fonctionne comme prévu

Faiblesse des modèles théoriques utilisés dans l'étude de la communication, et plus notamment dans le domaine spécifique de l'analyse de données.

Le concept de \textit{mème} (Dawkins ,1984; Blackmore, 2001) et à plus forte raison l{\textquoteright}étude de la diffusion de mèmes Internet se fonde sur l{\textquoteright}analogie mécaniste du virus (Leskovec, 1996, Adamic \& al., 2014)

ignorant ainsi les multiples phénomènes d{\textquoteright}appropriation entourant les réalités d{\textquoteright}usages des plateformes en lignes.

L'espace de représentation construit autour de l'analogie du réseau et la visualisation des échanges en ligne sous forme de graphes 
produit une vision simpliste et naturalisante qui correspond peu à la réalité des usages de l'Internet.

les utilisateurs sont des points interagissant gr\^ace à des traits ou des flèches. Ce modèle \textit{{\textquotedblleft}émetteur -{\textgreater} message -{\textgreater} récepteur{\textquotedblright} }

pourtant ces modèles ont largement été critiqué par près d{\textquoteright}un siècle de théories de la communication (Katz, 19.. ;Jakobson, 19..; Hall, 19.. ).

% légitimez votre travail en tant qu'élément essentiel de la recherche dans ce domaine
La lecture critique de visualisations, de graphes et de cartographies proposera notamment une approche méthodologique originale pour observer et analyser la diffusion de l{\textquoteright}information en ligne.

introduction du concept de topgrammes pour représenter
interrogation sur les limites du modèle de visualisation actuel

Les outils actuellement disponibles proposent des approches méthodologiques disparates pour des observations fragmentaires qui pêchent à restituer les nuances et les interactions en jeu entre les différents aspects de la diffusion.

Doc, développement d'outils capables d'observer précisément les différentes dynamiques présentes dans les actes de communication en ligne.

En réunissant un ensemble d'outils déjà existants au sein d'un seul et même dispositif d'analyse et surtout au sein d'un seul et même espace de représentation, nous souhaitons proposer une lecture concomitante des différents aspects de la diffusion.


\subsubsection{Méthodologie expérimentale et analyse de données}

% l'approche
La conception d{\textquoteright}un outil d{\textquoteright}analyse et de visualisation situe donc notre travail de rechercheà la croisée d'une approche ingénierique de développement technologique et d'une réflexion méthodologique sur l'application possible de théories existantes à l'analyse de données.

En utilisant le cas spécifique des mèmes sur Sina Weibo, nous souhaitons également proposer une lecture fortement contextualisée, partie intégrante d'une méthodologie d'analyse des données qui se veut pluri-disciplinaire.

Les spécificités du Web chinois nous amène à réfléchir au rôle joué par le contexte politique et économique dans la formation de pratiques des acteurs industriels que sont les plate-formes de réseaux sociaux en ligne.

Enfin, le type de contenus particuliers que nous avons choisi d'étudier nous donne l'opportunité d'une réflexion sur les régimes discursifs de conversation qui entoure la circulation des contenus sur les Internets.

% à completer

Afin de mener à bien ce travail, nous allons procéder à l'analyse d'un vaste corpus de plus de 200 millions de messages représentant l{\textquoteright}activité de Sina Weibo durant l{\textquoteright}année 2012 \cite{Fu2013}. 

En utilisant les technologies de l'analyse des réseaux conversationnels \cite{Weng2013}, du traitement automatique du langage et de la visualisation de données, nous proposons ici de concevoir un outil capable d'extraire puis de donner à voir les dynamiques entourant différents types de contenus circulant sur le plate-forme web Sina Weibo.

% limites : souligner les faiblesses de l'expérience, les défauts, qui permettent de juger de la validité de la recherche 
% Suppositions / postulats

nous allons donc construire ce dispositif méthodologique et technologique nous permettant d{\textquoteright}observer les différents dimensions des mèmes.


\subsubsection{Plan de la thèse}

% l'évolution de sa réalisation
% appréhender le contenu de la thèse.

La première partie de cette thèse présentera l'exemple historique du développement de l{\textquoteright}Internet en Chine et plus particulièrement du service de microblog \textit{Sina Weibo}. Nous proposerons le concept de {milieu numérique} pour problématiser les rapports existants entre usages et individuation dans le contexte d'Internet.

La seconde partie discutera la notion de \textit{mème Internet} en proposant une première typologie des différents usages que recouvrent ce mot mystérieux. Sur la base de cette typologie, nous introduirons l'idée de \textit{topogramme} afin de décrire les modèles et intentions de diffusion sous-jacents

La partie suivante présentera différentes réflexions épistémologiques et méthodologiques relatives à l'analyse et au traitement des données issues des réseaux sociaux. Une protocole expérimental mobilisant différentes approches algorithmiques s'interrogera sur la possibilité d'identifier des mèmes dans le vaste jeu de données. 

La dernière partie présentera dans le détail l'outil finalement développé lors de ce travail. Un échantillon d'une douzaine de mèmes sélectionnés permettra de présenter et discuter les figures obtenues lors de l'analyse et de dresser des conclusions provisoires. \'Egalement, nous nous interrogerons sur la portée et la validité des méthodes d'analyse de données en Science Humaines au regarder des résultats de la présente recherche.

Nous allons maintenant voir comment les outils conceptuels en communication existants peuvent être mobilisés dans le cadre de l'analyse de données pour mieux comprendre comment différents contenus circulent sur la Toile.


% A la fin de l'introduction, le lecteur devrait savoir exactement quel est votre but avec ce document. 
% se termine à la question de recherche et à l'énonciation de la thèse ou l'hypothèse. 

%l'hypothèse ou de l'énonciation de la thèse devrait finaliser l'introduction.
