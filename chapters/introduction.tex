\chapter*{Introduction}

% 1. contexte : planter le décor

\newthought{Parler, partager, commenter, discuter, écrire} sont les pratiques quotidiennes qui fondent la circulation des informations sur l'Internet. Du silicium des serveurs aux écrans des téléphones, des millions de messages se frayent chaque jour des chemins insoupçonnés au travers des multiples plate-formes du Web. Soutenues par l'appétit d'une industrie florissante, ces nouvelles formes de conversation viennent interroger les pratiques médiatiques, politiques, scientifiques et managériales. 

Les \textit{mèmes Internet}, petites blagues circulant rapidement sur la Toile, réunissent autour d'eux un grand nombre d'utilisateurs en un temps très court. Satire politique, action commerciale ou simple blague potache, la vélocité et le pouvoir fédérateur d'une photo de chat légendée agissent selon des modèles encore largement méconnus. En Chine notamment, le sarcasme et l'humour en tout genre ont permis à ces objets numériques atypiques de pénétrer les fenêtres des navigateurs Internet.

La réactivité permise par les services de réseaux sociaux a vu se multiplier des discussions d'un genre nouveau au sein d'un environnement médiatique chinois traditionnellement très contrôlé \citep{MacKinnon2009, Douzet2007, Yang2008}. Le service de microblog \textit{Weibo} lancé par le portail \textit{Sina} connaît notamment un succès fulgurant depuis son lancement en 2008.

Le présent travail se propose d'examiner les différents régimes d'expression et de discours qui régissent les usages de \textit{Sina Weibo}. Les études concernant l'Internet chinois se polarisent souvent sur la tension entre censure gouvernementale \citep{Ng2013a} et pratiques de contournement des utilisateurs \citep{Yang2014}. Dans cette recherche, nous souhaitons aborder plus profondément la complexité des relations entre pouvoir politique, industries culturelles et usages quotidiens qui propose une lecture contrastée des formes de discussions \citep{Fernandez2010}.

Nous allons donc dresser un portrait des différents modèles entourant la  diffusion sur \textit{Sina Weibo} de \textit{mèmes Internet} absurdistes, mais également de scandales politiques, débats de société ainsi que de campagnes commerciales. Nous proposons de concevoir un outil d'analyse et la visualisation de données afin d'observer dans le détail les interactions entre plate-formes, mots, lieux et utilisateurs.


\subsubsection{Modèles théoriques de la diffusion sur les réseaux sociaux}

% vision globale
L'analyse des activités humaines grâce aux informations disponibles en ligne promet de devenir une ressource méthodologique importante dans de multiples champs scientifiques \citep{Schreibman2007, Guichard2014}. Les données produites par l'usage des services de réseaux sociaux offrent notamment une nouvelle opportunité d'étude des pratiques de communication sur l'Internet \citep{Zook2007,Nettleton2013,Manovich2011}. L'observation des formes de la diffusion des conversations devient une approche possible pour comprendre les dynamiques entourant les discussions en ligne \citep{Conover2013,Leetaru2013}.
 
% la logique sous-jacente du travail 
Les assises théoriques utilisées dans la modélisation et l'analyse de ces conversations d'après des données restent encore faibles et peu élaborées. La visualisation sous forme de graphes propose d'approcher les conversations en ligne selon un modèle émetteur-récepteur basique pourtant largement décrié \citep{Proulx2000}. Rhétorique, discours, énonciation, espaces, territoires, code et lieux, différentes lectures théoriques peuvent être empruntées aux théories de la communication et des systèmes d'information, à la géographie des technologies et aux études sur le langage.

Notre réflexion s'articule autour de l'idée de \textit{milieu numérique}, construit des protocoles et interfaces technologiques, physiques et institutionnels du web. L'actualisation de ce milieu numérique par la circulation de formes spécifiques de contenus dessine des motifs particuliers que nous nommons \textit{topogrammes}. Mèmes humoristiques, campagnes publicitaires ou scandales politiques peuvent se comprendre selon leurs structures sémantiques, topologiques et spatio-temporelles.

\subsubsection{Outil d'observation et de visualisation de données}

% raison d'être de la recherche
Les études majeures utilisent l'analogie mécaniste du virus à l'origine de l'idée de \textit{mème} \citep{Dawkins1989, Blackmore2001} pour modéliser la diffusion des contenus sur les réseaux sociaux \citep{Leskovec2005, Adamic2014}. Afin de dépasser cette naturalisation que nous pensons réductrice, nous proposons donc de concevoir un dispositif d'analyse et de visualisation de données permettant d'observer dans le détail les topogrammes de différents type d'objets numériques. 

% approche et objectifs
Notre travail de recherche se situe à la croisée d'une approche ingénierique de développement technologique et d'une réflexion méthodologique sur l'application possible de théories existantes à l'analyse de données. En réunissant un ensemble d'outils existants dans un dispositif d'analyse, nous souhaitons porter un regard sur les phénomènes de diffusion en ligne au sein d'un seul et même espace de représentation.

\subsubsection{Le cas des mèmes sur Sina Weibo}

% méthodologie
Nous utilisons les technologies de l'analyse des réseaux conversationnels \citep{Weng2012}, du traitement automatique de la langue chinoise \citep{Xue2003} et de la visualisation de données \citep{Cairo2012} afin d'observer les dynamiques entourant différents types de contenus dans un vaste corpus de plus de 200 millions de messages représentant l'activité de \textit{Sina Weibo} durant l'année 2012 \citep{Fu2013}. 

% suppositions / postulats
Cette réflexion sera contextualisée autour du cas de \textit{Sina Weibo}. Les spécificités du contexte politico-économique chinois nous amèneront à discuter du rôle de l'industrie médiatique dans la formation des pratiques en ligne. Néanmoins, notre dispositif technologique est ré-utilisable sur d'autres terrains, domaines et plate-formes. 

% son domaine d'application
L'étude de \textit{Sina Weibo} nous limitera à un type de mèmes particuliers. Pourtant, nous chercherons à formuler des recommandations pratiques et méthodologiques concernant la création et l'analyse de la circulation des objets digitaux sur la base des formes et régimes d'expression observés.  

\subsubsection{Plan de la thèse}

% l'évolution de sa réalisation
% appréhender le contenu de la thèse.

La première partie de cette thèse présentera l'exemple historique du développement de l'Internet en Chine et plus particulièrement du service de microblog \textit{Sina Weibo}. Nous proposerons le concept de \textit{milieu numérique} pour problématiser les rapports existants entre protocoles et usages dans le contexte d'Internet.

La seconde partie discutera la notion de \textit{mème Internet} en proposant une première typologie des différents usages exprimés par ce mot mystérieux. Nous introduirons l'idée de \textit{topogramme} afin de décrire les structures observables lors de la diffusion en ligne.

La partie suivante introduira des réflexions épistémologiques et méthodologiques relatives à l'analyse et au traitement de larges jeux de données issues des réseaux sociaux. Nous discuterons la possibilité d'identifier des mèmes grâce à un protocole expérimental mobilisant différentes approches algorithmiques.

La dernière partie présentera dans le détail les choix retenus lors du développement de notre outil d'analyse et de visualisation. Un échantillon d'une douzaine de mèmes sélectionnés permettra de présenter et discuter les figures obtenues grâce à cet outil. Nous établirons des catégories d'après les modèles de circulation des contenus sur \textit{Sina Weibo}. \'Egalement, nous nous interrogerons sur la portée et la validité des méthodes d'analyse de données en sciences humaines au regard des résultats de la présente recherche.

% Nous allons maintenant voir comment les outils conceptuels en communication existants peuvent être mobilisés dans le cadre de l'analyse de données pour mieux comprendre comment différents contenus circulent sur la Toile.


% A la fin de l'introduction, le lecteur devrait savoir exactement quel est votre but avec ce document. 
% se termine à la question de recherche et à l'énonciation de la thèse ou l'hypothèse. 

%l'hypothèse ou de l'énonciation de la thèse devrait finaliser l'introduction.

% légitimez votre travail en tant qu'élément essentiel de la recherche dans ce domaine
