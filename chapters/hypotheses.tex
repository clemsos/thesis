% pages de transition entre les ch1/2 et ch3 en forme de conclusion de cette première partie de ton travail consacrée à une revue de la littérature de ton sujet et à l'aboutissement de la construction de ta problématique

Thèse :


\chapter*{Problématique et hypothèses}
% une problématique reprécisée (à l'aune de ces deux chapitres) 

\section{Problématique} 
Les particularités du Web chinois nous ont d{\textquoteright}abord  permis de considérer dans cette étude les dimensions économiques  et politiques des usages des réseaux sociaux, avec notamment  l{\textquoteright}aspect industriel de la production de contenus web.  Toutefois, nous avons choisis de nous intéresser à un type  d{\textquoteright}usage bien particulier : les mèmes Internet. Ces  petits objets numériques connaissent souvent une grande diffusion et  offre une grande diversité dans les sujets qu{\textquoteright}ils  abordent : humour, faits divers, scandales politiques, campagnes  marketing, etc.    

\section{Hypothèses}

\subsubsection{Hypothèse 1 (méthodologique) analyse enrichie de la diffusion} 

En considérant un ensemble de ces mèmes, nous pouvons identifier des récurrences dans la structure de leur diffusion en représentant sous formes de graphes multiples appelés \textit{topogrames} non seulement les échanges entre utilisateurs (conversation), mais également les structures lexicales (sémantique), temporelles et géographiques.

\subsubsection{Hypothèse 2 : réseaux sociaux = mass media classiques} 

Qu{\textquoteright}il s{\textquoteright}agisse de propagande politique ou de campagnes publicitaires, la dimension stratégique de la diffusion de contenus fait des médias un enjeu important pour tout système économique et politique. Nous pensons qu{\textquoteright}il existe de grandes similarités de diffusion entre les contenus majoritaires des services de réseaux sociaux et ceux des médias plus classiques comme la télévision ou la radio qui sont soumis à des contextes de production souvent similaires.

\subsubsection{Hypothèse 3: classification des discussions en ligne par leurs structures} 

topogrames

L{\textquoteright}analyse de la diffusion de discussions peut permettre de déterminer différents modèles de conversations d{\textquoteright}après les caractéristiques de leurs topogrames, notamment des conversations ouvertes à la participation dites \textit{associatives }et d{\textquoteright}autres dites \textit{dissociatives} qui autorisent plus difficilement l{\textquoteright}échange. L{\textquoteright}identification de ces modèles peut permettre une classification des discussions selon les modalités de leurs diffusions.  

\subsubsection{Hypothèse 4: milieu numérique et fragmentation (small worlds)} 

L{\textquoteright}observation des modèles de diffusion de conversations en ligne permet de comprendre la nature dissociative ou associative de leur \textit{milieu numérique, }compris comme l{\textquoteright}ensemble de protocoles et outils technologiques qui les produit (Hui, 2012). Nous pensons notamment que le design actuel des services de réseaux sociaux est largement dissociatif et conduit ainsi à une fragmentation accrue des conversations et relations en ligne (structure en small worlds), notamment à cause de la segmentation de l{\textquoteright}attention nécessaires pour la valorisation publicitaire de son modèle d{\textquoteright}affaire.  


\section{Méthodologie suivie} 
% le choix d'une méthodologie qui vise à explorer les hypothèses retenues

Afin d{\textquoteright}étudier ces différentes hypothèses, nous avons choisi d{\textquoteright}analyser 




