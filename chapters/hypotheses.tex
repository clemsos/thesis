\chapter*{Problématique et hypothèses}


% pages de transition entre les ch1/2 et ch3 en forme de conclusion de cette première partie de ton travail consacrée à une revue de la littérature de ton sujet et à l'aboutissement de la construction de ta problématique
% une problématique reprécisée (à l'aune de ces deux chapitres) 


\section*{Objet et problématique} 

% Contexte
L'apparition en Chine d'une industrie dédiée aux contenus des réseaux sociaux s'inscrit dans la continuité historique du développement industriel et politique du Web chinois. Les nombreuses discussions hébergées par le service de microblog \textit{Sina Weibo} recoupent souvent blagues, faits divers, scandales politiques, marketing politique et commercial, etc.

% Objet
Parmi ces nombreux contenus échangés sur les réseaux sociaux, les \textit{mèmes Internet} offre une forme typique faisant appel à l'humour pour mobiliser des références culturelles communes. La vitesse et l'échelle de leur circulation en font des objets numériques intéressants pour explorer les nouvelles interactions ayant lieu sur les plate-formes Web. 

% Thèse
\'A mi-chemin entre écriture et oralité, l’appellation de mèmes Internet recouvre pourtant des réalités diverses qui semblent procéder de motivations et de régimes d'expression forts différents. La naturalisation implicite du modèle ``viral'' souvent invoqué nous empêche de considérer l'ensemble hétéroclite de pratiques qui se déroulent lors la diffusion d'informations en ligne. Seule une observation détaillée peut nous permettre d'apporter un éclairage sur les régimes d'expression qui président à la production de ces contenus.

% Question(s) de recherche
Comment s'opère la diffusion des mèmes sur \textit{Sina Weibo}? Quelles sont les particularités de ce type d'objets par rapport à d'autres formes de contenus plus traditionnels? Comment pouvons-nous observer ces mèmes Internet? Quelles relations entretiennent-ils avec le milieu numérique qui les produit? Ont-ils une existence territoriale?

% Problématique
Afin d'observer les différents types de contenus qui circulent sur la plate-forme de réseau social \textit{Sina Weibo}, nous allons procéder au développement d'un dispositif utilisant l'analyse et la visualisation de données. La série d'hypothèses infra accompagnera la démarche méthodologique de conception de cet outil.

\section*{Hypothèses}
\label{sec:hypotheses}

\subsubsection{La majorité des contenus circulant sur les réseaux sociaux s'apparentent largement à ceux des médias traditionnels} 

L'objet numérique communément appelé dans la littérature \textit{mème Internet} recouvre différents types de contenus qui ne s'apparentent pas à ce terme. La plupart sont en effet des formes narratives, commerciales ou journalistiques qui pré-existent à l'apparition d'Internet. Qu{\textquoteright}il s{\textquoteright}agisse de propagande politique ou de campagnes publicitaires, la dimension stratégique de la diffusion médiatique est un enjeu important pour tout système économique et politique. Nous pensons qu{\textquoteright}il existe de grandes similarités de diffusion entre les contenus majoritaires des services de réseaux sociaux et ceux des médias plus classiques comme la télévision ou la radio, soumis à des contextes de production industrielle similaires.

\subsubsection{Les mèmes sont à comprendre comme des actes d'énonciation.}

Si les formes de contenus des médias traditionnels trouvent une continuité sur les réseaux sociaux, la fonctionnalité nouvelle de ce type de service est la possibilité de converser en ligne. Reconductions de phénomènes anciens le mème Internet peut se comprendre comme une rumeur, un débat ou une blague à forte circulation. Ces formes archétypales se constituent autour de pratiques essentiellement orales, existantes aujourd'hui sous le régime particulier de l'écriture en ligne. Ainsi, l'étude des mèmes Internet nécessite de mobiliser un cadre conceptuel adéquat considérant ces actes de communication comme formes de l'énonciation.


\subsubsection{L'analyse de la diffusion ne doit pas se contenter de l'analyse conversationnelle, mais doit mobiliser les modèles étudiant l'énonciation.}

Près d'un siècle de recherche en sciences de la communication nous ont permis de développer des modèles d'analyse complexes des actes de communication. Pour les échanges en ligne, nous ne pouvons nous contenter de percevoir uniquement la dimension conversationnelle sans prendre en compte les multiples aspects non-verbaux, para-verbaux ou même sémantiques qui entrent en jeu dans les actes de communication en ligne. Notre première approche cherchera à comprendre non seulement les échanges entre utilisateurs (conversation), mais également les structures lexicales (sémantique), temporelles et géographiques qui émergent de ces échanges. L'analyse de la diffusion des mèmes nous servira de base pour discuter et comprendre comment se structurent ces actes de communication en ligne.


\subsubsection{L'analyse et la visualisation de données permettent d'établir une classification des discussions en ligne selon les structures de leurs diffusions.}

La disponibilité d'immenses volume de conversations sous la forme de vaste jeu de données issues des réseaux sociaux nous permet aujourd'hui de pratiquer de nouveaux types d'analyse pour comprendre les phénomènes entourant la communication en ligne. Nous visualisons la diffusion d'un ensemble de mèmes issus du réseau social \textit{Sina Weibo} sous formes de graphes nommés \textit{topogrammes}. Cette observation permet de distinguer et classifier les mèmes d'autres formes de contenus en ligne pour dresser une typologie des modèles de conversations.

\subsubsection{La circulation des contenus sur les réseaux sociaux accroît la fragmentation du milieu numérique.}

L{\textquoteright}observation de ces topogrammes permet de comprendre comment les conversations en ligne actualisent le \textit{milieu numérique}, compris comme l{\textquoteright}ensemble d'outils et protocoles technologiques qui les produit \citep{Hui2012}. \'Egalement, nous pouvons identifier des formes de communication et d'échanges plus ouvertes à la participation dites \textit{associatives} et d{\textquoteright}autres dites \textit{dissociatives}autorisant plus difficilement l{\textquoteright}échange. Nous pensons notamment que la segmentation de l{\textquoteright}attention suscitée par la valorisation publicitaire des contenus induit un design largement dissociatif des services de réseaux sociaux, qui conduit à une fragmentation accrue des relations en ligne (une structure en \textit{small worlds}).

\bigskip
% le choix d'une méthodologie qui vise à explorer les hypothèses retenues

Afin d{\textquoteright}approcher ces différentes hypothèses, nous avons choisi de nous livrer à la conception d'un outil nous permettant d'analyser de larges jeux de données issus de \textit{Sina Weibo}. La dimension expérimentale de ce travail nous amènera à effectuer de nombreux choix méthodologiques que nous allons maintenant discuter.