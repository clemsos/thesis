\chapter*{Problématique et hypothèses}
% pages de transition entre les ch1/2 et ch3 en forme de conclusion de cette première partie de ton travail consacrée à une revue de la littérature de ton sujet et à l'aboutissement de la construction de ta problématique
% une problématique reprécisée (à l'aune de ces deux chapitres) 


\section{Objet et problématique} 

% Thèse :
Les mèmes Internet sont une forme de contenus possédant de nombreuses particularités. Nous avons besoin de développer un nouvel outil afin de pouvoir les observer dans le détail et de mieux les connaître.

% Objet
Les particularités du Web chinois nous ont d{\textquoteright}abord  permis de considérer dans cette étude les dimensions économiques  et politiques des usages des réseaux sociaux, avec notamment  l{\textquoteright}aspect industriel de la production de contenus web.  Toutefois, nous avons choisis de nous intéresser à un type  d{\textquoteright}usage bien particulier : les mèmes Internet. Ces  petits objets numériques connaissent souvent une grande diffusion et  offre une grande diversité dans les sujets qu{\textquoteright}ils  abordent : humour, faits divers, scandales politiques, campagnes  marketing, etc.

% Question(s) de recherche
Comment s'opère la diffusion des mèmes sur Sina Weibo? A plus forte raison, quelles sont les particularités de ce type de contenu par rapport à d'autres contenus plus traditionnels? Comment pouvons-nous observer ces mèmes Internet? Quelles relations entretiennent-t-ils avec le milieu numérique qui les produit? Ont-t-ils une existence territoriale.

% Problématique
Nous nous proposons ici de considérer différents types de contenus à forte circulation sur la plate-forme de réseau social chinoise Sina Weibo et d'observer les particularités de leurs diffusions. Plus particulièrement, nous souhaitons considérer les mèmes Internet. Néanmoins,la définition fluctuante de ces petits objets numériques nous oblige à imaginer une méthodologie particulière pour les observer qui passe par la création d'un outil dédié à cet usage.

\section{Hypothèses}
\label{sec:hypotheses}

Au regard de ce que nous avons discuté précédemment, nous formulons plusieurs hypothèses qui  vont appuyer la conception de cet outil.

\subsubsection{La majorité des contenus circulant sur les réseaux sociaux s'apparentent largement à ceux des médias traditionnels} 

L'objet numérique communémént appelé mème Internet dans la littérature recouvre tout type de contenus qui ne s'apparent pas à ce terme. La plupart des contenus du web sont en effet des formes narratives, commerciales ou journalistiques qui pré-existent déjà bien avant l'Internet.Qu{\textquoteright}il s{\textquoteright}agisse de propagande politique ou de campagnes publicitaires, la dimension stratégique de la diffusion de contenus fait des médias un enjeu important pour tout système économique et politique. Nous pensons qu{\textquoteright}il existe de grandes similarités de diffusion entre les contenus majoritaires des services de réseaux sociaux et ceux des médias plus classiques comme la télévision ou la radio qui sont soumis à des contextes de production souvent similaires.

\subsubsection{Les échanges en ligne sont à comprendre comme des actes d'énonciation, dont les mèmes sont l'expression la plus actuelle.}

Si les contenus médiatique ont été majoritairement reconduit dans la création des réseaux sociaux, reste néanmoins une fonctionnalité primordiale et nouvelle qui est celle de la conversation en ligne. Comme nous l'avons vu, les mèmes peuvent être considéré d'un point de vue ethnographique comme la reconduction de phénomènes. La rumeur ou la blague sont notamment des formes archétypales qui nous permettent de considérer les mèmes Internet comme des phénomènes pas nécessairement ou entièrement nouveau. Néanmoins, contrairement à la tradition écrite de la publicité, du tract politique ou du journalisme, ces phénomènes constituent des réalités essentiellement \textit{orales}. Ainsi, nous nous devons de mobiliser un cadre conceptuel adéquat si nous souhaitons nous intéresser aux mèmes Internet. Cette dimension éminemment orale nous amène donc à considérer les actes de communication entourant les mèmes sous le régime de l'énonciation (par opposition à celui de l'écriture notamment). Plus encore, nous pensons que les échanges prenant place sur les réseaux sociaux procèdent de ce régime de l'énonciation.


\subsubsection{L'analyse de la diffusion ne doit pas se contenter de l'analyse conversationnelle, mais doit mobiliser les modèles étudiant l'énonciation} 

Près d'un siècle de recherche en Sciences de la Communication nous ont permis de développer des modèles d'analyse complexe des actes de communication. Pour les échanges en ligne, nous ne pouvons nous contenter de percevoir uniquement la dimension conversationnelle sans prendre en compte les multiples aspects non-verbaux, para-verbaux ou même sémantiques qui rentrent en jeu dans les actes de communication en ligne. La parole en ligne ou hors-ligne est un acte et doit à ce titre être comprise comme telle. Une première approche que nous proposerons ici cherchera à comprendre non seulement les échanges entre utilisateurs (conversation), mais également les structures lexicales (sémantique), temporelles et géographiques qui émergent de ces échanges. L'analyse de la diffusion des mèmes nous servira de base pour discuter et comprendre comment se structurent ces actes de communication en ligne.


\subsubsection{L'analyse et la visualisation de données permette d'établir une classification des discussions en ligne selon les structures de leurs diffusions.}

La disponibilité d'immenses volume de conversations sous la forme de vaste jeu de données issues des réseaux sociaux nous permet aujourd'hui de pratiquer de nouveaux types d'analyse pour comprendre les phénomènes entourant la communication en ligne. En analysant un jeu de données issu du réseau social Sina Weibo, nous pensons pouvoir observer des éléments particuliers et constitutifs entrant en jeu dans la diffusion des mèmes et d'autres formes de contenus en ligne. En représentant sous formes de graphes multiples appelés \textit{topogrames} différents aspects de leur diffusion, l{\textquoteright}analyse des figures visualisant ces discussions peut notamment permettre de déterminer différents modèles de conversations d{\textquoteright} caractéristiques.Les topogrames pourraient notamment permettre d'identifier des modèles pour classifier des discussions selon les modalités de leurs diffusions. 

\subsubsection{La circulation des contenus sur les réseaux sociaux accroît la fragmentation du milieu numérique.}

en définissant des circuits dans le réseau

L'étude des topogrammes obtenus peut également permettre d'identifier des conversations ouvertes à la participation dites \textit{associatives }et d{\textquoteright}autres dites \textit{dissociatives} qui autorisent plus difficilement l{\textquoteright}échange. 

L{\textquoteright}observation des modèles de diffusion de conversations en ligne permet de comprendre la nature dissociative ou associative de leur \textit{milieu numérique, }compris comme l{\textquoteright}ensemble de protocoles et outils technologiques qui les produit (Hui, 2012). Nous pensons notamment que le design actuel des services de réseaux sociaux est largement dissociatif et conduit ainsi à une fragmentation accrue des conversations et relations en ligne (structure en small worlds), notamment à cause de la segmentation de l{\textquoteright}attention nécessaires pour la valorisation publicitaire de son modèle d{\textquoteright}affaire.  


% le choix d'une méthodologie qui vise à explorer les hypothèses retenues
\section{Méthodologie d'analyse et conception d'un outil} 

Afin d{\textquoteright}approcher ces différentes hypothèses, nous avons choisi de nous livrer à l'analyse d'un large jeu de données issus de Sina Weibo. Pour mener à bien cette tâche, nous avons décidé de développer un outil car aucune solutions existantes ne permettait l'observation des mèmes de façon satisfaisante. Afin de nous permettre de poursuivre l'exploration des hypothèses que nous nous sommes fixées, nous allons maintenant aborder dans le détail les aspects méthodologiques et les expérimentations qui nous permis de mener à bien ce travail.