\chapter*{Problématique et hypothèses}


% pages de transition entre les ch1/2 et ch3 en forme de conclusion de cette première partie de ton travail consacrée à une revue de la littérature de ton sujet et à l'aboutissement de la construction de ta problématique
% une problématique reprécisée (à l'aune de ces deux chapitres) 


\section*{Objet et problématique} 

% Contexte
L'apparition d'une industrie des contenus pour les réseaux sociaux en Chine s'inscrit dans la continuité historique du développement économique et politique du Web chinois. Le services de microblog \textit{Sina Weibo} héberge notamment de nombreuses discussions recoupant souvent blagues, faits divers, scandales politiques, marketing politique et commercial, etc.

% Objet
Parmi les nombreuses contenus échangés sur les réseaux sociaux, les \textit{mèmes Internet} offre une forme typique faisant appel à l'humour pour mobiliser des références culturelles communes. La vitesse et les vastes dimensions de leur circulation en font des objets numériques intéressants pour explorer les nouvelles interactions ayant lieu sur les plate-formes Web. 

% Thèse
\'A mi-chemin entre écriture et oralité, l’appellation de mèmes Internet recouvre pourtant des réalités diverses qui semblent procéder de motivations et de régimes d'expression forts différents. Le modèle ``viral'' souvent invoqué nous empêche de considérer l'ensemble hétéroclite de pratiques qui se déroulent lors la diffusion d'informations en ligne. Seule une observation détaillée peut nous permettre d'apporter un éclairage sur les différents régime d'expression qui préside à la production des contenus largement diffusés en ligne.

% Question(s) de recherche
Comment s'opère la diffusion des mèmes sur \textit{Sina Weibo}? A plus forte raison, quelles sont les particularités de ce type d'objets par rapport à d'autres formes de contenus plus traditionnels? Comment pouvons-nous observer ces mèmes Internet? Quelles relations entretiennent-t-ils avec le milieu numérique qui les produit? Ont-t-ils une existence territoriale?

% Problématique
Afin d'observer les différents types de contenus qui circulent sur la plate-forme de réseau social chinoise \textit{Sina Weibo}, nous allons procéder au développement d'un dispositif utilisant l'analyse et la visualisation de données. La série d'hypothèses infra accompagnera la démarche méthodologique de conception de cet outil.

\section*{Hypothèses}
\label{sec:hypotheses}

\subsubsection{La majorité des contenus circulant sur les réseaux sociaux s'apparentent largement à ceux des médias traditionnels} 

L'objet numérique communémént appelé mème Internet dans la littérature recouvre tout type de contenus qui ne s'apparente pas à ce terme. La plupart des contenus du web sont en effet des formes narratives, commerciales ou journalistiques qui pré-existent déjà bien avant l'Internet. Qu{\textquoteright}il s{\textquoteright}agisse de propagande politique ou de campagnes publicitaires, la dimension stratégique de la diffusion de contenus fait des médias un enjeu important pour tout système économique et politique. Nous pensons qu{\textquoteright}il existe de grandes similarités de diffusion entre les contenus majoritaires des services de réseaux sociaux et ceux des médias plus classiques comme la télévision ou la radio qui sont soumis à des contextes de production industrielle souvent similaires.

\subsubsection{Les mèmes sont à comprendre comme des actes d'énonciation.}

Si les formes de contenus des médias traditionnels trouvent se retrouvent sur les réseaux sociaux, la fonctionnalité nouvelle de ce type de service est la possibilité de converser en ligne.Reconductions de phénomènes anciens les mèmes Internet peuvent se comprendre comme une rumeur, un débat ou une blague à forte circulation. Ces formes archétypales nous se constituent autour de pratiques essentiellement orales, existants aujourd'hui sous le régime particulier de l'écriture en ligne. Ainsi, l'étude des  mèmes Internet nécessite de mobiliser un cadre conceptuel adéquat considérant ces actes de communication sous le régime de l'énonciation.


\subsubsection{L'analyse de la diffusion ne doit pas se contenter de l'analyse conversationnelle, mais doit mobiliser les modèles étudiant l'énonciation.}

Près d'un siècle de recherche en Sciences de la Communication nous ont permis de développer des modèles d'analyse complexe des actes de communication. Pour les échanges en ligne, nous ne pouvons nous contenter de percevoir uniquement la dimension conversationnelle sans prendre en compte les multiples aspects non-verbaux, para-verbaux ou même sémantiques qui rentrent en jeu dans les actes de communication en ligne. La parole en ligne ou hors-ligne est un acte et doit à ce titre être comprise comme tel. Une première approche que nous proposerons ici cherchera à comprendre non seulement les échanges entre utilisateurs (conversation), mais également les structures lexicales (sémantique), temporelles et géographiques qui émergent de ces échanges. L'analyse de la diffusion des mèmes nous servira de base pour discuter et comprendre comment se structurent ces actes de communication en ligne.


\subsubsection{L'analyse et la visualisation de données permettent d'établir une classification des discussions en ligne selon les structures de leurs diffusions.}

La disponibilité d'immenses volume de conversations sous la forme de vaste jeu de données issues des réseaux sociaux nous permet aujourd'hui de pratiquer de nouveaux types d'analyse pour comprendre les phénomènes entourant la communication en ligne. En analysant un jeu de données issu du réseau social Sina Weibo, nous pensons pouvoir observer des éléments particuliers et constitutifs entrant en jeu dans la diffusion des mèmes et d'autres formes de contenus en ligne. En représentant sous formes de graphes multiples appelés \textit{topogrames} différents aspects de leur diffusion, nous pouvons classifier ces discussions selon des modèles caractéristiques de conversations.

\subsubsection{La circulation des contenus sur les réseaux sociaux accroît la fragmentation du milieu numérique.}

L{\textquoteright}observation des modèles de diffusion de conversations en ligne permet de comprendre comment s'actualise le \textit{milieu numérique}, compris comme l{\textquoteright}ensemble de protocoles et outils technologiques qui les produit \citep{Hui2012}. L'étude de ces topogrammes peut également permettre d'identifier des modèle plus ouverts à la participation dites \textit{associatives }et d{\textquoteright}autres dites \textit{dissociatives} qui autorisent plus difficilement l{\textquoteright}échange. Nous pensons notamment que la segmentation de l{\textquoteright}attention suscité par la valorisation publicitaire des contenus induit un design largement dissociatif des services de réseaux sociaux, qui conduit à une fragmentation accrue des relations en ligne (structure en small worlds).

\bigskip
% le choix d'une méthodologie qui vise à explorer les hypothèses retenues

Afin d{\textquoteright}approcher ces différentes hypothèses, nous avons choisi de nous livrer à la conception d'un outil nous permettant d'analyser de larges jeux de données issus de \textit{Sina Weibo}. La dimension expérimentale de ce travail nous amènera à effectuer de nombreux choix méthodologiques que nous allons maintenant discuter.