\chapter{Protocole de recherche}

\section*{Calendrier et étapes de la recherche}

Le tableau suivant retrace les grandes étapes du déroulement de cette recherche.

{\small
\begin{table}[h!]
\begin{tabulary}{\linewidth}{ l C C C C}

    \textbf{Tâches} &  
    \textbf{Date} & 
    \textbf{Rendu} &
    \textbf{Encadrement} \\
    
    \hline \\[-1.2ex]

    \textbf{2012}  & & & \\

    21 février & 
    Inscription en doctorat à ParisTech Telecom &
    Paris & 
    Mme Valérie \sc{Fernandez} \& M. Gilles \sc{Puel} \\

    12 Avril & 
    Première présentation publique &
    Paris & 
    ParisTech Telecom \\

    28 Mai & 
    CaiYuan Pei 2012&
    Toulouse-Nanjing & 
    M. \sc{Zhen} Feng \\
    
    Juin-Octobre& 
    Social Brain Framework (SNS mining)&
    Online & 
    Sharism Lab \& M. Isaac \sc{Mao} \\

    Octobre-Décembre & 
    Séjour de recherche 1 &
    Nanjing / Shanghai & 
    M. \sc{Zhen} Feng \\

    10-12 Novembre & 
    Meeting CaiYuan Pei &
    Université de Nanjing & 
    Equipe Cai Yuanpei \\

    8 Décembre & 
    TEDx Wuxi &
    Wuxi & 
    TEDx \\

    \textbf{2013}  & & & \\

    Février & 
    Echidna project  &
    Shanghai & 
    Ricky Ng-Adam (Xinchejian) \\

    22-24 Mai & 
    AIM2013 Big Data  &
    Lyon & 
    Colloque \\

    6 Juin & 
    Journées i3 &
    Paris & 
    Institut Mines-Telecom \\

    Août-Octobre & 
    Séjour de recherche 3 &
    Nanjing / Shanghai & 
    M. \sc{Zhen} Feng \\

    Septembre & 
    Analyse langue chinoise  &
    Beijing & 
    Yuan Mingli \& \textit{Guokr} \\

    20 Septembre & 
    Talk at Anti-Tectonics  &
    Beijing & 
    Beijing Design Week \\

    10-30 Septembre & 
    Smart City Exhibition  &
    Beijing & 
    Beijing Design Week \\

    Novembre-Décembre  & 
    Collectes de données   &
    Lyon & 
    Atelier des Médias \\

    \textbf{2014}  & & & \\

    Janvier & 
    Algorithmique et détection  &
    Lyon & 
    Adrien \sc{Guille} et Fabien \sc{Veclin} (Lyon 2)\\

    Février-Avril & 
    Développement de l'outil  &
    Lyon & 
    Atelier des Médias \\

    Janvier-Avril & 
    Séminaire \textit{Atelier Internet Lyonnais}  &
    Lyon & 
    \'Eric Guichard (ENSSIB) \\

    Mai-Juin & 
    Ecriture thèse  &
    Lyon & 
    Atelier des médias \\

    Mai & 
    AIM2014  &
    Aix-en-Provence & 
    Colloque AIM \\

    Juin & 
    Chinese Internet Research Conference  &
    Hong Kong & 
    Polytechnic HK \\

    Septembre & 
    Soutenance thèse  &
    Paris & 
    EDITE / ParisTech Telecom \\
\end{tabulary}
\end{table}}


% Vérifier la 

% Où est la preuve > ces algortihmes sont-ils pertinents ?


% - équipes 
% - noms de chercheurs que tu as cotoyés dans ce domaine de l'ingénierie informatique 
% - périodes de temps durant lesquelles tu y as travaillé 
% - principaux jalons de l'étude 
% - dates / durées de  traitement
% - supports informatiques 
