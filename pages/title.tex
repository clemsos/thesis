% *-----------------------------------------------------------*
% *         Page de Garde (basé sur le modèle UPMC)
% * Dernière modif le : 2012/02/13
% *-----------------------------------------------------------*
% * README: 
% *   - se compile avec pdflatex
% *   - l'entête de ce fichier sert pour générer une page
% *     unique. Ce texte latex peut s'inclure (partie comprise 
% *     entre les lignes et  ) dans un document existant.
% *-----------------------------------------------------------*

% \thispagestyle{empty}
\setlength{\textwidth}{16cm}
\setlength{\textheight}{25cm}
\setlength{\oddsidemargin}{-0cm}
\setlength{\topmargin}{-1cm}
\title{}     
\date{} 
\author{}
% \begin{document}

% - - - - - - - début de la page 
\thispagestyle{empty}

\includegraphics[scale=0.12]{images/logoTPT.png}

{\large

\vspace*{1cm}

\begin{center}

{\bf TH\`ESE DE DOCTORAT}

\vspace*{0.5cm}

Sp\'ecialit\'e \\ [2ex]
{\bf Systèmes d'information}\ \\ 

\vspace*{0.5cm}

École doctorale Informatique, Télécommunications et Électronique (Paris)

\vspace*{1cm}


Pr\'esent\'ee par \ \\


\vspace*{0.5cm}


{\Large {\bf Clément RENAUD}}

\vspace*{1cm}
Pour obtenir le grade de \ \\[1ex]
{\bf DOCTEUR de l'UNIVERSIT\'E PIERRE ET MARIE CURIE} \ \\

\vspace*{1cm}

\end{center}

\flushleft{Sujet de la th\`ese :\ \\
\ \\
{\Large {\bf Le titre de ma Thèse \\ }}

\vspace*{1.5cm} 
\flushleft{soutenue le 30 février 2042}\\[2ex]
\flushleft{devant le jury composé de :  }\\[1ex]
\flushleft{\begin{tabular}{r@{\ }ll}
  & Mme. Valérie {\sc Fernandez} & Directeur de thèse\\
  & M. Gilles {\sc Puel} & Examinateur  \\
  & M. Prénom {\sc Nom} & Rapporteur \\
  & M. Prénom {\sc Nom} & Rapporteur  \\
  & M. Prénom {\sc Nom} & Examinateur  \\
\end{tabular}}

}