% the abstract

% % \selectlanguage{english} % début du passage en typographie anglaise
Les discussions en ligne se diffusent autour de dynamiques multiples, à la fois conversationnelles et géographiques mais également temporelles et sémantiques. Sur le service de réseau social chinois Sina Weibo, l’activité intense s’articule quotidiennement entre discussions personnelles et campagnes médiatiques portées par de grands groupes médiatiques. Les mèmes Internet, contenus web éphémères et très prisés, circulent à une cadence très soutenue entre les utilisateurs du service selon des modèles de diffusion encore peu connus. 

Afin d’explorer ces structures de propagation, nous avons procéder à l’analyse et la visualisation de différents mèmes au sein d’un échantillon de 200 millions de messages postés sur Sina Weibo durant l’année 2012. Une première série d’analyses nous a amené à écarter les hashtags, représentatifs d’usages majoritairement commerciaux et stratégiques. Nous avons ensuite sélectionné une dizaine de mèmes aux intentions et contenus divers d’après la littérature scientifique et secondaire : mèmes humoristiques, actualité, scandale politique, faits divers et marketing promotionnel. 

La visualisation sous formes de graphes des aspects temporels, sémantiques, géographiques et conversationnels de leurs diffusions nous ont permis d’identifier des caractéristiques propres à chacun des types de mèmes. A plus forte raison, les différents espaces de représentation mis à jour par la visualisation propose une lecture critique des différentes dimensions de la diffusion offrant la possibilité de dégager des lectures stratégiques et analytiques pour les actes de communication d’organismes tant privés que publics.

% % \selectlanguage{frenchb}% retour à la typographie française
% Résumé en français