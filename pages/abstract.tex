% the abstract

% % \selectlanguage{english} % début du passage en typographie anglaise
Nous proposons de concevoir et développer un outil permettant d’analyser la diffusion d’information sur les services de réseaux sociaux en ligne grâce au traitement et à la visualisation de données. Fruit d’une réflexion méthodologique, ce dispositif permet d’observer les relations entre les dimensions conversationnelles, sémantiques, temporelles et géographiques des actes de communication en ligne.

Courts messages se propageant rapidement sur la Toile selon des modèles encore mal connus, les \textit{mèmes Internet} comptent parmi les contenus les plus prisés sur les plate-formes web. Les mèmes Internet circulant sur le service de microblog chinois \textit{Sina Weibo} articulent notamment discussions personnelles, débats sociétaux et vastes campagnes médiatiques.

Mobilisant des méthodes issues de l’analyse des réseaux et du traitement automatisé de la langue chinoise, nous procédons à l’analyse d’un vaste corpus de 200 millions de messages représentant l’activité sur \textit{Sina Weibo} durant l’année 2012. 

Notre première tâche consiste à identifier des mèmes dans ce large ensemble de données. L’identification de mèmes dans un ensemble de messages est notamment possible grâce à un algorithme de détection non supervisé. Néanmoins, le volume de calculs nécessaires pour obtenir des résultats fiables sur un large corpus nous amène à abandonner cette approche, montrant par là-même la complexité d’une définition intéressante de l’objet numérique composite mème. Notre seconde série d’analyses porte sur le volume de conversations entourant les hashtags du corpus. Les résultats montrent que les usages majoritaires de \textit{Sina Weibo} sont similaires à ceux des médias traditionnels (publicité, divertissement, loisirs...). Néanmoins, nous écartons les hashtags comme représentants des mèmes Internet, artefacts d’usages commerciaux et stratégiques à la diffusion cadrée et planifiée. 

L’approche finalement retenue utilise la recherche par mots-clés pour constituer les corpus de messages décrivant une dizaine de mèmes sélectionnés dans la littérature académique et secondaire pour leurs intentions diverses: humour, actualité, scandale politique, faits divers et marketing promotionnel. S’inspirant de la critique des schémas théoriques de communication, une analyse des mots et des réseaux d’échanges entre utilisateurs met à jour les dynamiques discursives de chaque mème. L’organisation de ces informations selon un axe temporel dans un espace de visualisation interactif autorise une lecture détaillée de leur diffusion. La projection de ces réseaux conversationnels et lexicaux sur des cartes géographiques montre également les relations entre leurs aspects textuels et actuels.

Les figures obtenues permettent d’ébaucher une typologie structurelle de la diffusion de ces contenus, montrant comment différents régimes d'expression cohabitent sur les réseaux sociaux. La tension entre énonciation et discours qui régit les plate-formes Web se manifeste dans des motifs particuliers de circulation des contenus en ligne. Nous pouvons ainsi formuler des recommandations pour l’analyse et la conception de stratégies de communication en ligne d’organismes tant privés que publics. Néanmoins, le caractère exploratoire de cette étude et la difficulté de comprendre les actions humaines par une simple analyse de données nous invite à refuser une généralisation a priori des résultats, préférant considérer ce travail comme la première validation d’une méthodologie pouvant être étendue à d’autres formes de conversations en ligne.


% % \selectlanguage{frenchb}% retour à la typographie française
% Résumé en français