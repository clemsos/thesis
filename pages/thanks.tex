% the acknowledgments section



% \begin{flushright}
    \begin{figure}[th!]
    
       \hfill\begin{minipage}{8cm}\centering
        \includegraphics[scale=0.6]{media/nyan_cat_thumb.jpg}
        \caption*{Ce \href{https://www.youtube.com/watch?v=QH2-TGUlwu4}{``Nyan cat''} symbolise la reconnaissance infinie de l'auteur à ses pairs.}
        \end{minipage}
    \end{figure}
% \end{flushright}




\newthought{Loin d'être un travail solitaire}, cette recherche doctorale a été jalonnée de rencontres, de discussions et d'échanges qui lui permettent aujourd'hui de voir le jour. L'auteur tient donc à rendre hommage à celles et ceux qui ont contribué, parfois sans le savoir, à l'écriture de ce document.

En premier lieu, je tiens à remercier mes directeurs de thèse Mme. Valérie {\sc Fernandez} et M. Gilles {\sc Puel} à qui a incombé la lourde tâche de guider mes pas sur la route biscornue que j'avais choisie d'emprunter. Merci pour le soutien et les conseils avisés. Sans leurs précieuses relectures et questionnements, je serai sûrement encore en train de m'ébahir devant l'immensité des sujets qu'il me reste à traiter.

Merci également à : 
\begin{itemize}
    \item {\sc{Zhen}} Feng, {\sc{Xi}} Guangliang et l'Institut d'Urbanisme de l'Université de Nanjing pour  nos échanges lors du projet Cai Yuanpei.
    \item Isaac {\sc{Mao}}, {\sc{Yuan}} Mingli, {\sc{Liu}} Yong et l'équipe du \textit{Sharism Lab} pour les longues discussions sur l'analyse de données et les réseaux sociaux chinois.
    \item Fabien {\sc{Veclin}} et Adrien {\sc{Guille}} pour les discussions algorithmiques.
    \item Cédric {\sc{Sam}}, {\sc{Cong}} Ding et {\sc{Qi}} Gao pour les jeuc de données.
    \item Gabriele {\sc{de}} Seta pour nos échanges épistolaires et épistémologiques sur les mèmes et Michelle {\sc{Proksell}} pour nous avoir présenté
    \item  Min {\sc{Lin}}, Ricky {\sc{Ng-Adam}}, David {\sc{Li}} et l'équipe de \textit{Transi.st} pour le temps-réel, le Javascript et la magie au hackerspace \textit{Xinchejian}.
    \item Thierry {\sc{Joliveau}}, Eric {\sc{Guichard}} et Clément {\sc{Levallois}} et les participants du séminaire AIL (\textit{Atelier Internet Lyonnais}) de Lyon pour l'énergie.
    \item Jon {\sc{Philips}}, Lionel {\sc{Radisson}} et Nicolas {\sc{Maigret}} pour l'amour du code et du travail bien fait.
    \item An Xiao {\sc{Mina}}, Denis {\sc{Servant}} et Pascal {\sc{Jouxtel}} pour l'attention portée aux mèmes.
    \item Bernhard {\sc{Rieder}} et Nicolas {\sc{Kaiser-Bril}} pour l'analyse sans faille des réseaux.
    \item Yann {\sc{Charles}} pour m'avoir mis le pied à l'étrier.
    \item Yuk {\sc{Hui}} pour le défrichage du milieu numérique.
    \item {\sc{Liu}} Yan, Yang {\sc{Lei}}, Ingrid {\sc{Fischer-Schreiber}} et Silvia {\sc{Lindtner}} pour leur enthousiasme.
    \item Ivan {\sc{Zhai}}, Cui {\sc{Anyong}} et Yolanda {\sc{Ma}} pour le futur du journalisme en Chine.
    \item {\sc{Zhang}} Yixuan, {\sc{Tong}} Xin et l'équipe d'\textit{Urban China}, Jean-François {\sc{Doulet}} et les membres du site \textit{Villes Chinoises} pour toutes les réflexions sur l'urbanité en Chine.
    \item {\sc{Lu}} Rui, Federico {\sc{Cecchini}}, Habib {\sc{Belaribi}}, Marie {\sc{Bellot}}, Nicolas {\sc{Guillot}}, David {\sc{Dusa}}, Emilie {\sc{Blézat}} pour les moments passés ensembles à Pékin, Shanghai ou ailleurs.
    \item Tous les co-workers de l'\textit{Atelier des Médias} pour l'ambiance unique qui a rendu tolérable mon quotidien de doctorant
    \item Ma femme, mes amis et ma famille pour leur soutien inconditionnel.
\end{itemize}

Enfin, je souhaiterai remercier spécialement les millions d'internautes pour les heures passées devant leurs écrans à poster des messages en apparence idiots et inutiles, mais qui m'ont néanmoins donné suffisamment de fil à retordre pour mener à bien ce travail de thèse.

\bigskip
Cette recherche a bénéficié du soutien du partenariat Hubert Curien avec la Chine dans le cadre du programme \textit{Cai Yuanpei} (\zh{蔡元培}) intitulé \textit{Cities and information sharing : social and mobile networks in urban context} (2012-2014).