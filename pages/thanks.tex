% the acknowledgments section


\begin{figure}[th!]*
  \raggedleft
  \begin{minipage}{8cm}
  \includegraphics[scale=0.6]{media/nyan_cat_thumb.jpg}
  \caption*{Un \href{https://www.youtube.com/watch?v=QH2-TGUlwu4}{``Nyan cat''} symbolise ici la reconnaissance infinie de l'auteur à ses pairs.}
  \end{minipage}
\end{figure}


\newthought{Précieux et salutaires}, les nombreux soutiens reçus depuis plusieurs mois permettent aujourd'hui à ce travail de thèse de voir le jour. Son auteur tient donc à remercier celles et ceux qui ont contribué, parfois sans le savoir, à l'écriture de ce document.

En premier lieu, je tiens à remercier mes directeurs de thèse Mme. Valérie {\sc Fernandez} et M. Gilles {\sc Puel} à qui à incomber la lourde tâche de guider mes pas sur la route biscornue que j'avais chois d'emprunter. Merci pour les relectures, les mails interminables et les conseils avisés sans lesquels je serai encore sûrement en train de m'ébahir devant l'immensité des livres qu'il me reste à lire.

Merci également à : 
\begin{itemize}
    \item {\sc{Zhen}} Feng, {\sc{Xi}} Guangliang et l'Institut d'Urbanisme de l'Université de Nanjing pour  nos échanges lors du projet Cai Yuanpei.
    \item Thierry {\sc{Joliveau}}, Eric {\sc{Guichard}} et Clément {\sc{Levallois}} pour le séminaire AIL (\textit{Atelier Internet Lyonnais}) à l'\textit{ENSSIB}.
    \item Isaac {\sc{Mao}}, {\sc{Yuan}} Mingli, {\sc{Liu}} Yong et l'équipe du \textit{Sharism Lab} pour les longues discussions sur l'analyse de données et les réseaux sociaux chinois.
    \item Fabien {\sc{Veclin}} et Adrien {\sc{Guille}} pour les discussions algorithmiques.
    \item Gabriele {\sc{de}} Seta pour nos échanges épistolaires et épistémologiques sur les mèmes et Michelle {\sc{Proksell}} pour nous avoir présenté
    \item Ricky {\sc{Ng-Adam}}, Min {\sc{Lin}} et l'équipe de \textit{Transi.st} pour le temps-réel, le Javascript et l'ambiance du hackerspace.
    \item Jon {\sc{Philips}}, Lionel {\sc{Radisson}} et Nicolas {\sc{Maigret}} pour l'amour du code et du travail bien fait.
    \item An Xiao {\sc{Mina}}, Denis {\sc{Servant}} et Pascal {\sc{Jouxtel}} pour l'attention portée aux mèmes.
    \item Bernhard {\sc{Rieder}} et Nicolas {\sc{Kaiser-Bril}} pour l'analyse des réseaux sociaux.
    \item Yann {\sc{Charles}} pour m'avoir mis le pied à l'étrier.
    \item Yuk {\sc{Hui}} pour le défrichage du milieu numérique.
    \item {\sc{Liu}} Yan, Yang {\sc{Lei}}, Ingrid {\sc{Fischer-Schreiber}} et Silvia {\sc{Lindtner}} pour la curiosité et la bienveillance.
    \item Cui {\sc{Anyong}} pour l'enthousiasme et le futur du journalisme en Chine
    \item {\sc{Zhang}} Yixuan, {\sc{Tong}} Xin et l'équipe d'\textit{Urban China}, Jean-François {\sc{Doulet}} et les membres du site \textit{Villes Chinoises} pour toutes les réflexions sur l'urbanité en Chine.
    \item {\sc{Lu}} Rui, Federico {\sc{Cecchini}}, Habib {\sc{Belaribi}}, Marie {\sc{Bellot}}, Nicolas {\sc{Guillot}}, David {\sc{Dusa}}, Emilie {\sc{Blézat}} pour les moments passés ensembles à Pékin ou Shanghai.
    \item Tous les co-workers de l'\textit{Atelier des Médias} pour l'ambiance unique qui a rendu tolérable mon quotidien de doctorant
    \item Ma femme, mes amis et ma famille pour leur soutien inconditionnel.
\end{itemize}

Enfin, je souhaiterai remercier spécialement les millions d'internautes pour les heures passées devant leurs écrans à poster des messages en apparence idiots et inutiles, mais qui me permettent aujourd'hui de concrétiser cette thèse.